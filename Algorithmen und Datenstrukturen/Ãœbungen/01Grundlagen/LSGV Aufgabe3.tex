\documentclass{article}
\begin{document}
	\section*{Lsg Vorschlag ADS Ü01 A3 Maximilian Maag}
	\section*{Aufgabenteil a}
	\begin{itemize}
		\item Der Algorithmus durchläuft bei gleicher Eingabe, z.B. 1, immer die gleichen Zustände und ist damit deterministisch.
		\item Da der Algorithmus die strengeren Bedingungen für Determinismus erfüllt determiniert er auch.
		\item Der Algorithmus terminiert für alle Eingaben x $\in$ R $|$ x $\leq$ 1.
		Für diese Eingaben greift die Abbruchbedingung x $\leq$ 1 in Zeile 2.
		\item $f^{alg} x \in R \to x \in N$ 
		\item$f^{alg}(x) = \log_{3}(x)$
	\end{itemize}
	\section*{Aufgabenteil b}
	\begin{itemize}
		\item Der Algorithmus durchlauft bei gleicher Eingabe unterschiedliche Zustände, bedingt durch die Zufallszahl r die bei jedem Durchlauf zufällig bestimmt wird. Er ist also nicht deterministisch.
		\item Das Ergebnis bleibt bei gleicher Eingabe nicht dasselbe, der Algorithmus determiniert nicht.
		\item Der Algorithmus ist weder deterministisch noch determinierend, daher ist der Inhalt der Funktion nicht bestimmbar.
		
	\end{itemize}
	
\end{document}