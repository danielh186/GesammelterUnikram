\documentclass{article}
\usepackage{mathtools}
\begin{document}
	\section*{Lsg Vorschlag ADS Ü02 A2 Maximilian Maag}
	\subsection*{Aufgabe a}

	

	\subsection*{Aufgabe b}
	Zu zeigen ist durch Folgekonvergenz: $n^3 
	\notin O(n^2 + n +4)$  \\ \\
	$f_n = n^3$; $g_n = n^2 + n + 4$ \\
	$\frac{f_n}{g_n} \lim\limits_{x \to \infty}$ = $\frac{n^3}{n^2 + n + 4}$ \\
	$\frac{f_n}{g_n} \lim\limits_{x \to \infty}$ = $\frac{n * n^2}{n(n + \frac{4}{n})}$ \\
	$\frac{f_n}{g_n} \lim\limits_{x \to \infty}$ = $\frac{n^2}{n + \frac{4}{n}}$ \\
	$\frac{f_n}{g_n} \lim\limits_{x \to \infty}$ = $\frac{n^2}{n + \frac{4}{n}}$ \\
	$\frac{f_n}{g_n} \lim\limits_{x \to \infty}$ = $\infty$ \\ \\
	Der Grenzwert der beiden Folgen ist ungleich 0 daher gilt $n^3 \notin O(n^2 + n + 4)$.
	\subsection*{Aufgabe c}
	$a_n \in o(b_n)$; $b_n \in o(c_n)$ \\ \\
	Daraus folgt: \\
	$c_n > b_n$ \\
	$b_n > a_n$  \\
	\\
	Daraus wiederum folgt: \\
	$a_n < b_n < c_n$ \\
	Geschuldet durch Transitivität: \\
	$a_n < c_n \equiv a_n \in o(c_n)$ 
\end{document}