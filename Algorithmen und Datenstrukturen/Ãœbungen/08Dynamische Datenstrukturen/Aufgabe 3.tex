\documentclass{article}
\begin{document}
	\section*{Lsg Vorschlag ADSÜ08 A3 Maximilian Maag}
	
	\subsection*{Aufgabe 1}
	Unter der Voraussetzung, ein Baum sei zyklenfrei und vollständig lassen sich folgende Überlegungen treffen: \\
	\begin{itemize}
		\item Bei einer höhe von 2 gehen von der Wurzel zu n Knoten je eine Kante also n Kanten.
		\item Erhöht man die Höhe um 1 müssen nun von n Knoten n Kanten zu n Kinderknoten je Knoten n gezogen werden.
		\item Wiederholt man dies h-mal um auf  die Höhe des Baumes zu kommen müsste man h $\cdot$ (n $\cdot$ n $\cdot$ n ......... n) rechnen.
	\end{itemize}
	Aus den obigen Überlegungen ergibt sich dann für einen Baum der Höhe h eine Anzahl von Kanten k = $n^{h}$.
	\subsection*{Aufgabe 2}
	Die Anzahl der Knoten eines vollständigen Binärbaumes, der Höhe h, lassen sich durch $2^{h} - 1$ bestimmten. \\ \\
	Zu Zeigen ist: $n = 2^{h} - 1$ \\ \\
	IV: \\ \\
	Für die Eingabe 1 an der Wurzel ergibt sich: \\ \\
	$2^{1} - 1$ = 1 \\ \\
	Dieses Ergebnis hätten wir auch erwartet, da ein Baum der Höhe 1 nur die Wurzel als einzigen Knoten besitzt. \\ \\
	IS: h $\to$ h+1 \\
	$2^{h} - 1 = n$ \\
	$2^{h+1} - 1 = 2n$ + 1 \\
	$2^{h+1} - 1 = 2(2^{h} - 1)$ + 1 \\
	$2^{h+1} - 1 = 2 \cdot 2^{h} - 2$  + 1\\
	$2^{h+1} - 1 = 2^{h+1} - 2$  + 1\\
	$2^{h+1} - 1 = 2^{h+1} - 1$ \\
	q.e.d
	
\end{document}