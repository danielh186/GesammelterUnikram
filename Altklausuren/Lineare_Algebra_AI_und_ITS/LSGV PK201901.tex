\documentclass{article}
\usepackage{mathtools}
\begin{document}
	\section*{LSGV PK201901 Maag}
	\subsection*{Aufgabe 1}
	x = ha Kartoffeln; y = ha Zuckerrüben \\
	1. $x + y \leq 90$ \\
	2. $y \geq 50$ \\
	3. $400x + 200y \leq 24000$ \\
	4. $G = 450x + 150y$ \\
	5. $x \geq 0; y \geq 0$ \\ \\
	Als LGS \\
	A1: $x + y \leq 90$ \\
	B1: $y \geq 50$ \\
	C1: $4x + 2y \leq 240$ \\ \\
	B1 in A1: \\
	A2: $x + 50 \leq 90$ \\
	A3: $x \leq 40$ \\
	B1 in C1 \\
	C2: $4x + 2 \cdot 50 \leq 240$ \\
	C3: $4x \leq 140$ \\
	C4: $x  \leq 35$ \\ \\
	Der Gewinn wird maximal für x = 35 und y = 50.
	\subsection*{Aufgabe 2}
	x = Flugzeug vom Typ A; y Flugzeug vom Typ B \\
	\\
	K = 4000x + 1000y \\
	1. $x \leq 11; y \leq 8$ \\
	2. $200x + 100y \geq 1600$ \\
	3. $6x + 15y \geq 96$ \\ \\
	Als LGS: \\
	A1:$x \leq 11$ \\
	B1:$y  \leq 8$ \\
	C1: $2x + y \geq 16$ \\
	D1: $6x + 15y \geq 96$ \\ \\
	C2: $x\geq 8 - \frac{1}{2}y$ \\
	C2 in D1 \\
	D2: $6(8 - \frac{1}{2}y) + 15y \geq 96$ \\
	D3: $48 - 3y + 15y \geq 96$ \\
	D4: $12y \geq 48$ \\
	D5: $y \geq 4$ \\
	
	\subsection*{Aufgabe 3}
	\subsection*{a)}
	g1:
	$\left(
	\begin{array}{c}
	-4 \\ -8 \\ 0
	\end{array}
	\right)$
	+
	r
	$\cdot$
	$\left(
	\begin{array}{c}
	1 \\ 1 \\ 15
	\end{array}
	\right)$ \\
	g2:
	$\left(
	\begin{array}{c}
	24 \\ 32 \\ 0
	\end{array}
	\right)$
	+
	s
	$\cdot$
	$\left(
	\begin{array}{c}
	-1 \\ -2 \\ 20
	\end{array}
	\right)$ \\
	g1 = g2 \\
	Als LGS: \\ \\
	A1: -4 + r = 24 - s \\
	B1: -8 + r = 32 - 2s \\
	C1: 15r = 20s \\
	A2: r = 28 - s \\
	C2: 15(28 - s) = 20s \\
	C3: 420 - 15s = 20x \\
	C4: 420 = 35s \\
	C5: s = 12 \\
	s in g2: \\
	P = 
	$\left(
	\begin{array}{c}
	24 \\ 32 \\ 0
	\end{array}
	\right)$
	+
	12
	$\cdot$
	$\left(
	\begin{array}{c}
	-1 \\ -2 \\ 20
	\end{array}
	\right)$ \\
	$\left(
	\begin{array}{c}
	24 \\ 32 \\ 0
	\end{array}
	\right)$
	+
	$\left(
	\begin{array}{c}
	-12 \\ -24 \\ 240
	\end{array}
	\right)$ \\
	P = (12 $\|$ 8 $\|$ -240)
	\subsection*{b)}
	Geschwindigkeit(Weg/Zeit): \\ \\
	$\vec{PQ}$ =
	$\left(
	\begin{array}{c}
	10\\20\\456
	\end{array}
	\right)$ \\
	$\|\vec{PQ}\|$ = 
	$\sqrt{10^2+20^2+456^2}$ = 456,548 km \\
	V = $\frac{456,548}{60}$ \\
	V = 7,60913 $\frac{km}{h}$
	\subsection*{Aufgabe 4}
	\subsection*{a)}
	E: $\vec{x} = 
	\left(
	\begin{array}{c}
	6 \\ 0 \\ 3
	\end{array}
	\right)
	+
	s \cdot
	\left(
	\begin{array}{c}
	-3 \\ 3 \\ 4
	\end{array}
	\right)
	+
	r \cdot
	\left(
	\begin{array}{c}
	0 \\ 10 \\ 0
	\end{array}
	\right)
	$
	\subsection*{b)}
	$\vec{n} = 
	\left(
	\begin{array}{c}
	-4 \\ 0 \\ -3
	\end{array}
	\right)$  \\
	E: -4x -3z = d \\
	d = -4 * 6 - 3*3 \\
	d = -24 - 9 \\
	d = -33 \\
	E: -4x -3z = -33 \\
	E: 4x + 3z = 33
	\subsection*{c)}
	Gerade Durch den Stützpunkt der Fahnenstange in Richtung des Normalenvektors der Ebene und somit auch durch die Ebene. \\ \\
	q: 
	$\vec{x}$ =
	$\left(
	\begin{array}{c}
	 4\\4\\3
	\end{array}
	\right)
	+
	r
	\cdot	
	\left(
	\begin{array}{c}
	0 \\ 0 \\ 1
	\end{array}
	\right)
	$ \\ \\
	q in E \\ \\
	$4(4 + 0 \cdot r) + 3(3 + r) = 33$ \\
	$16 + 9 + 3r = 33$ \\
	$25 + 3r = 33$ \\
	$3r = 8$ \\
	$r = \frac{8}{3}$
	\\
	r in q \\ \\
	$\vec{x}$ =
	$\left(
	\begin{array}{c}
	4\\4\\3
	\end{array}
	\right)
	+
	\frac{8}{3}
	\cdot	
	\left(
	\begin{array}{c}
	0\\ 0 \\ 1
	\end{array}
	\right)
	$ \\ \\
	$\vec{x}$ =
	$\left(
	\begin{array}{c}
	4\\4\\3
	\end{array}
	\right)
	+
	\left(
	\begin{array}{c}
	0\\0\\\frac{8}{3}
	\end{array}
	\right)
	$ \\ \\
	$\vec{x}$ =
	$\left(
	\begin{array}{c}
	4\\4\\\frac{17}{3}
	\end{array}
	\right)$ \\
	\\
	
	\subsection*{Aufgabe 5}
	Höhe der Pyramide \\ \\
	E: $\vec{x}$ =
	$\left(
	\begin{array}{c}
	 0\\0\\0 
	\end{array}
	\right)$
	+
	r $\cdot$
	$\left(
	\begin{array}{c}
	8\\4\\2 
	\end{array}
	\right)$
	+ s $\cdot$
	$\left(
	\begin{array}{c}
	-2\\6\\-4 
	\end{array}
	\right)$
	\\
	$\vec{n} =$ 
	$\left(
	\begin{array}{c}
	-28\\28\\56
	\end{array}
	\right)$
	$\|\vec{n}\| = \sqrt{(-28^2) + 28^2 + 56^2} = 56$ \\ \\
	H Normalform: \\ \\
	$\frac{1}{56} \cdot
	\left(
	\begin{array}{c}
	-28 \\ 28 \\ 56
	\end{array}	
	\right)  \left[\vec{x} - 
	\left(
	\begin{array}{c}
	 8\\4\\2
	\end{array}
	\right) \right]$
	= 0 \\ \\
	P in H Form \\
	\\
	d = $\| \frac{1}{56} (-28 * (1 - 8) + 28*(7-4) + 56*(56-2))  \|$ \\
	d = $\| \frac{1}{56} (-28 * -7  + 28*3 + 56*54)\|$ \\
	d = 59
	\subsection*{Aufgabe 6}
	\subsection*{a)}
	Die Matrix A gibt an wie viel Material für ein Basisregal und eines Anbauregals benötigt werden. \\
	Die Matrix B gibt an wie viel Material für den Bau eines Regals Typ A und Typ B benötigt werden. \\
	\subsection*{b)}
	A $\cdot$ B =
	$\left(
	\begin{array}{cc}
	2&1\\5&4 \\20&0 \\0&16
	\end{array}
	\right)$
	$\cdot$
	$\left(
	\begin{array}{cc}
	1&1\\1&2
	\end{array}
	\right)$
	\\
	A $\cdot$ B =
	$\left(
	\begin{array}{cc}
	 2 * 1 + 1 * 1 & 2* 1 + 1 *2 \\ 5 * 1 + 4 * 1& 5*1 + 4 * 2\\ 20 * 1 + 0 * 1& 20 + 0\\ 0 + 16 & 2 * 16
	\end{array}
	\right)$
	\\
	A $\cdot$ B =
	$\left(
	\begin{array}{cc}
	3 & 4\\ 9&13 \\20 & 20\\ 16& 32
	\end{array}
	\right)$ \\ \\
	Das Produkt der Beiden Matrizen A und B beschreibt den gesamten Materialverbrauch für die Herstellung eines Regals beider Typen.	
	\subsection*{c)}
	$\left(
	\begin{array}{cc}
	3 & 4\\ 9&13 \\20 & 20\\ 16& 32
	\end{array}
	\right)$
	$\cdot$
	$\left(
	\begin{array}{c}
	10 \\ 5
	\end{array}
	\right)$
	=
	$\left(
	\begin{array}{c}
	 50 \\ 155 \\ 300 \\ 320 
	\end{array}
	\right)$
	
\end{document}