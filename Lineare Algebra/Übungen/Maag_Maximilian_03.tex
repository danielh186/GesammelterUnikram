\documentclass{article}
\usepackage{mathtools}
\usepackage{mathtools}
\begin{document}
	\section*{Lsg Vorschlag LA Ü03 Maximilian Maag}
	\section*{Aufgabe A}
	\begin{itemize}
		\item richtig
		\item richtig
		\item richtig
		\item falsch
		\item falsch
	\end{itemize}
	\section*{Aufgabe B}
	\begin{itemize}
		\item I $\to$ l
		\item II $\to$ k 
		\item III $\to$ n
		\item IV $\to$ g 
		\item V $\to$ p
		\item VI $\to$ h
		\item VII $\to$ f
		\item VIII $\to$ e
		\item IX $\to$ m
	\end{itemize}
	\section*{Aufgabe 1}
	\subsection*{a)}
	$g_1$ = $\left(\begin{array}{c}
	3 \\ 2 \\ 0
	\end{array}\right) $
	+ r
	$\left(\begin{array}{c}
	0 \\ 1 \\ 0
	\end{array}\right) $
	\subsection*{b)}
	$g_2$ = 
	$\left(\begin{array}{c}
	0 \\ 0 \\ 0
	\end{array}\right)$
	+ r 
	$\left(\begin{array}{c}
	2 \\ 4 \\ -2
	\end{array}\right)$
	\subsection*{c)}
	$g_3$ = 
	$\left(\begin{array}{c}
	0 \\ 0 \\ 0
	\end{array}\right)$
	+ r
	$\left(\begin{array}{c}
	1 \\ 1 \\ 0
	\end{array}\right)$
	\section*{Aufgabe 2}
	\subsection*{a)}
	Ansatz mit Vektoren als LGS in Abhängigkeiit von x, y, z und u. \\ \\
	A1: x + z + 4u = 0\\
	B1: 2x + y + z = 0\\
	C1: y + z -3u = 0 \\
	D1: -x + y + z = 0\\
	\\
	A2: A1 + C1 \\
	A2: x + z + 4u + y + z -3u = 0 \\
	A2: x + y + 2z + u = 0 \\
	B2: B1 + C1 \\
	B2: 2x + y + z + y + z - 3u  = 0\\
	B2: 2x + 2y + 2z - 3u = 0 \\
	B3: B2 + 2*D1 \\
	B3: 2x + 2y + 2z - 3u + 2*(-x + y + z) = 0 \\
	B3: 2x + 2y + 2z - 3u -2x + 2y + 2z = 0 \\
	B3: 4y + 4z - 3u = 0 \\
	C2: C1 - $\frac{1}{4}$B3 \\
	C2: y + z -3u - $\frac{1}{4}$(4y + 4z - 3u = 0) = 0 \\
	C2: y + z -3u -y - z + $\frac{3}{4}$u = 0 \\
	C2: -3u + $\frac{3}{4}$u = 0 \\
	C2: u = 0 \\
	A3: u in A1 \\
	A3: x + z + 0 = 0 \\
	A3: x  = -z \\
	D2: x in D1 \\
	D2: -z + y + z = 0\\
	D2: y = 0\\
	B4: y und u in B3 \\
	B4: 0 + 4z - 0 = 0 \\
	B4:  z = 0 \\
	A4: z in A3
	A4: z = 0 \\ 
	x = 0; y = 0; z = 0 \\ \\
	Das LGS hat nur eine triviale Lösung daher sind die Vektoren linear unabhängig.
	\subsection*{b)}
	Ansatz mit LGS in Abhängigkeit von x, y, z und u, sollte ich eigentlich machen aber es ist Dienstag Abend ich will fertig werden daher mit Kehrmatrix \\ \\
	A * X = B \\
	A = $
	\left(\begin{matrix}
		1 & -2 & 0 & 3 \\
		2 & 1 & -3 & 3 \\
		3 & 4 & -8 & 6 \\
		3 & -4 & 10 & 8
	\end{matrix}\right)
	$ \\
	B = 
	$
	\left(\begin{matrix}
		0 \\
		0 \\
		0 \\
		0
	\end{matrix}\right)	$ \\
	$A^{-1} = 
	\left(\begin{matrix}
	\frac{-2}{5} & \frac{8}{5} & \frac{-3}{5} & 0 \\
	\frac{-2}{5} & \frac{-19}{85} & \frac{33}{170} & \frac{3}{34} \\
	\frac{-1}{5} & \frac{-2}{85} & \frac{-1}{170} & \frac{3}{34} \\
	\frac{1}{5} & \frac{-58}{85} & \frac{28}{85} & \frac{1}{17}
	\end{matrix}\right)
	$ \\
	$X = A^{-1} * B =
	\left(\begin{matrix}
	\frac{-2}{5} & \frac{8}{5} & \frac{-3}{5} & 0 \\
	\frac{-2}{5} & \frac{-19}{85} & \frac{33}{170} & \frac{3}{34} \\
	\frac{-1}{5} & \frac{-2}{85} & \frac{-1}{170} & \frac{3}{34} \\
	\frac{1}{5} & \frac{-58}{85} & \frac{28}{85} & \frac{1}{17}
	\end{matrix}\right) * 
	\left(\begin{matrix}
	0 \\
	0 \\
	0 \\
	0
	\end{matrix}\right)
	=
	\left(\begin{matrix}
	0 \\
	0 \\
	0 \\
	0
	\end{matrix}\right)
	$ \\ \\
	Das LGS hat ausschließlich eine triviale Lösung daher sind die Vektoren linear unabhängig.
	\section*{Aufgabe 3}
	\subsection*{a)}
	:= $s_1$: Seitenhalbierende $\vec{AB}$ \\
	:= $s_2$: Seitenhalbierende $\vec{BC}$ \\
	:= $s_3$: Seitenhalbierende $\vec{CA}$ \\ \\
	$s_1$ = $|$$\vec{OC}$-$\vec{AB}$*$\frac{1}{2}$ $|$ \\
	$\vec{OP}$ = ($\vec{AB}$)*$\frac{1}{2}$ \\
	$\vec{OP}$ =
	$
	\left(\begin{array}{c}
	6 \\ -10 \\ 10
	\end{array}\right)
	$ * $\frac{1}{2}$ \\
	$\vec{OP}$ =
	$
	\left(\begin{array}{c}
	3 \\ -5 \\ 5
	\end{array}\right)
	$  \\
	$s_1$ = 
	$
	\left|\left(\begin{array}{c}
		-1 \\ -5 \\ 4
	\end{array}\right)\right|
	$ \\
	$s_1$ = $\sqrt{1^2 + 5^2 + 4^2}$ \\
	$s_1$ = $\sqrt{1 + 25 + 16}$ \\
	$s_1$ = $\sqrt{26 + 16}$ \\
	$s_1$ = $\sqrt{42}$ \\ \\
	$s_2 = \vec{OA} - \frac{\vec{BC}}{2}$  \\
	$\frac{\vec{BC}}{2}$ = 
	$\left(\begin{array}{c}
	-3 \\ 4 \\ -4
	\end{array}\right)$ \\
	$s_2$ = 
	$\left|\left( \begin{array}{c}
	7 \\ -2 \\ 3
	\end{array} \right)\right|$ \\
	$s_2 = \sqrt{7^2 +  (-2)^2 + 3^2}$ \\
	$s_2 = \sqrt{49 +  4 + 9}$ \\
	$s_2 = \sqrt{62}$ \\ \\
	$s_3 = \vec{OB} - \frac{\vec{CA}}{2}$ \\
	$\frac{\vec{CA}}{2} = 
	\left(\begin{array}{c}
	0 \\ -1 \\ 1
	\end{array}\right)$ \\
	$s_3 = \left|\left(\begin{array}{c}
	10 \\ -7 \\ 8
	\end{array}\right)\right|$ \\
	$s_3 = \sqrt{100 + 49 + 64}$ \\
	$s_3 = \sqrt{213}$
	\subsection*{b)}
	$\vec{OS} = $$\frac{1}{3}$*($\vec{OA}$+$\vec{OB}$+$\vec{OC}$) \\
	$\vec{OS} = \vec{OC} + \frac{2}{3}\overrightarrow{CM_{AB}}$ \\
	$\vec{OS} = \vec{OC} + \frac{2}{3}\overrightarrow{CM_{AB}}$ \\
	$\vec{OS}$ = $\vec{OC}$ + $\frac{2}{3}$*($\frac{\vec{OA}+\vec{OB}}{2}$-$\vec{OC}$) \\
	$\vec{OS}$ = $\vec{OC} + \frac{\vec{OA}+\vec{OB}-2\vec{OC}}{3}$ \\
	$\vec{OS}$ =$\frac{3\vec{OC}+\vec{OA}+\vec{OB}-2\vec{OC}}{3}$ \\
	$\vec{OS}$ =$\frac{\vec{OC}+\vec{OA}+\vec{OB}}{3}$ \\
	$\vec{OS}$ =$\frac{1}{3}(\vec{OC}+\vec{OA}+\vec{OB})$ \\
	
	\subsection*{c)}
	Ansatz: Der Vektor $\vec{OC}$ beschreibt die Verschiebung vom Ursprung zu Punkt C, daher ist dieser Vektor in Zeilenschreibweise der Punkt C. \\
	\\
	$\vec{OS} = \frac{1}{3}(\vec{OA}+\vec{OB}+\vec{OC})$ \\
	$3*\vec{OS} = \vec{OA}+\vec{OB}+\vec{OC}$ \\
	$3*\vec{OS} = \vec{OA}+\vec{OB}+\vec{OC}$ \\
	$3*\vec{OS} - \vec{OA} - \vec{OB} = \vec{OC}$ \\
	\\
	$
	\vec{OC}= 3
	\left(\begin{array}{c} 
	5 \\ 6 \\ 3 \end{array}\right)
	$ - 
	$\left(\begin{array}{c} 
		4 \\ 9 \\ 2 
	\end{array}\right)
	$
	$ - 
	\left(\begin{array}{c} 
	15 \\ 18 \\ 9 \end{array}\right)
	$ \\
	$
	\vec{OC}=
	\left(\begin{array}{c} 
	11 \\ 9 \\ 7 \end{array}\right)
	$
	$ - 
	\left(\begin{array}{c} 
	15 \\ 18 \\ 9 \end{array}\right)
	$ \\
	$
	\vec{OC}=
	\left(\begin{array}{c} 
	-4 \\ -9 \\ -2 \end{array}\right)
	$ \\ \\
	C = (-4$|$-9$|$-2)
\end{document}