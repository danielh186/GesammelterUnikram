\documentclass{article}
\begin{document}
	\section*{Lsg Vorschlag Ü09 Maximilian Maag}
	\subsection*{Aufgabe A}
	2 1 9 5 7 8 4 6 3
	\subsection*{Aufgabe B}
	 homogenes LGS lösen
	 \subsection*{Aufgabe 1}
	 Die Bilder Der Einheitsvektoren sind die Spalten der Abbildungsmatrix!!!
	 \subsection*{a)}
	 P = (1 $\|$ 0) P' = (1 $\|$ -2) \\
	 B = (0 $\|$ 1) B' = (0 $\|$ -1) \\
	 A = 
	 $\left(
	 \begin{array}{cc}
	 1 & 0\\ -2 &-1 
	 \end{array}
	 \right)$
	 \subsection*{b)}
	 A = 
	 $\left(
	 \begin{array}{cc}
		\cos(135°)& -\sin(135°) \\ \sin(135°) & \cos(135°) 
	 \end{array}
	 \right)$
	 \subsection*{c)}
	 P = (1 $\|$ 0) P' = (1 $\|$ -1) \\
	 B = (0 $\|$ 1) B' = (0  $\|$ 0) \\
	 A =
	 $\left(
	 \begin{array}{cc}
	 1 & 0\\ -1&0
	 \end{array}
	 \right)$
	 \subsection*{d)}
	 P = (1 $\|$ 0) P' = (0 $\|$ 0) \\
	 B = (0 $\|$ 1) B' = (0  $\|$ 1) \\
	 A =
	 $\left(
	 \begin{array}{cc}
	 0 & 0\\ 1&0
	 \end{array}
	 \right)$
	 \subsection*{Aufgabe 2}
	 Leider zu visuell, aufgrund meiner Sehschädigung ist die Lösung der Aufgabe für mich nicht bestimmbar.
	 \subsection*{Aufgabe 3}
	 \subsection*{a)}
	 Kern: homogenes LGS \\ \\
	 A $\cdot$ $\vec{x}$ =
	 $\left(
	 \begin{array}{c}
	 0 \\ 0 \\ 0
	 \end{array}
	 \right)$ \\
	 Als LGS: \\ \\
	 A1: $\frac{1}{5}$ x + 0y + $\frac{2}{5}$ z = 0 \\
	 B1: 0x + y + 0z = 0 \\
	 C1: $\frac{2}{5}$ x + 0 y + $\frac{4}{5}$ z = 0 \\ \\
	 A2: $\frac{1}{5}$ x = -$\frac{2}{5}$ z  \\
	 A3: x = -2z \\
	 C2: x in C1 \\
	 C2: $\frac{2}{5}$ -2z + $\frac{4}{5}$ z = 0 \\
	 C3: -4z + 4z = 0 \\
	 C4: 0 = 0 \\
	 x = -2z \\ \\
	 Unendlich viele Lösungen für die gelten mus: x = -2z.0
	 \subsection*{b)}
	 A $\cdot$ $\vec{x}$ = $\vec{x}$ \\ \\
	 Als LGS: \\ \\
	 A1: $\frac{1}{5}$ x + 0y + $\frac{2}{5}$ z = x \\
	 B1: 0x + y + 0z = y \\
	 C1: $\frac{2}{5}$ x + 0 y + $\frac{4}{5}$ z = z \\ \\
	 A2: -$\frac{4}{5}$ x + $\frac{2}{5}$ z = 0\\
	 B2: 0 = 0 \\
	 C2: $\frac{2}{5}$ x - $\frac{1}{5}$ z = 0\\
	 C3: $\frac{2}{5}$ x = $\frac{1}{5}$ z\\
	 C4: 2 x = z\\
	 C5: x = $\frac{1}{2}$ z \\
	 A3: -$\frac{4}{5}$ $\frac{1}{2}$ z + $\frac{2}{5}$ z = 0\\
	 A4: -4 $\frac{1}{2}$ z + 2 z = 0\\
	 A5: -2 z + 2 z = 0\\
	 A5: 0 = 0\\ \\
	 Die Fixpunkte sind unendlich viele Punkte für die gelten muss: x = $\frac{1}{2}$ z;
	 
	 \subsection*{c)}
	 
	   
\end{document}