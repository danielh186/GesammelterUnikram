\documentclass{article}
\begin{document}
	\section*{Lsg Vorschlag LAÜ05 Maximilian Maag}
	\subsection*{Aufgabe A}
	\begin{itemize}
		\item falsch
		\item richtig
		\item richtig
		\item falsch
		\item richtig
	\end{itemize}
	\subsection*{Aufgabe B}
	\subsection*{a)}
	Annahme: Das Dreieck ABC umfasst die Winkel $\alpha$, $\beta$ und $\gamma$. \\
	\\
	$\cos$($\alpha$) = $\frac{\vec{AB} \cdot \vec{AC}}{\|\vec{AB}\| \cdot \|\vec{AC}\|}$ \\
	$\vec{AB}$ = 
	$
	\left(\begin{array}{c}
	-1 \\ 4  \\ -1
	\end{array}\right) \\
	$
	$\vec{AC}$ = 
	$
	\left(\begin{array}{c}
	-5 \\ -1 \\ 2
	\end{array}\right)
	$ \\
	$\cos$($\alpha$) =
	$\frac{  \left(\begin{array}{c}
		-1 \\ 4  \\ -1
		\end{array}\right) 
	\cdot
	 \left(\begin{array}{c}
	 -5 \\ -1 \\ 2
	 \end{array}\right)}{\left\|\left(\begin{array}{c}
	 -1 \\ 4  \\ -1
	 \end{array}\right)\right\| \cdot \left\|\left(\begin{array}{c}
	 -5 \\ -1 \\ 2
	 \end{array}\right)\right\| }$ \\
	$\cos$($\alpha$) = $\frac{-1 \cdot 5 + 4 \cdot -1 + -1 \cdot 2}{\sqrt{1^2 + 4^2 +1^2} \cdot \sqrt{5^2 + 1 + 2^2}}$ \\
	$\cos$($\alpha$) = $\frac{-1 \cdot -5 + 4 \cdot -1 + -1 \cdot 2}{\sqrt{18} \cdot \sqrt{30}}$ \\
	$\cos$($\alpha$) = $\frac{5  - 4 - 2}{\sqrt{18} \cdot \sqrt{30}}$ \\
	$\cos$($\alpha$) = $\frac{-1}{\sqrt{18} \cdot \sqrt{30}}$ \\
	$\cos$($\alpha$) $\approx$ 0,0430331 \\
	$\alpha$ = $\cos^{-1}$ \\
	$\alpha$ $\approx$ 61,7471° \\ \\
	$\vec{BC}$ = 
	$\left(\begin{array}{c}
	-4 \\ -5 \\ 3
	\end{array}\right)$\\
	$\vec{BA}$ = $\left(\begin{array}{c}
	-1 \\ 4 \\ -1
	\end{array}\right)$ \\
	
	
	\subsection*{b)}
	A = $\frac{1}{2}$ $\cdot$ $|$ $\vec{AB}$ x $\vec{AC}$ $|$ \\
	$\vec{AB}$ = 
	$
	\left(\begin{array}{c}
	-1 \\ 4  \\ -1
	\end{array}\right) \\
	$
	$\vec{AC}$ = 
	$
	\left(\begin{array}{c}
	-5 \\ -1 \\ 2
	\end{array}\right)
	$ \\
	A = $\frac{1}{2}$ $\cdot$
	$
	\left\|\left(\begin{array}{c}
	7 \\ 7 \\ 21
	\end{array}\right)\right\|
	$ \\
	A = $\frac{1}{2}$ $\cdot$ $\sqrt{7^2 + 7^2 + 21^2}$ \\
	A = $\frac{1}{2}$ $\cdot$ $\sqrt{49 + 49 + 441}$ \\
	A = $\frac{1}{2}$ $\cdot$ $\sqrt{539}$ \\
	A = 11,6082
	\section*{Aufgabe 1}
	Bilde Ebene in Normalform. \\
	\\
	Normalenvektor $\vec{n}$ = 
	$\left(\begin{array}{c}
	-6 \\ 2 \\ 1
	\end{array}\right)$
	X
	$\left(\begin{array}{c}
	4 \\1 \\ -3
	\end{array}\right)$ = 
	$\left(\begin{array}{c}
	-7 \\ -14 \\ -14
	\end{array}\right)$
	\\  \\
	Hilfsebene E mit $\vec{n}$ \\ \\
	E: 
	$\left(\begin{array}{c}
	-7 \\ -14 \\ -14
	\end{array}\right)$
	 $\cdot$ $\left[\vec{x} - 
	\left(\begin{array}{c}
	9 \\ 3 \\ 8
	\end{array}\right)
	\right]$
	= 0 \\
	\\
	Überführung in Koordinatenform: \\
	\\
	E: -7x - 14y -14z = 63 + 42 112 \\
	E: -7x - 14y -14z = 105 + 112 \\
	E: -7x - 14y -14z = 217 \\ \\
	Überführung in hesse'sche Normalform: \\ \\
	$\left\|\vec{n}\right\|$ = $\sqrt{7^2 +14^2 +14^2}$ \\
	$\left\|\vec{n}\right\|$ = $\sqrt{49 + 196 + 196}$ \\
	$\left\|\vec{n}\right\|$ = $\sqrt{49 + 196 + 196}$ \\
	$\left\|\vec{n}\right\|$ = $\sqrt{441}$ \\
	$\left\|\vec{n}\right\|$ = 21 \\ \\
	Die H-Form liegt vor, wenn die Normalform um den Betrag des Normalvektors dividiert wird. \\ \\
	E: $\frac{1}{21}$ $\|-7x - 14y -14z -217 \| = 0 \\ \\$
	Stützpunkt von h in E liefert die Distanz d: \\ \\
	d = $\frac{1}{21}$ $\cdot$ $\|-7 \cdot 4 - 14 \cdot 2 -14 -217 \| \\$
	d =  $\frac{1}{21}$ $\cdot$ $\|-28 - 28 -14 -217 \| \\$
	d =  $\frac{1}{21}$ $\cdot$ $\|-70 -217 \| \\$
	d =  $\frac{1}{21}$ $\cdot$ $\|-287 \| = 0 \\$
	d = $\frac{1}{21}$ $\cdot$ 287 \\$ \\
	d = 13,6667 RE$ (Raumeinheiten)
	\section*{Aufgabe 2}
	\subsection*{a)}
	E: 
	$\left(\begin{array}{c}
	2 \\ 1 \\ -3
	\end{array}\right)$
	$\cdot$ $\left[\vec{x} - 
	\left(\begin{array}{c}
	1 \\ 1 \\ 1
	\end{array}\right)
	\right]$
	= 0 \\
	\subsection*{b)}
	E: 
	$\left(\begin{array}{c}
	0 \\ 0 \\ 1
	\end{array}\right)$
	$\cdot$ $\left[\vec{x} - 
	\left(\begin{array}{c}
	1111111111 \\ 2222222222 \\ 0
	\end{array}\right)
	\right]$
	= 0 \\
	\subsection*{c)}
	E: 
	$\left(\begin{array}{c}
	1 \\ 0 \\ 0
	\end{array}\right)$
	$\cdot$ $\left[\vec{x} - 
	\left(\begin{array}{c}
	1 \\ 1 \\ 0
	\end{array}\right)
	\right]$
	= 0 \\
	\subsection*{d)}
	Ebene E in Parameterform \\
	\\
	E: $\vec{x} =$
	$\left(\begin{array}{c}
		0 \\ 2 \\ 0
	\end{array}\right)$ + 
	r $\cdot$ 
	$\left(\begin{array}{c}
	2 \\ -1 \\ 2
	\end{array}\right)$ +
	s $\cdot$ 
	$\left(\begin{array}{c}
	1 \\ -2 \\ 2
	\end{array}\right)$ \\ \\
	Überführung in Normalform \\ \\
	E:
	$\left(\begin{array}{c}
	2 \\ -2 \\ -3
	\end{array}\right)$
	$\cdot$
	$\left[\vec{x} - \left(\begin{array}{c}
		0 \\ 2 \\ 0
	\end{array}\right) \right]$ = 0
	\section*{Aufgabe 3}
	\subsection*{a)}
	Berechnung der Höhe der Pyramide als Abstand des Punktes S und dem Durchstoßpunkt einer Geraden h, die senkrecht auf der Ebene E stehe. \\ \\
	E: $\vec{x} =$
	$\left(\begin{array}{c}
	7 \\ 1 \\ 0
	\end{array}\right)$ + 
	r $\cdot$ 
	$\left(\begin{array}{c}
	0 \\ 6 \\ 2
	\end{array}\right)$ +
	s $\cdot$ 
	$\left(\begin{array}{c}
	-6 \\ 6 \\ 4
	\end{array}\right)$ \\
	Normalenvektor $\vec{n}$ =
	$\left(\begin{array}{c}
	 12 \\ -12 \\ 36
	\end{array}\right) $ \\ \\
	E in Normalform: \\ \\
	E:
	$\left(\begin{array}{c}
	12 \\ -12 \\ 36
	\end{array}\right)$
	$\cdot$
	$\left[\vec{x} - \left(\begin{array}{c}
	7 \\ 1 \\ 0
	\end{array}\right) \right]$ = 0 \\
	\\
	E in Koordinatenform: \\ \\
	E: 12x -12y + 36z - 72 = 0 \\
	E: 12x -12y + 36z =  72 \\
	E: 2x -2y + 6z = 12 \\ \\
	Hilfsgerade h: \\
	h: 
	$\vec{x} =$
	$\left(\begin{array}{c}
	7 \\ 2 \\ 4
	\end{array}\right)$ + 
	r $\cdot$ 
	$\left(\begin{array}{c}
	2 \\ -2 \\ 6
	\end{array}\right)$ \\
	h in E: \\
	2(7 + 2r) -2(2 -2r) + 6(4 + 6r) = 12 \\
	14 + 4r -4 + 4r + 24 + 36r = 12 \\
	34 + 44r = 12 \\
	44r = -22 \\
	r = -$\frac{1}{2}$ \\
	r in h: \\
	$\vec{OP}$ = 
	$\left(\begin{array}{c}
	7 \\ 2 \\ 4
	\end{array}\right)$ - 
	$\frac{1}{2}$ $\cdot$ 
	$\left(\begin{array}{c}
	2 \\ -2 \\ 6
	\end{array}\right)$ \\
	$\vec{OP}$ = 
	$\left(\begin{array}{c}
	7 \\ 2 \\ 4
	\end{array}\right)$ - 
	$\left(\begin{array}{c}
	-1 \\ 1 \\ -6
	\end{array}\right)$ \\
	$\vec{OP}$ = 
	$\left(\begin{array}{c}
	8 \\ 1 \\ 10
	\end{array}\right)$ \\ \\
	Durchstoßpunkt P = (8 $|$ 1 $|$ 10)
	Höhe der Pyramide ergibt sich aus dem Abstand zwischen S und dem Durchstoßpunkt der Hilfsgeraden h. \\ \\
	$\vec{PS}$ = 
	$\left(\begin{array}{c}
	7 \\ 2 \\ 4
	\end{array}\right)$ - 
	$\left(\begin{array}{c}
	8 \\ 1 \\ 10
	\end{array}\right)$ \\
	$\vec{PS}$ = 
	$\left(\begin{array}{c}
	-1 \\ 1 \\ -6
	\end{array}\right)$ \\ \\
	h = $\|\vec{PS}\|$ \\
	h = $\sqrt{1 + 1 + 6^2}$ \\
	h = $\sqrt{1 + 1 + 36}$ \\
	h = $\sqrt{38}$ \\
	h = 6,16441 \\
	h $\approx$ 6,16 RE
	\subsection*{b)}
	Für die Grundfläche der Pyramide gilt: $\|\vec{AB} x \vec{AC}\|$  \\
	$\to$ $V_{Pyramide}$ = $\frac{1}{3} \cdot \|\vec{AB}$ x $\vec{AC}\| \cdot h$ \\
	G = 
	$\left(\begin{array}{c}
	0 \\ 6 \\ 2
	\end{array}\right)$
	x
	$\left(\begin{array}{c}
	-6 \\ 6 \\ 4
	\end{array}\right)$ \\
	G = 
	$\left(\begin{array}{c}
	12 \\ -12 \\ 36
	\end{array}\right)$ \\
	V = $\frac{1}{3} * G * h$ \\
	V = 12 * 6,16441 \\
	V = 73,9729 \\
	V $\approx$ 73,973 VE (Volumeneinheiten)
\end{document}