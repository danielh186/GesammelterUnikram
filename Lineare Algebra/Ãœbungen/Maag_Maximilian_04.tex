\documentclass{article}
\usepackage{mathtools}
\usepackage{graphicx}
\begin{document}
	\section*{Lsg Vorschlag LA Ü04 Maximilian Maag}
	\subsection*{Aufgabe A}
	\subsection*{a)}
	g =
	$
	\left(\begin{array}{c}
	3 \\ 0 \\ 1
	\end{array}\right)
	$
	+
	r *
	$
	\left(\begin{array}{c}
	-3 \\ 6 \\ 3
	\end{array}\right)
	$
	
	\subsection*{b)}
	Prüfe ob P auf AB liegt: \\	\\
	$
	\left(\begin{array}{c}
	1 \\ 4 \\ 3
	\end{array}\right)
	$	=
	$
	\left(\begin{array}{c}
	3 \\ 0 \\ 1
	\end{array}\right)
	$
	+
	r *
	$
	\left(\begin{array}{c}
	-3 \\ 6 \\ 3
	\end{array}\right)
	$
	\\ \\
	Daraus folgt folgendes LGS: \\
	A1: 1 = 3 - 3r \\
	B1: 4 = 0 + 6r \\
	C1: 3 = 1 + 3r \\
	r = $\frac{2}{3}$ \\ \\
	r ist zwischen 0 und 1 daher liegt P zwischen A un B.
	\subsection*{Aufgabe B}
	\subsection*{a)}
	Schnittpunkte Achsen für Ebene E. \\ \\
	E: 2x + 4y + 5z = 20 \\
	Z:  5z = 20 \\
	Z = (0$|$0$|$4) \\ \\
	X: 2x = 20 \\
	X: x = 10 \\
	X = (10$|$0$|$0) \\ \\
	Y: 4y = 20 \\
	Y: y = 5 \\
	Y = (0$|$5$|$0) \\ \\
	\subsection*{b)}
	Muss aus bekannten Gründen leider entfallen.
	\subsection*{c)}
	Aus der Gleichung können drei Punkte leicht extrahiert werden: \\
	A = (0$|$0$|$4)  \\
	B = (10$|$0$|$0)  \\
	C = (0$|$5$|$0) \\
	$\vec{OA}$ =
	$\left(\begin{array}{c}
	0 \\ 0 \\ 4
	\end{array}\right)$ \\
	$\vec{AB}$ =
	$\left(\begin{array}{c}
	10 \\ 0 \\ -4
	\end{array}\right)$ \\
	$\vec{AC}$ =
	$\left(\begin{array}{c}
	0 \\ 5 \\ -4
	\end{array}\right)$ \\
	\\
	E: $\vec{z}$ = 
	$\left(\begin{array}{c}
	0 \\ 0 \\ 4
	\end{array}\right)$ + r *
	$\left(\begin{array}{c}
	10 \\ 0 \\ -4
	\end{array}\right)$ + s *
		$\left(\begin{array}{c}
	0 \\ 5 \\ -4
	\end{array}\right)$
	\subsection*{Aufgabe 1}
	Eckpunkte abgelesen aus Abbildung: \\
	\\
	A = (4$|$0$|$0)\\
	B = (4$|$6$|$0)\\
	C = (0$|$6$|$0)\\
	D = (0$|$0$|$0)  \\
	F = (4$|$6$|$3)\\
	G = (0$|$6$|$3)\\
	H = (0$|$0$|$3)  \\
	I = (4$|$3$|$0)\\
	J = (2$|$6$|$0)\\
	\\
	$g_{HI}:$ $\vec{x} = 
	\left(\begin{array}{c}
	 0 \\ 0 \\ 3
	\end{array}\right)$
	+ r * 
	$
	\left(\begin{array}{c}
	4 \\ 3 \\ -3
	\end{array}\right)
	$
	\\
	$g_{HB}:$ $\vec{x} = 
	\left(\begin{array}{c}
	0 \\ 0 \\ 3
	\end{array}\right)$
	+ r * 
	$
	\left(\begin{array}{c}
	-4 \\ -6 \\ 3 
	\end{array}\right)
	$
	\\
	$g_{HF}:$ $\vec{x} = 
	\left(\begin{array}{c}
	0 \\ 0 \\ 3
	\end{array}\right)$
	+ r * 
	$
	\left(\begin{array}{c}
	4 \\ 6 \\ 0
	\end{array}\right)
	$
	\\
	$g_{GJ}:$ $\vec{x} = 
	\left(\begin{array}{c}
	0 \\ 6 \\ 3
	\end{array}\right)$
	+ r * 
	$
	\left(\begin{array}{c}
	2 \\ 0 \\ -3
	\end{array}\right)
	$
	\subsection*{Aufgabe 2}
	\subsection*{a)}
	E: 2x - 2y + z = 8 \\
	g in E \\
	2*(4 + 2r) - 2(1 + r) + 1 - 2r = 8 \\
	8 + 4r - 2 - 2r + 1 - 2r = 8 \\
	8 - 2 + 1 = 8 \\
	7 = 8 \\
	Widerspruch, daher ist g windschief zu E.
	\subsection*{b)}
	E: 3x + 2z = 12 \\
	3r + 2*(8 - 2r) = 12 \\
	3r + 16 - 4r = 12 \\
	16 -r = 12 \\
	-r = -4 \\
	r = 4 \\
	Durchstoßpunkt mit r = 4: \\ \\
	$
	\vec{x} = 
	\left(\begin{array}{c}
	0 \\ -1 \\ 8
	\end{array}\right)
	$
	+
	$
	4 * 
	\left(\begin{array}{c}
	1 \\ 2 \\ -2
	\end{array}\right)
	$
	\\
	$
	\vec{x} = 
	\left(\begin{array}{c}
	4 \\ 7 \\ 0
	\end{array}\right)
	$ \\ \\
	x = (4$|$7$|$0)
	\subsection*{c)}
	E: 3x - 3y + 2z = 6 \\
	3*(-2 - r) - 3r + 2*(6 + 3r) = 6 \\
	-6 - 3r - 3r + 12 + 6r = 6 \\
	6 = 6 \\
	Allgemeingültige Aussage, daher liegt g in E.
	\subsection*{Aufgabe 3}
	\subsection*{a)}
	Der Lichtstrahl schneidet die yz-Ebene im Punkt S, dieser kann aus der Abbildung abgelesen werden. \\
	S = (0$|$4$|$4)
	\subsection*{b)}
	Stützpunkt ist der Punkt S. \\
	Richtungsvektor zeigt von S nach T. \\
	S = (0$|$4$|$4) \\
	T = (1$|$2$|$0) \\
	$\vec{ST} $ = $
	\left(\begin{array}{c}
	1 \\ -2 \\ -4 
	\end{array}\right)
	$ \\ \\
	Daraus folgt folgende Gerade: \\ \\
	$g:$ $\vec{x} = 
	\left(\begin{array}{c}
	0 \\ 4 \\ 4
	\end{array}\right)$
	+ r * 
	$
	\left(\begin{array}{c}
	1 \\ -2 \\ -4
	\end{array}\right)
	$
	
	\subsection*{c)}
	Durch die Spiegelung der Geraden g in Punkt S, T und U wechselt der Richtungsvektor der Geraden g je einmal pro Koordinate das Vorzeichen. Daraus folgt, dass der Richtungsvektor der Geraden k entgegensetzt zum Richtungsvektor der Geraden g verläuft. Daraus folgt, dass die Richtungsvektoren von g und k parallel verlaufen müssen.
\end{document}