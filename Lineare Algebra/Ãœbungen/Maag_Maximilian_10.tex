\documentclass{article}
\begin{document}
	\section*{Lsg Vorschlag LAÜ10 Maximilian Maag}
	\subsection*{Aufgabe 1}
	A = 
	$\left(
	\begin{array}{ccc}
	 1&-1&1 \\ 0&-6&12 \\ -2&2&-2
	\end{array}
	\right)$ \\
	T = 
	$\left(
	\begin{array}{ccc}
	 -1&-1&-2 \\ 0&2&0 \\ 1&1&4
	\end{array}
	\right)$ \\
	$T^{-1}$ =
	$
	\left(\begin{array}{ccc}
	-2 & \frac{-1}{2} & -1 \\
	0 & \frac{1}{2} & 0 \\
	\frac{1}{2} & 0 & \frac{1}{2}
	\end{array}\right)
	$ \\
	B = $T^{-1}$ $\cdot$ A $\cdot$ T \\
	B = 
	$
	\left(\begin{array}{ccc}
	-2 & \frac{-1}{2} & -1 \\
	0 & \frac{1}{2} & 0 \\
	\frac{1}{2} & 0 & \frac{1}{2}
	\end{array}\right)
	$ $\cdot$ 
	$\left(
	\begin{array}{ccc}
	1&-1&1 \\ 0&-6&12 \\ -2&2&-2
	\end{array}
	\right)$
	$\cdot$
	$\left(
	\begin{array}{ccc}
	-1&-1&-2 \\ 0&2&0 \\ 1&1&4
	\end{array}
	\right)$ \\
	B = 
	$
	\left(\begin{array}{ccc}
	-2 & \frac{-1}{2} & -1 \\
	0 & \frac{1}{2} & 0 \\
	\frac{1}{2} & 0 & \frac{1}{2}
	\end{array}\right)
	$
	$\cdot$
	$\left(
	\begin{array}{ccc}
	 0&-2&2 \\ 12&0&48 \\ 0&4&-4
	\end{array}
	\right)$ \\
	B = 
	$
	\left(\begin{array}{ccc}
	-6 & 0 & -24 \\
	6 & 0 & 24 \\
	0 & 1 & -1
	\end{array}\right)
	$
	\subsection*{Aufgabe 2}
	\subsection*{a)}
	Berechnung der Eigenwerte: \\ \\
	$\xi_{A}(\lambda)$ = det(A - $\lambda$ $\cdot$ E) \\ \\
	= det
	$\left(
	\left(
	\begin{array}{ccc}
	2&1&0 \\ 1&1&1 \\ 0&1&2 
	\end{array}
	\right)
	-
	\lambda \cdot 
	\left(\begin{array}{ccc}
	1&0&0 \\ 0&1&0 \\ 0&0&1
	\end{array}\right)
	\right)$ \\
	= det
	$\left(
	\left(
	\begin{array}{ccc}
	2&1&0 \\ 1&1&1 \\ 0&1&2 
	\end{array}
	\right)
	-
	\left(\begin{array}{ccc}
	\lambda&0&0 \\ 0&\lambda&0 \\ 0&0&\lambda
	\end{array}\right)
	\right)$ \\
	= det
	$\left(
	\begin{array}{ccc}
	2 - \lambda&1&0 \\ 1&1- \lambda&1 \\ 0&1&2- \lambda 
	\end{array}
	\right)$ \\
	= $((2-\lambda) \cdot (1 - \lambda) \cdot (2-\lambda)) + (1 \cdot 1 \cdot 0) + (0 \cdot 1 \cdot (2 - \lambda)) - (0 \cdot (1-\lambda) \cdot 0) - (1 \cdot 1 \cdot (2 - \lambda)) - ((2 - \lambda) \cdot 1 \cdot 1)$ \\
	=
	$
	(4 - 2\lambda - 4\lambda + 2\lambda^{2} -2\lambda + \lambda^{2} + 2\lambda^{2} - \lambda^{3}) -(1 \cdot 1 \cdot (2 - \lambda)) - ((2 - \lambda) \cdot 1 \cdot 1)$ \\
	=
	$
	(-\lambda^{3}+ 5\lambda^{2}- 8\lambda + 4) -(1 \cdot 1 \cdot (2 - \lambda)) - ((2 - \lambda) \cdot 1 \cdot 1)$ \\
	$
	=
	(-\lambda^{3}+ 5\lambda^{2}- 8\lambda + 4) -2 + \lambda - 2 + \lambda$ \\	
	$
	-\lambda^{3}+ 5\lambda^{2} - 6\lambda + 4 - 4$ \\
	$\xi_{A}(\lambda)$ =$ -\lambda^{3}+ 5\lambda^{2}- 6\lambda$ \\ \\
	Bestimmung Nullstellen des Polynoms: \\ \\
	$\xi_{A}(\lambda)$ =$ -\lambda^{3}+ 5\lambda^{2}- 6\lambda$ \\
	$-\lambda^{3}+ 5\lambda^{2}- 8\lambda$ = 0 \\
	$-\lambda \cdot (\lambda^{2} - 5\lambda + 6) = 0$ \\
	$x_1$ = 0 \\
	$\lambda^{2} - 5\lambda + 6 = 0$  \\
	$x_{2, 3} = \frac{5}{2} +- \sqrt{\frac{5}{2}^2 - 6}$ \\
	$x_2$ = 2 \\
	$x_3$ = 3 \\
	\\
	\\
	Eigenvektoren \\
	\\
	$
	\left(
	\begin{array}{ccc}
	2&1&0 \\ 1&1&1 \\ 0&1&2 
	\end{array}
	\right)
	\cdot
	\left(
	\begin{array}{c}
	x \\ y \\ z
	\end{array}
	\right)
	=
	0 \cdot
	\left(
	\begin{array}{c}
	x \\ y \\ z
	\end{array}
	\right)
	$ \\ \\
	Als homogenes LGS: \\ \\
	A1: 2x + y  = 0\\
	B1: x + y + z = 0\\
	C1: y + 2z = 0\\
	B2: $\frac{1}{2}$y + z = 0 \\
	C2: y - y +2z -2z = 0 \\
	C2: 0 = 0 \\ 
	A2: 3x + 2y + z = 0 \\ \\
	In Stufenform: \\
	A2: 3x + 2y + z = 0 \\
	B2: $\frac{1}{2}$y + z = 0 \\
	C2: z = z \\
	z = z; y = -2z; x = z \\
	Lösung in Abhängigkeit von z: \\
	$\vec{x}$ = 
	$\left(
	\begin{array}{c}
	z\\-2z\\z
	\end{array}
	\right)$
	\\
	Daraus Folgt eine Lösung für den Eigenwert 0: \\
	$\vec{\lambda_{0}}$ =
	$\left(
	\begin{array}{c}
	1 \\ -2 \\ 1
	\end{array}
	\right)$
	$ \\ \\
	\left(
	\begin{array}{ccc}
	2&1&0 \\ 1&1&1 \\ 0&1&2 
	\end{array}
	\right)
	\cdot
	\left(
	\begin{array}{c}
	x \\ y \\ z
	\end{array}
	\right)
	=
	2 \cdot
	\left(
	\begin{array}{c}
	x \\ y \\ z
	\end{array}
	\right)
	$ \\
	Als LGS: \\
	A1: 2x + y = 2x \\
	B1: x + y + z = 2y \\
	
	C1: y + 2z = 2z \\
	Als homogenes LGS: \\
	A1: y = 0 \\
	B1: x - y + z = 0 \\
	C1: y = 0 \\
	x = -z; y = 0; z = z  \\
	$\vec{\lambda_{2}}$ =
	$\left(
	\begin{array}{c}
	-1 \\ 0 \\ 1 
	\end{array}
	\right)$
	$ \\ \\
	\left(
	\begin{array}{ccc}
	2&1&0 \\ 1&1&1 \\ 0&1&2 
	\end{array}
	\right)
	\cdot
	\left(
	\begin{array}{c}
	x \\ y \\ z
	\end{array}
	\right)
	=
	3 \cdot
	\left(
	\begin{array}{c}
	x \\ y \\ z
	\end{array}
	\right)
	$ \\
	Als LGS:  \\
	A1: 2x + y = 3x \\
	B1: x + y + z = 3y \\
	C1: y + 2z = 3z \\ \\
	Als homogenes LGS: \\
	A1: -x + y = 0 \\
	B1: x -2y + z = 0 \\
	C1: y = z \\
	C1 in A1 \\
	A2: -x + z = 0 \\
	A3: x = z
	A3 und C1 in B2 \\
	B2: z -2z + z = 0 \\
	B3: 0 = 0 \\
	Lösung in Abhängigkeit von z: \\
	$\vec{x}$ = 
	$\left(
	\begin{array}{c}
	z \\ z\\ z
	\end{array}
	\right)$
	$\vec{\lambda_{3}}$ =
	$\left(
	\begin{array}{c}
	1 \\ 1 \\1
	\end{array}
	\right)$
	\subsection*{b)}
	Wir recyclen die Eigenvektoren aus Aufgabe a und zeigen durch $T^{-1}$ $\cdot$ A $\cdot$ T das B eine Diagonalmatrix ist mit den Eigenwerten aus A auf ihrer Diagonalen. \\ \\
	T Ergibt sich aus den Eigenvektoren: \\ \\
	T = 
	$\left(
	\begin{array}{ccc}
	 1& -1&1\\ -2&0&1\\ 1&1&1
	\end{array}
	\right)$ \\ \\
	$T^{-1}$ ist entsprechend: \\
	$T^{-1}$ = 
	$
	\left(\begin{array}{ccc}
	\frac{1}{6} & \frac{-1}{3} & \frac{1}{6} \\
	\frac{-1}{2} & 0 & \frac{1}{2} \\
	\frac{1}{3} & \frac{1}{3} & \frac{1}{3}
	\end{array}\right)
	$ \\ \\
	B = 
	$
	\left(\begin{array}{ccc}
	\frac{1}{6} & \frac{-1}{3} & \frac{1}{6} \\
	\frac{-1}{2} & 0 & \frac{1}{2} \\
	\frac{1}{3} & \frac{1}{3} & \frac{1}{3}
	\end{array}\right)
	$
	$\cdot$
	$
	\left(\begin{array}{ccc}
	2 & 1 & 0 \\
	1 & 1 & 1 \\
	0 & 1 & 2
	\end{array}\right)
	$
	$\cdot$
	$
	\left(\begin{array}{ccc}
	1 & -1 & 1 \\
	-2 & 0 & 1 \\
	1 & 1 & 1
	\end{array}\right)
	$ \\
	B = 
	$
	\left(\begin{array}{ccc}
	\frac{1}{6} & \frac{-1}{3} & \frac{1}{6} \\
	\frac{-1}{2} & 0 & \frac{1}{2} \\
	\frac{1}{3} & \frac{1}{3} & \frac{1}{3}
	\end{array}\right)
	$
	$\cdot$
	$
	\left(\begin{array}{ccc}
	0 & -2 & 3 \\
	0 & 0 & 3 \\
	0 & 2 & 3
	\end{array}\right)
	$ \\
	B = 
	$
	\left(\begin{array}{ccc}
	0 & 0 & 0 \\
	0 & 2 & 0 \\
	0 & 0 & 3
	\end{array}\right)
	$
	\subsection*{Aufgabe 3}
	\subsection*{a)}
	Die Z werte einer Koordinate werden bei einer Multiplikation mit der Matrix $R_{z}$ nicht verändert. Durch entsprechende Multiplikationen mit Sinus und Kosinus der X und Y-Werte der entsteht eine Drehung. Insgesamt bewirkt die Matrix damit eine Drehung um die Z-Achse.
	\subsection*{b)}
	$R_{x} = 
	\left(
	\begin{array}{ccc}
	1&0&0 \\ 0&\cos(\alpha)&-\sin(\alpha) \\ 0&\sin(\alpha)& \cos(\alpha)
	\end{array}
	\right)
	$ \\
	$R_{y} = 
	\left(
	\begin{array}{ccc}
	\cos(\alpha)&0&\sin(\alpha) \\ 0&1&0 \\ -\sin(\alpha)&0&\cos(\alpha)
	\end{array}
	\right)
	$
	\subsection*{c)}
	$R_{y} = 
	\left(
	\begin{array}{ccc}
	\cos(60^{\circ})&0&\sin(60^{\circ}) \\ 0&1&0 \\ -\sin(60^{\circ})&0&\cos(60^{\circ})
	\end{array}
	\right)
	$  \\
	$A'	
	=
	\left(
	\begin{array}{ccc}
	\cos(60^{\circ})&0&\sin(60^{\circ}) \\ 0&1&0 \\ -\sin(60^{\circ})&0&\cos(60^{\circ})
	\end{array}
	\right)
	\cdot
	\left(
	\begin{array}{c}
	8\\6\\0
	\end{array}
	\right)
	=
	\left(
	\begin{array}{c}
	4\\54\\-4\sqrt{3}
	\end{array}
	\right)
	$ \\
	$B'	
	=
	\left(
	\begin{array}{ccc}
	\cos(60^{\circ})&0&\sin(60^{\circ}) \\ 0&1&0 \\ -\sin(60^{\circ})&0&\cos(60^{\circ})
	\end{array}
	\right)
	\cdot
	\left(
	\begin{array}{c}
	8\\10\\0
	\end{array}
	\right)
	=
	\left(
	\begin{array}{c}
	4\\58\\-4\sqrt{3}
	\end{array}
	\right)
	$ \\
	$C'	
	=
	\left(
	\begin{array}{ccc}
	\cos(60^{\circ})&0&\sin(60^{\circ}) \\ 0&1&0 \\ -\sin(60^{\circ})&0&\cos(60^{\circ})
	\end{array}
	\right)
	\cdot
	\left(
	\begin{array}{c}
	8\\10\\2
	\end{array}
	\right)
	=
	\left(
	\begin{array}{c}
	\sqrt{3}+4\\58\\-4\sqrt{3}+1
	\end{array}
	\right)
	$ \\
	$D'	
	=
	\left(
	\begin{array}{ccc}
	\cos(60^{\circ})&0&\sin(60^{\circ}) \\ 0&1&0 \\ -\sin(60^{\circ})&0&\cos(60^{\circ})
	\end{array}
	\right)
	\cdot
	\left(
	\begin{array}{c}
	8\\6\\2
	\end{array}
	\right)
	=
	\left(
	\begin{array}{c}
	\sqrt{3}+4 \\
	54 \\
	-4*\sqrt{3}+1
	\end{array}
	\right)
	$ \\ \\
	Die Ortsvektoren $A'$, $B'$, $C'$ und $D'$ können waagerecht als Bildpunkte der Punkte A, B, C und D gelesen werden. \\ \\
	$A'$ = ( 4 $\|$ 54 $\|$ $-4\sqrt{3}$ ) \\
	$B'$ = ( 4 $\|$ 58 $\|$ $-4\sqrt{3}$ ) \\
	$C'$ = ( $\sqrt{3}+4$ $\|$  58 $\|$ $-4\sqrt{3}+1$ ) \\
	$D'$ = ( $\sqrt{3}+4$ $\|$ 54 $\|$ $-4*\sqrt{3}+1$ ) \\
\end{document}