\documentclass{article}
\usepackage{mathtools}
\begin{document}
	\section*{Lsg Vorschlag LAÜ11 Maximilian Maag}
	\subsection*{Aufgabe 1}
	$\xi_{A}(\lambda)$ =
	det$\left(
	\left(
	\begin{array}{ccc}
	\frac{1}{5} & 0 & \frac{2}{5} \\ 0&1&0 \\ \frac{2}{5}&0&\frac{4}{5}
	\end{array}
	\right)
	-
	\left(
	\begin{array}{ccc}
	 \lambda&0&0\\ 0&\lambda&0 \\ 0&0&\lambda 
	\end{array}
	\right)
	\right)$ \\
	$\xi_{A}(\lambda)$ =
	det$\left(
	\begin{array}{ccc}
	\frac{1}{5} - \lambda & 0 & \frac{2}{5} \\ 0&1 - \lambda&0 \\ \frac{2}{5}&0&\frac{4}{5}  - \lambda
	\end{array}
	\right)$ \\
	$\xi_{A}\lambda)$ = $((\frac{1}{5} - \lambda) \cdot (1 - \lambda) \cdot (\frac{4}{5} - \lambda)) - (\frac{2}{5} \cdot (1 - \lambda) \cdot \frac{2}{5})$ \\
	$\xi_{A}(\lambda)$ = $-\lambda^{3} + 2\lambda^{2} - \lambda$ \\ \\
	Eigenwerte bestimmen \\
	\\
	$\lambda^{3} - 2\lambda^{2} + \lambda = 0$ \\
	$\lambda \cdot (\lambda^{2} - 2\lambda + 1) = 0$ \\
	$\lambda_{1}$  = 0 \\
	$\lambda^{2} - 2\lambda + 1 = 0$ \\
	$\lambda_{2} = 1 +-  \sqrt{1 - 1}$  \\
	$\lambda_{2} = 1$ Doppelte Nullstelle \\ 
	\\
	Eigenvektoren: \\ \\
	LGS1: \\
	A1: $\frac{1}{5}$ x + $\frac{2}{5}$z = 0 \\
	B1: y = 0\\
	C1: $\frac{2}{5}$ x + $\frac{4}{5}$z = 0 \\ \\
	$\vec{\lambda_{0}} =
	\left(
	\begin{array}{c}
	-2z\\0\\z
	\end{array}
	\right)$ 
	$\to$
	$\left(
	\begin{array}{c}
	-2\\0\\1
	\end{array}
	\right)$
	\\ \\
	LGS2: \\
	A1: $\frac{1}{5}$ x + $\frac{2}{5}$z = x \\
	B1: y = y\\
	C1: $\frac{2}{5}$ x + $\frac{4}{5}$z = z \\ \\
	A2: $-\frac{4}{5}$ x + $\frac{2}{5}$z = 0 \\
	B2: 0 = 0\\
	C2: $\frac{2}{5}$ x - $\frac{1}{5}$z = 0 \\ \\
	$\vec{\lambda_{1}} =
	\left(
	\begin{array}{c}
	\frac{1}{2}z\\0\\z
	\end{array}
	\right)$ $\to$
	$
	\left(
	\begin{array}{c}
	1\\0\\2
	\end{array}
	\right)$
	\subsection*{Aufgabe 2}
	Aus der Kurvendiskussion in Aufgabe 1 ergab sich eine doppelte Nullstelle daher gilt folgende lineare Zerlegung: \\ \\
	$(\lambda - 1)^{2} \cdot (-(\lambda-0)) = -\lambda^{3} + 2\lambda^{2} - \lambda$ \\ \\
	Der Eigenwert $\lambda_{1}$ ist zweimal enthalten, woraus sich folgende LGS Aussage ableiten lässt: \\
	Der Vektor $\vec{\lambda_{1}}$ hat nur eine linear unabhängige Lösung der Form
	$\left(
	\begin{array}{c}
	\frac{1}{2}z  \\ 0 \\ z
	\end{array}
	\right)$
	erscheint aber in der Linearfaktorzerlegung zweimal aus diesem Grund ist die Matrix A nicht diagonalisierbar. 
	\subsection*{Aufgabe 3}
	\subsection*{a)}
	$U' =$
	$\left(
	\begin{array}{cc}
	0 & -i \\ -i&0
	\end{array}
	\right)$ \\ \\
	$\left(
	\begin{array}{cc}
	0 & -i \\ -i&0
	\end{array}
	\right)$
	$\cdot$
	$\left(
	\begin{array}{cc}
	0 & i \\ i&0
	\end{array}
	\right)$ =
	$\left(
	\begin{array}{cc}
	1 & 0 \\ 0&1
	\end{array}
	\right)$
	\subsection*{b)}
	$U' = \frac{1}{2} \cdot
	\left(
	\begin{array}{cc}
	1-i & 1+i \\ 1 + i & 1 - i
	\end{array}
	\right)
	$ \\ \\	
	$\frac{1}{2} \cdot \left(
	\begin{array}{cc}
		1-i & 1+i \\ 1 + i & 1 - i
	\end{array}
	\right)
	\cdot
	\frac{1}{2}
	\left(
	\begin{array}{cc}
	1+ i & 1-i \\ 1 - i & 1 + i
	\end{array}
	\right)
	$ \\
	=
	$\frac{1}{2}$
	$\left(
	\begin{array}{cc}
	2 & 0  \\ 0 & 2
	\end{array}
	\right)$
	\\
	=
	$\left(
	\begin{array}{cc}
	1 & 0  \\ 0 & 1
	\end{array}
	\right)$
	
\end{document}