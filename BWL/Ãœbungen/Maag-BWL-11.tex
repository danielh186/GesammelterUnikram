\documentclass{article}
\usepackage{hyperref}
\begin{document}
	\section*{Lsg Vorschlag BWL Ü011 Maximilian Maag}
	\section*{Aufgabe 1}
	\begin{itemize}
		\item  Product - Das Iphone gibt es nur in luxuriösen Varianten zu kaufen. Apple würde nie eine Billigversion veröffentlichen.
		\item  Price - Apple gibt keine Rabatte.
		\item  Promotion - Apple verkauft ein Lebensgefühl. 
		\item  Place - Ladenstandort ist entscheidend für Absatz.
	\end{itemize}
	\section*{Aufgabe 2}
	Der Marketing-Mix koordiniert Maßnahmen um den Absatz eines Unternehmens anzuregen. \\
	Er besteht unter anderem aus Positionierung und Distribution. Anhand des Iphones lässt sich feststellen wie der Marketing-Mix bei Apple aussieht. \\
	Apple hat eine klare Positionierung als Livestyle und Luxusmarke. Daraus resultiert eine klare Preispolitik. Apple vergibt keine Rabatte. \\
	Die Vertriebsstrategie von Apple sieht eine weite Fächerung vor. Das Iphone kann überall gekauft werden. Es ist schwer zu sagen wo es nicht gekauft werden kann.
	\section*{Aufgabe 3}
	\begin{itemize}
		\item Finanzwirtschaft - Steuerung der Zahlungsströme eines Unternehmens mit dem Ziel die Zahlungsfähigkeit zu erhalten.
		\item Selbstfinanzierung - Der eigene Gewinn wird für die Finanzierung neuer Anschaffungen verwendet.
		\item Kapital - Summe der bewerteten Verpflichtungen gegenüber Eigentümern und Gläubigern.
		\item Anlagevermögen - Umfasst Güter die dauerhaft im Unternehmen angesiedelt sind. Zum Beispiel Fuhrpark, Grundstücke und Gebäude.
		\item Umlaufvermögen - Wirtschaftsgüter, welche das Unternehmen nach kurzer Zeit wieder verlassen bzw. nur kurz verbleiben. Zum Beispiel: Forderungen, Bargeldbestände, Wertpapiere.
	\end{itemize}
	\section*{Aufgabe 4}
	\begin{itemize}
		\item Unter Eigenkapital verstehen sich mit Geld bewertete Einlagen der Eigentümer eines Unternehmens.
		\item Gewinnrücklagen, Gewinn-/Verlustvortrag, Jahresüberschuss und -fehlbetrag.
	\end{itemize}
	\section*{Aufgabe 5}
	Unter Fremdkapital versteht man Verbindlichkeiten gegenüber dritten. Insbesondere unterscheidet sich das FK in seiner Laufzeit. Unterschieden wird langfristiges Fremdkapital und kurzfristiges Fremdkapital.
	
	\section*{Aufgabe 6}
	\begin{itemize}
		\item Um eine AG zu gründen wird in Deutschland ein Eigenkapital von 50.000€ benötigt.
		\item Um eine GmbH zu gründen benötigt man ein Eigenkapital von 25.000 €.
	\end{itemize}
	\section*{Aufgabe 7}
	\begin{tabular}[l]{l|l|l|l}
		Name & Eigenkapital (in Mio GE) & Fremdkapital (in Mio Euro) & EK-Qoute (in \%) \\
		\hline
		SAP & 30.822 &29.393& 51\\
		Facebook &128.290 & 31.026 & 80\\
		Alphabet (Google)&201.442&74.487& 73 \\
		Apple &65.339&258.549&20\\
		VW (Volkswagen AG)&110.988&406.710&21 \\
		
	\end{tabular} \\ \\
	Bei den vorliegenden Zahlen fällt auf, dass Unternehmen mit hoher Bilanzsummer eine geringe EK-Qoute aufweisen. 
	\section*{Aufgabe 8}
	\begin{tabular}[h]{c|c|c}
		Name & Umlaufvermögen & Anlagevermögen \\
		\hline
		SAP &15.213&45.002 \\
		Facebook &50.480&46.854 \\
		Alphabet (Google)&152.578&123.331 \\
		Apple &143.713&180.175 \\
		VW (Volkswagen AG)&187.463&300.608 \\
	\end{tabular} \\ \\
\end{document}