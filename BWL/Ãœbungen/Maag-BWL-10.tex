\documentclass{article}
\begin{document}
	\section*{Lsg Vorschlag BWL Ü010 Maximilian Maag}
	\section*{Aufgabe 1}
	Das PPS ist ein System um die Produktion zu planen und zu steuern. \\
	CIM (Computer integrated Management) bezieht Computer systematisch in das PPS ein. \\
	PPS bezieht sich auf die BWL-Sicht der Produktion. \\
	CIM mit allen Teilschritten (CAD,CAP,CAM,CAQ) beziehen sich auf die technische Sicht der Produktion und ermöglichen digitales design, digitale Planung, Überwachung und Qualitätssicherung.
	\section*{Aufgabe 2}
	Arbeitsproduktivität ist eine Kennzahl um die Produktivität von Mitarbeitern festzustellen, bezogen auf eine Kenngröße und einen bestimmten Zeitraum. \\ \\
	Beispielsweise Tonnen Stahl je Mitarbeiter pro Stunde. \\
	Anzahl Vertragsabschlüsse je Verkäufe pro Jahr.
	\section*{Aufgabe 3}
	\begin{itemize}
		\item Absatz, letzte Phase der Leistungserstellung. Verwertung der erstellten Leistung steht im Vordergrund.
		\item Marketing, kundenorientiertes Unternehmenskonzept. 
		\item Marktsegmentierung, Aufteilung des Marktes nach bestimmten Kriterien.
		\item Marketing-Mix, Kombination von produktbezogen Marketingmaßnahmen.
		\item Qualitätskontrolle mit dem Ziel der ständigen Verbesserung.
	\end{itemize}
	\section*{Aufgabe 4}
	Der Absatz ist die letzte Phase des betrieblichen Wertschöpfungsprozesses. In dieser entscheidenden Phase wird die erstellte Leistung in Geld umgewandelt.
	\section*{Aufgabe 5}
	Nachstehend möchte ich den Unterschied zwischen Absatz und Marketing darstellen. \\
	Marketing ist ein Konzept der Kundenorientierung. Das Unternehmen richtet sich nach dem Kunden aus. \\
	Absatz bezeichnet die Verwertung der erstellten Leistung und ist Teil des Marketings. \\
	Die Begriffe sind eng verknüpft es muss jedoch betont werden, dass Marketing ein umfassenderer Begriff ist.
	\section*{Aufgabe 6}
	\begin{itemize}
		\item a) Hohe Qualität ist kein Merkmal einer Marke mehr sondern eine Voraussetzung für eine Kaufentscheidung. Mundpropaganda über Social-Media wird immer wichtiger.
		\item b) Preispolitik würd eingeschränkt, da Preise in Echtzeit verglichen werden können. Es muss ein starker Markenname mit hoher Kundenloyalität geschaffen werden. 
		\item c) Social-Media-Kanäle müssen regelmäßig gepflegt werden.
		\item d) Kunden Daten müssen systematisch erfasst werden. Daten über das Kaufverhalten von Kunden ist besonders interessant. (Data Mining)
	\end{itemize}
	\section*{Aufgabe 7}
	Aus gesammelter Big Data soll Smart Data gewonnen werden, die als Entscheidungsgrundlage für Marketingmaßnahmen dienen kann.
	\section*{Aufgabe 8}
	\begin{itemize}
		\item Kaufverhalten
		\item Deckungsbeitrag
		\item geographische Kundensegmentierung
		\item Soziodemographische Kundensegmentierung
	\end{itemize}
	Mit dem AOI-Ansatz wird versucht Lebensstiele anhand von Aktivitäten zu bestimmen um werberelevante Zielgruppen gezielt anzusprechen. 
\end{document}