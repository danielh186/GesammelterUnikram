\documentclass{article}
\begin{document}
	\section*{Lsg Vorschlag BWL Ü005 Maximilian Maag}
	\section*{Aufgabe 1}
	Die Digitalisierung ist eine der wesentlichsten Transformationen des 21. Jahrhunderts. Nachfolgend beschreibe ich Veränderungen im technologischen, gesellschaftlichen, wirtschaftlichen und rechtlichem Umfeld. \\
	Im wirtschaftlichen Umfeld hat sich die Kommunikation durch die Digitalisierung entscheidend verändert. Online-Banking, Video-Chat, SAP-Resourceplaning ersetzen Vertreterbesuche, Bankbesuche und haben das papierbehaftete Rechnungswesen abgelöst. \\
	Im rechtlichen Umfeld stellen sich neue Fragen zum Thema digitale Identität und Urheberrecht. Große Videoplattformen und Streamingdienste stellen unmengen an Inhalten zur Verfügung, die unter Anderem auch urheberrechtlich geschützt sind. Die EU reagierte vor kurzem mit massiven Zensurbeschlüssen um diesem Problem schlecht als recht zu begegnen. \\
	Gesellschaftlich betrachtet rückt digitale Technik immer näher an den Menschen heran. Wir werden zunehmend abhängig von digitaler Technik. Navigation, Baning, Terminplanung, Einkauf etc. Es gibt keinen gesellschaftlichen Bereich, der nicht digital ist. \\
	Technologisch betrachtet müssen immer größer werdende Datenmengen immer schneller übertragen werden. Im Vergleich zum Kupfer hat Glasfaser die Geschwindigkeit mindestens vervierfacht. Alle die diesen Trend nicht mitmachen sind nicht zukunftsfähig und können zukünftige Dienste nicht mehr nutzen. Deutschland hat den Ausbautrend mit Glasfaser in den 90-ern leider verpasst und muss nacharbeiten um wettbewerbsfähig zu bleiben. \\
	Eine schlechte bis sehr schlechte Internetverbindung ist ein Kriterium für die Standortwahl eines Unternehmens.
	\section*{Aufgabe 2}
	Die Sozialversicherung teilt sich in Unfallversicherung, Rentenversicherung, Krankenversicherung, Arbeitslosenversicherung und Pflegeversicherung. \\
	Die Beitragshöhen gestalten sich wie folgt (in Prozent): \\
	\begin{itemize}
		\item KV 14,6 
		\item RV 18,6 
		\item PV 3,05
		\item AV 2,4
	\end{itemize}
Bis auf die Unfallversicherung und einige Zusatzbeiträge werden die Beiträge vom Arbeitnehmer und Arbeitgeber paritätisch geteilt.
		
	\section*{Aufgabe 3}
	
	Der Deutsche Gewerkschaftsbund ist die Dachorganisation der Arbeitnehmervertretung in Deutschland. Er vertritt die Interessen seiner Mitglieder und Arbeitnehmer gegenüber der Politik und gegenüber den Arbeitgebern. \\
	Die Bundesvereinigung Deutscher Arbeitgeberverbände vertritt die Interessen der deutschen Arbeitgeber. 
	
	\section*{Aufgabe 4}
	Eine Gesellschaft mit beschränkter Haftung. Sie zählt zu den Kapitalgesellschaften, sodass die Eigentümer (Gesellschafter) nicht mit ihrem Privatvermögen für Haftungen herangezogen werden können. Das gesetzliche Mindestkapital liegt bei 25.000 €. Die Geschäftsführung übernehmen die Gesellschafter in der Regel selbst. 2016 gab es in Deutschland ca. 1,15 Millionen GmbHs.
	\section*{Aufgabe 5}
	2019 hat Deutschland insgesamt 799,3 Mrd Euro Steuern eingenommen. Den Größten Anteil machen hierbei die Umsatzsteuer und die Lohnsteuer aus. Besonders relevant für Unternehmen sind die Gewerbesteuer, Lohnsteuer für Angestellte und die Einkommenssteuer.
	\section*{Aufgabe 6}
	MOST ist ein Planungsinstrument um verschiedene Planungsebenen zu verbinden. Aus einer Vision wird ein Objektiv abgeleitet, dieses wird in eine Strategie heruntergebrochen und Anschließend in konkrete Handlungsformen überführt (Tactics). Als Vision formuliert VW in seinem Nachhaltigkeitsbericht Klimaneutralität, daraus resultiert die Zielsetzung. Anschließend die Strategie auf Elektroautos zu setzen. Und zum Schluss die konkrete Handlung mehr Elektroautofabriken zu bauen. 
	\section*{Aufgabe 7}
	Wer als Kleinunternehmer gilt muss lediglich eine Einnahmen Ausgabenrechnung vorlegen und keine Mehrwertsteuervoranmeldung einreichen oder doppelte Buchführung vorlegen. Das soll vor allem kleinen Firmen Bürokratie ersparen. Als KU gilt wer nach § 19 UStG im letzten Jahr nicht mehr als 22000 € Gewinn verzeichnet hat und im laufenden Kalender nicht mehr als 50000 € Umsatz verzeichnet. \\
	Harte Standortfaktoren sind quantifizierbare Eigenschaften eines Standortes, die für unternehmerisches Handeln eine Rolle spielen. Klassischerweise handelt es sich um die Verkehrsanbindung für eine Fabrik. Durch die Digitalisierung wird die verfügbare Bandbreite an einem Standort ebenfalls zu einem Standortfaktor. \\
	Katalogberufe sind freiberufliche Tätigkeiten, welche auch ohne Anmeldung ausgeübt werden können. Das EstG listet diese Berufe genau auf. Beispielsweise der Diplom Informatiker.
\end{document}