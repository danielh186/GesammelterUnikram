\documentclass{article}
\begin{document}
	\section*{Lsg Vorschlag BWL Ü009 Maximilian Maag}
	\section*{Aufgabe 1}
	\begin{itemize}
		\item Produktionsfaktoren: Arbeit, Betriebsmittel, Werkstoffe insb. Elementarfaktoren fließen direkt ein. Bsp. menschliche Arbeit, Montage. Dispositive Faktoren die nicht direkt einfließen. Bsp: Planung/Gestaltung der Produktion.
		\item Fertigungstypen: Massennfertigung bzw. Seriienfertiung insbs. Werkstattfertigung, Baustellenfertigung, Fließbandfertigung
		\item Leistung Transformation von Input zu Output.
		\item Produktlebenszyklus: Beschreibt den Prozess eines Produktes von der Markteinführung bis zur Herausnahme. 
	\end{itemize}
	\section*{Aufgabe 2}
	Produktionsfaktoren: \\
	menschliche Arbeit \\
	Organisation, Kontrolle, Leistung, Überwachung
	Werkstoffe \\
	Roh- Hilfs- und Betriebsstoffe
	Betriebsmittel
	
	\section*{Aufgabe 3}
	Die Begriffe Produktion und Fertigung unterscheiden sich wie folgt: \\
	Fertigung bezeichnet die Herstellung eines Produktes. Produktion umfasst jedoch den gesamten Herstellungsprozess. Dazu gehören auch die Planung und Verwaltung der Herstellung.
	\section*{Aufgabe 4}
	\begin{itemize}
		\item Terminplanung
		\item Zeitplanung
		\item Kapazitätsplanung
	\end{itemize}
	\section*{Aufgabe 5}
	Durch die Digitalisierung in der Produktion ergeben sich unter anderem verschiedene Veränderung zwei davon möchte ich näher beleuchten. \\
	Zum einen ergibt sich durch einen hohen Grad der Automatisierung in nahezu allen operativen Aufgabenfeldern ein weitaus geringerer Bedarf an menschlicher Arbeit. Der Mensch wird immer mehr zum dispositiven Faktor und muss die Produktion nicht mehr selbst durchführen sondern planen und überwachen. \\
	Zum Anderen ergibt sich die Abhängigkeit von Echtzeit. Daraus folgt ein hoher bedarf an Bandbreite und ein leistungsfähiger Internetanschluss wird zum Standortfaktor.
	\section*{Aufgabe 6}
	Eine Smart Factory zeichnet sich neben dem hohen Automatisierungsgrad durch eine systematische Vernetzung innerhalb der Fabrik Schnittstellen zu allen wichtigen externen Akthören aus. \\
	Smart Production bezeichnet einen vollautomatisierten Prozess der sich selbst überwacht und reguliert. 
	\section*{Aufgabe 7}
	Industrie 4.0 ist ein Prozess in dessen Rahmen die Industrie insbesondere die Produktion voll digitalisiert werden soll. Um Smart Production real werden zu lassen müssen Unternehmen innerhalb einer Wertschöpfungskette standardisierte Schnittstellen nutzen um ein Netzwerk zu bilden. Das ist eine wichtige Grundlage um Information innerhalb der Wertschöpfungskette reibungslos auszutauschen. \\
	In Zukunft wird ein Produkt Träger von Dienstleistungen werden. Unternehmen müssen flexibler werden um sich auf neue Anforderungen einstellen zu können.
	\section*{Aufgabe 8}
	CPM steht für die Critical-Path-Method und ist in der Produktionsplanung angesiedelt. Die Idee dahinter besteht darin den längsten Produktionsschritt zu erst zu beginnen und so viele Teilschritte wie möglich parallel zu bewältigen um Durchlaufzeiten von Produkten zu minimieren. 
\end{document}