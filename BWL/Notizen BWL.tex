\documentclass{article}
\usepackage{hyperref}
\begin{document}
	\section*{Notizen BWL}
	\subsection*{18.11.2020}
	Aktuelles aus der Wirtschaft Gewinner und Verlierer.
	Plattformeigentümer sind immer Gewinner.
	Aktienhandel große IT Unternehmen. Kapitalgesellschaften. \\
	Renditen und Kredite. Bank verdient an der Differenz. \\
	"Nichts ist so Beständig wie der Wandel" \\
	Digitalisierung, Klimawandel, Globalisierung \\
	Erfolgsmessung Kosten, Qualität, Zeit \\
	Ökonomisches Handeln über Maximalprinzip. Vorhandene Ressourcen optimal nutzen. \\
	Ökonomisches Prinzip im Spannungsfeld. Magisches Dreieck. \\
	Kennzahlen: Produktivität, Wirtschaftlichkeit, Eigenkapitalrentabilität, Gesamtkapitalrentabilität \\
	Nachhaltigkeit: Sozial, Ökologie, Wirtschaft. \\
	Wirtschaft: private Haushalte; Betriebe - Gewinn und öffentlich; \\
	Ewerbswirtschaftlich und nicht erwerbswirtschaftlich \\
	BWL = Betriebliche Einzelanalyse Einzelwirtschaftlich \\
	VWL makroökonomische Analysen. Gesamtwirtschaftlich \\
	
	\subsection*{Allgemeine BWL}
	Allgemeine BWL ist verallgemeinert. Funktionslehre \\
	Beschaffung, Absatz, Produktion \\
	Finanzwirtschaft Eigenkapital Fremdkapital \\
	Personalwirtschaft \\
	Informationswirtschaft \\
	Strategische Planung aktuelle Planung Home Office \\
	Leistungsfunktionen Produkterstellung \\
	POSDCORB \\
	Transformationsprozess, Faktoreneinsatz Outputgüter erzeugen \\
	\subsection*{}
	
	Transformationsprozess Input wird in Output umgewandelt, Faktoren werden in Ertrag umgewandelt. \\
	Output wird auch als  Faktorenertrag bezeichnet \\
	Wertschöpfung  Transformation schafft einen Wert über Input. \\
	Wertschöpfung Differenz zwischen Input und Output. \\
	Verwendung Wertschöpfung hps. Bezahlen von Mitarbeitern \\
	Übersicht Verwendung Wertschöpfung \\
	
	\subsection*{Betriebstypen}
	Leistungserstellung \\
	Betriebsgröße usw. \\
	Wie lassen sich Unternehmen unterscheiden?
	HGB = Handelsgesetzbuch \\
	Regelt den Handel und baut auf das BGB auf \\
	HGB § 267 unterscheidet Görße von Kapitalgesellschaften \\
	3,5 Mio Unternehmen in Deutschland. \\
	99,4 der deutschen Wirtschaftsleistung wird von KMUs erbracht \\
	KMUs sind kleine und mittelständische Unternehmen \\
	KMUs erzeugen jedoch nur $\frac{1}{3}$ des Umsatzes.
	Art der Leistungserstellung \\
	Massenfertigung, Individualfertigen, Sortenfertigung, Maschinenfertigung, Variantenfertigung  \\
	\subsection*{}
	Organisationstypen \\
	Fliesband-, Gruppen-, Werkstattfertigung \\
	Lesen bis S 50
	\section*{25.11.2020}
	aktuelles Black Friday und Gaja-x \\
	Input Transformation output \\
	Wertschöpfung ist die Differenz des outputs und inputs. \\
	Informatiker ist an der Wertschöpfung beteiligt. \\
	Je die Wertschöpfung desto höher der Gewinn \\
	Gewinn ist ein Teil der Wertschöpfung. \\
	
	Betriebstypen Leistungserstellung, Leistungsprogramm, Wirtschaftszweig, Betriebsgröße, vorherrschender Produktionsfaktor. \\
	HGB Handelsgesetz \\
	VWL gesamtwirtschaftlich \\
	BWL betriebswirtschaftlich \\
	Theoriebildung \\
	Erklärung, Prognosefunktion, Technologische Funktion \\
	Induktiv vom besonderen wird auf allgemeines geschlossen. \\
	deduktiv Erkenntnis wird abgeleitet aus bekannten Zusammenhängen. \\
	Das Umfeld von Unternehmen. \\
	rechtliches Umfeld. \\
	wirtschaftliches Umfeld. \\
	Rechtsordnung, Staat \\
	Technologie \\
	Stakeholderansatz jede Organisation hat unterschiedliche Interessen. Gegenteil Shareholder. \\
	Shareholder will hohen Unternehmenswert \\
	Öffentlichkeit zum Beispiel FFF. \\
	Stakeholder sind Anspruchsgruppen die heterogen sind.  \\
	Shareholder besitzen ein Unternehmen, \\
	Stakeholder sind mit dem Unternehmen verbunden. \\
	CSR \\
	tripple bottom line Messung der drei Säulen der Nachhaltigkeit \\
	CSR auch als ISO Norm. \\
	Struktur und Prozesse \\
	Säulen der Sozialversicherung. \\
	Magisches Viereck nach dem Stabilitätsgesetz
	\section*{02.12.2020}
	aktuelles Teslafabrik in Brandenburg \\
	Strategische Planung \\
	Unternehmenspyramiede \\
	Ziele rechtfertigen Handlungen \\
	Sachziel, Formziel \\
	Ziele müssen operabel sein. \\
	Kapitaleigner sind shareholder \\
	Stake- und Shareholder haben Einfluss auf Zielsetzung. \\
	Zielbildungsprozess Koalitionsansatz \\
	\section*{09.12.2020}
	aktuelles aus BWL Forschung \\
	gesellschaftliche Verantwortung Themenüberlauf \\
	Kapitel 4 Organisation. \\
	Management als Funktion Führung durch Führungskräfte. \\
	Ebenen der Unternehmensführung \\
	Normativ, Strategie, Operativ \\
	Pyramide mit Normativ als Spitze \\
	Normativ ist der Kern der Firma \\
	MOST \\
	Mission/Vision, Objectives, Strategie, Tactics \\
	Mission ist die Umsetzung der Vision. \\
	Vision Bekennung zu einem Ziel. \\
	Strategischer Planungsprozess \\
	Mission/Vision in Ziele umsetzen.  \\
	Zielsystem, Atribute des Zielsystems. \\
	Führt dazu das Beziehungen zwischen Zielen ermittelt werden können. \\
	Umfeldanalyse. \\
	PORTER five forces Modell. Systematische Analyse des Umfeldes eines Unternehmens. \\
	Portfolioanalyse. \\
	Def. Organisation ist das Regelwerk über ein Arbeitsteiliges System innerhalb einer Firma. \\
	Aufbau und Ablauf müssen organisiert werden. \\
	Dominanz des Prozesses. \\
	Managementfunktion = Organisation Aufbau und Ablauf (Prozessse)\\
	\section*{16.12.2020}
	Aktuelles aus der W: Lieferkettengesetz. \\
	Block-chain: https://www.youtube.com/watch?v=4Eoela-Ai-o \\
	Management: Steuerung, Überwachung, Organisation \\
	Qualitätsmanagement \\
	organisatorische Analyse; organisatorische Synthese \\
	Aufbauorganisation, Ablauforganisation, Prozessorganisation \\
	Prozesse können nicht ohne Ablauf organisiert werden. \\
	Prozess orientiert sich am Ablauf und Aufbau. \\
	Primärorganisation organisiert alles unbefristete. Sekundärorganisation regelt alle zeitlich begrenzten Abläufe. \\
	Typischerweise Projekte. \\
	Optimaler Organisationsgrad: nicht zu viel nicht zu wenig. \\
	Matrixorganisation \\
	Enthält Produktgruppen. \\
	Funktionsorganisation \\
	Weisungsbefugnis \\
	Einliniensystem: Jeder Mitarbeiter hat einen Vorgesetzten. \\
	Mehrliniensystem: Mehrfachunterstellungen. \\
	Stabliniensystem: Einliniensystem mit Stabstellen. \\
	Leitungsspanne und -tiefe \\
	Hierachietypen Zentralisation, Partizipation, Delegation \\
	Prozesse sind immer kundenbezogen. \\
	\section*{Materialwirtschaft und Beschaffung}
	ist kein Synonim. Ein Teil der Beschaffung ist die Materialwirtschaft \\
	ABC-Analyse \\
	E-Precurement \\
	internationale Beschaffung \\
	\subsection*{06.01.2021}
	Aktuelles Einzelhandel viele Verlierer. \\
	vor 100 Jahren Warenhäuser, vor 40 Jahren Discounter, 80-er Verwandhäuser um 20000 Gründung Amazon E-Bay exponentielles Wachstum. \\
	Multi-Chanel-Marketing \\
	Verschiebung von Einzelhandel in Online-Handel \\
	Organisation \\
	Funktionsorierntierung Prozessorientierung \\
	Unternehmen als Pyramide ganz unten operatives Geschäft Spitze Management steuert dispositiv \\
	Leistungserstellung: Beschaffung, Produktion, Absatz \\
	Zielkonflikte Materialwirtschaft \\
	Bruttoproduktionswert \\
	Zukauf von Außen \\
	
	Funktionsorientierung= Bereiche optimieren
	Prozessorientierung = Gesamtoptimum \\
	
	Zielkonflikte.
	Als Manager: niedrige Kosten
	Man will aber eine hohe quali
	Zulieferer muss auch willig sein zuzuliefern \\
	
	Als Manager: Strategische Ziele bei den Zielkonflikten beachten. (Wenn man nur das hochwertigste produzieren will, dann <Qualität) \\
	Make or Buy \\
	Qualitätssmanagement \\
	Totall-Quality-Management Qualität ist die Aufgabe aller Mitarbeiter. Kundenorientierung, hoher Nutzen \\
	Qualität muss im passenden Verhältnis zu Kosten stehen, Kunde muss dies bezahlen. \\
	ABC-Analyse \\
	A Produkte 80 \% Wertanteil am Produkt \\
	B Produkte 10 \% Wertanteil aber nur 30 \% der Gütermenge \\
	C Produkte 10 \% Wertanteil 50 \% Anteil an Materialmenge \\
	\section*{13.01.2021}
	Produktion \\
	Selbständige Steuerung Produktion. Industrie 4.0 \\
	Vollautomatisierung \\
	Big Data \\
	Alles hängt von Echtzeit ab. \\
	Internet of Things \\
	Smart "mentainments" \\
	Immer warten. \\
	SCM untersucht Prozesse entlang einer Wertschöpfungskette \\
	Schnittstellen Unternehmen \\
	Custermore Relationship Management CRM
	Supply Chain Management SCM \\
	ERP Enterprise Ressource Planing \\
	Industrie 4.0 Link: \url{https://www.digital-in-nrw.de/de/digitalisierung-live} \\
	Kap. 6 Produktionswirtschaft \\
	Absatzmarkt \\
	Industrie 4.0 \\
	Klausurrellevant: Produktsionsfaktoren \\
	Dienstleistungen \\
	Prozess zielgerichteter Kombination von Produktionsfaktoren. \\
	Werkstoffe, Betriebsmittel, menschliche Arbeit. \\
	Elementarfaktoren Leistungserstellung im operativen Sinne. Fließen unmittelbar ein. \\
	Dispossitive Faktoren Verwaltung, Planung. Fließen nicht direkt ein. \\
	Betriebsmittel ermöglichen Betrieb. \\
	Betriebsstoffe Verbrauchsstoffe für den Betrieb. \\
	Roh- und Hilfsstoffe gehen direkt in das Produkt ein. \\
	Klausur: Werkstoffe, Betriebsmittel, menschliche Arbeit sind elementar. \\
	Wirtschaftlichkeit = $\frac{finanzieller Output}{finanzieller Input}$ \\
	Produktivität  = $\frac{Outputmenge}{Inputmenge}$ \\
	Wirtschaftlichkeit muss im Optimalfall $\geq$ 1 sein. \\
	Produktionsplanung \\
	Auftrag Zeit  \\
	Terminplanung CPM \\
	\section*{20.01.2021}
	aktuelles aus der Wirtschaft neues Elektroauto von Apple. Das Icar. \\
	Produktionsfaktoren: Betriebsmittel, Werkstoffe, Menschliche Arbeit \\
	Output ist Faktorenertrag. \\
	TQM kundenorientiert - IMMER! \\
	\\
	Humanisierung der Arbeit \\
	Beispiel Änderung der Arbeitsteilung \\
	CIM Computer Integrated Manufacturing \\
	BWL-Sicht PPS System Produktionsplanung und Steuerung \\
	technische Sicht CAD, CAP, CAM, CAQ \\
	Das Y-Integrationsmodell \\
	Technik und BWL kommt zusammen. \\
	Digitalisierung Potenziale \\
	\\
	Absatzwirtschaft \\
	letzte Phase der Wertschöpfung \\
	Orientierung immer am Nachfrager. \\
	Konsequente Ausrichtung anhand des Kunden. \\
	4 Ps im Marketing sind folgende Instrumente: Produkt, Preis, Promotion, Place \\
	Bedrohung und Chances des Marktes müssen laufend erkannt werden. \\
	Five-Forces von Porter \\
	\section*{27.01.2021}
	Volkswirtschaft Indikator unter Anderem BIP. \\
	Lean Production \\
	Nicht wertschöpfende Prozesse eliminieren bzw. so gering wie möglich. \\
	Beispiel Digitalisierung Briefmarke. \\
	Anstatt Durck nur noch ein Code der auf den Brief geschrieben wird. \\
	Leistungswirtschaftliche Prozesse: Beschaffung, Produktion, Absatz \\
	4P: Product, Price, Promotion, Place \\
	Marktsegmentierung: Aufteilung des Marktes in Segmente (Käufergruppen) \\
	Positionierung - Abgrenzung vom Konkurrenten \\
	Positionierung muss an Kundenbedürfnissen orientiert sein. \\
	Apple hat eine klare Positionierung \\
	Markenbekanntheitspyramiede \\
	Antithese Nokia \\
	Ziele der Positionierung: kognitives Erlebnis, emotionales Erlebnis \\
	Komponenten des Marketings: Preis, Distribution, Produktpolitik, Kommunikationspolitik \\
	Distributionsmix: Direktabsatz sofort an den Endverbraucher, indirekter Absatz: Verkauf über den Handel. \\
	\\
	Finanzwirtschaft \\
	Innen- und Außenfinanzierung \\
	Beschaffung von Fremd und Eigenkapital. \\
	Außenfinanzierung: z.B. Kredit \\
	Innenfinanzierung: z.B. Ausgabe von Aktien
	Ausgleich der Finanzströme. Alle Zahlungsverpflichtungen müssen erfüllt werden. \\
	EK Rentabilität = $\frac{Gewinn}{EK}$ \\
	Kassenüberschuss = Cashflow \\
	Kapitalbedarf Differenz zwischen Einnahmen und Ausgaben. \\
	\section*{03.02.2021}
	Aktuelles Finanzwirtschaft Commerzbank baut 10000 Stellen abbauen. \\
	Weniger als die Hälfte der Fillialen bleibt übrig. \\
	strategische Planung Management. \\
	Kapital fremd von außen eigen von Gesellschaftern. \\
	Finanzierung. \\
	Ohne Finanzierung keine Firma. \\
	Finanzierung gleicht Finanzströme aus. \\
	Kenngröße Rentabilität EK $\frac{Gewinn}{EK}$ \\
	Rentabilität $\frac{Gewinn}{Kapital}$ \\
	Cash-Flow Kassenüberschuss in einer Zeiteinheit. \\
	Verwendung: Durchführung von Investitionen, Auszahlung an Gesellschafter, Schuldentilgung \\
	Kapitalbedarf \\
	Ausgaben und Einnahmen sind zeitversetzt. Versatz muss überbrückt werden. \\
	Finanzierung langfristig und kurzfristig \\
	Finanzierungsformen: Innen und Außenfinanzierung kann unterteilt werden eigen und Fremdfinanzierung. \\
	share-holder-value nicht vergessen \\
	Beteiligungsfinanzierung \\
	Charakteristiken Kapital. \\
	Besondere Unterschiede Laufzeit, Rückzahlung Stellung \\
	Verzinsung bei EK erfolgsabhängig, FK fest. \\
	Finanzierung aus Hybridfinanzierung möglich. \\
	Hybridanleihe. \\
	Kapitalstruktur Verhältnis zwischen EK und FK relativ gesehen. \\
	optimale Kapitalstruktur kommt shareholder-value-konzept entgegen. \\
	goldene Bilanzregel langfristiges Kapital langfristig finanzieren. \\
	Investitionen finanziell, immaterial oder in Sachgüter. \\
	\section*{10.02.2021}
	Aktuelles: Finanzierung ohne Bank möglich. \\
	Finanzwirtschaft \\
	Anlageklassen: Aktien, Anleihen, Spareinlagen \\
	EK FK Unterschiede Klausurrelevant. \\
	Investitionen Sachinvestition, Ersatz oder Erweiterung. \\
	Investitionsrechnung \\
	Absolute Vorteilhaftigkeit: Rendite \\
	Relative Vorteilhaftigkeit:  \\
	indirekte Finanzierung: Man in the Middel häufig Bank oder Börse vermittelt Kapital. \\
	Direktfinanzierung: Face to face \\
	Rechnungswesen als teil des Controllings \\
	Controlling ist teil des Führungssystems. \\
	Rechnungswesen umfasst alle quantifizierbaren Beziehungen. \\
	Rechnungswesen extern und intern. \\
	extern für Finanzamt nach HGB streng geregelt. \\
	intern freiwillig. \\
	Controlling und Rechnungswesen liefern Informationen Management entscheidet. \\
	wichtigste Information ist das Rechnungswesen. \\
	Bilanz, JÜ Jahresüberschuss, GuV Gewinn und Verlustrechnung \\
	Strömungsgrößen, stetige Veränderung \\
	Bestandsgrößen, zu einem Stichzeitpunkt gibt es einen zählbaren bestand. \\
	Aufwand und Ertrag sind Strömungsgrößen \\
	Abschreibungen. \\
	
\end{document}