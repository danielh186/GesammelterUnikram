\documentclass{article}
\usepackage{hyperref}
\begin{document}
	\section*{Lsg Vorschlag BWL Ü013 Maximilian Maag}
	\section*{Aufgabe 1}
	Die Investitionsrechnung hat die Aufgabe das die lukrativste Investition zu ermitteln um das Investitionsprogramm auflegen zu können. Statische und dynamische Methoden können dabei entweder eine einfache Näherung oder eine komplexe Berechnung unter Berücksichtigung zeitlicher Veränderung liefern. \\ \\
	Der Return of Investment gibt den Gewinn am eingesetzten Gesamtkapital in Prozent an. \\ \\
	$ROI = \frac{Gewinn}{Gesamtkapital} * 100$
	\section*{Aufgabe 2}
	$ROI = \frac{60000}{400000}*100$ \\
	$ROI = \frac{6}{40}*100$ \\
	$ROI = \frac{60}{4}$ \\
	$ROI = 15 \%$
	\section*{Aufgabe 3}
	Der Jahresabschluss muss mindestens eine Gewinn und Verlust Rechnung sowie eine Bilanz zum Stichtag am 31. Dezember aufweisen. \\ \\
	Die Bilanz gibt Auskunft über das Gesamtvermögen eines Unternehmens sowie dessen Herkunft. \\
	Die GuV listet Einnahmen und Ausgaben in der laufenden Abrechnungsperiode auf.
	\section*{Aufgabe 4}
	Kennzahlen in Mio € 2019: \\
	Umsatzerlöse: 252.632 \\
	Operatives Ergebnis: 16.960 \\
	Kosten der Umsatzerlöse: 203.490 \\
	Anlagevermögen: 300.608 \\
	Umlaufvermögen: 187.463 \\
	Bilanzsumme: 488.071\\
	FK langfristig: 196.497 \\
	FK kurzfristig: 167.924 \\ \\
	Quelle: \hyperlink{https://www.volkswagenag.com/presence/investorrelation/publications/annual-reports/2020/volkswagen/Y_2019_d.pdf agerufen: 16.02.2021}{Jahresabschluss Volkswagen AG}
	\section*{Aufgabe 5}
	GuV steht für Gewinn und Verlustrechnung. \\
	Aufwendungen sind fortlaufende Ausgaben, sie haben keinen Bestand und gehören zu den Erfolgskonten. Erträge sind fortlaufende Einnahmen, sie haben keinen Bestand und gehören ebenfalls zu den Erfolgskonten. \\
	Erträge stehen auf der Passivseite der GUV und Aufwändungen auf der Aktivseite der Bilanz. \\
	Der Gewinn ist im Beispiel 0 €, da Aufwände und Erträge gleich groß sind.
	\section*{Aufgabe 6}
	\begin{tabular}{|c|c|c|}
		& Externes Rechnungswesen & Internes Rechnungswesen \\
		\hline
		Ziele & Rechnungslegung nach Außen.& Gestaltung der internen Kostenstruktur. \\
		Regeln & klar geregelt durch HGB& Betriebliche Willkür\\
		Rechnungsgrößen & Bilanz GuV & frei Wählbar 
	\end{tabular}
	\section*{Aufgabe 7}
	Die Digitalisierung ermöglicht im Rechnungswesen die Erhebung von Daten in Echtzeit. \\
	Darüber hinaus werden Belege maschinell erfasst und Buchungen durch ERP-Systeme erfasst wodurch sich eine erhebliche Personalersparnis ergibt. 
	\section*{Aufgabe 8}
	Enterprice Ressource Planing Planing Systeme können in Echtzeit Daten innerhalb des Unternehmens erfassen und mit einem Sollzustand vergleichen. Außerdem enthalten sie in der Regel Module für das Rechnungswesen, den Zahlungsverkehr und können darüber hinaus individuell erweitert werden. \\
	Die Bekanntesten Vertreter sind: SAP und Navision. \\
	Der größte Vorteil liegt definitiv in der nahezu vollständigen Automatisierung des Rechnungswesens, dass nur noch überwacht werden muss und in Echtzeit abgewickelt wird. \\
	ERP-Systeme sind allgegenwärtig und automatisieren die Kernaufgaben der BWL.
\end{document}