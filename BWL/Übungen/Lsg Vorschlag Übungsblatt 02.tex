\documentclass{article}
\usepackage{hyperref}
\begin{document}
	\section*{Lösungsvorschlag BWL 02 Maximlian Jakob Maag}
	
	Ich habe das Buch Einführung in die Betriebswirtschaft ausgeliehen und werde es nach bestem Wissen und Gewissen nutzen.
	
	\subsection*{Aufgabe 1}
	
	Kompetenzen sind Fertigkeiten und Fähigkeiten - sowie Wissen und Kenntnisse, welche in Arbeits- und Lernsituationen in nachweisbaren und messbaren Erfolg umgesetzt werden können.
	
	\subsection*{Aufgabe 2}
	Die BWL ist zusammen mit der VWL teil der Wirtschaftswissenschaften und befasst sich mit dem Wirtschaften.  BWL befasst sich mit dem Wirtschaften innerhalb eines Betriebes. BWL lehrt die Organisation bestimmter Funktionen innerhalb eines Betriebes. Hierzu zählen zum Beispiel die Produktiosnplanung, das Rechnungswesen oder die Kosten und Leistungsrechnung mit dem Ziel der Bedürfnisbefriedigung und Gewinnorientierung.
	
	\subsection*{Aufgabe 3}
	Aus meiner sicht ist ein Unternehmen erfolgreich, wenn es in der Lage ist an sich gerichtete Bedürfnisse zeitnah zu erfüllen. Darüber hinaus sollte es einen ausgeglichenen Jahresabschluss vorlegen können und stets über liquide Mittel verfügen.
	\subparagraph*{}
	Beispielhaft lässt sich hier die Conrad SE nennen. Diese ist in ganz Europa im Vertrieb verschiedener Elektronikbauteile aktiv und hat sich seit Ihrem bestehen als starke Handelsmarke für Elektronikbauteile platziert.
	\subparagraph*{}
	Ferner ist aus meiner Sicht die Saturn Mediamarkt Gruppe erfolgreich weil diese nach einer Neuausrichtung im Jahr 2018/19 Verstärkt auf den Verkauf passender Dienstleistungen rund um Konsumelektronik setzt um Ihre Kundenbedürfnisse gerechter zu werden. 
	
	\subsection*{Aufgabe 4}
	Das Allgemeine Managementdreieck stützt auf drei Schwerpunkte. Die Leistung, die ein Unternehmen erbringt. Die Dafür aufgewendeten Kosten sowie die erzielte Qualität. Alle Drei Ecken des Dreiecks widersprechen sich gegenseitig und das Unternehmen muss sein Handeln gemäß seiner Ziele innerhalb dieses Dreiecks austarieren. Um mit einer optimalen Mischung aus Qualität, Kosteneffizienz und Leistung am Wettbewerb bestehen zu können.
	
	\subsection*{Aufgabe 5}
	
	Gablers Wirtschaftslexikon sieht die BWL als Wissenschaft um Handlungsweisen von Kaufläuten zu verstehen und zu lehren. Erste Formen der Rechnungslegung finden sich bereits im alten Orient ca. 3.000 v.Chr. BWL ist heute eine moderne Wissenschaft und Lehre rund um das unternehmerische Handeln.
	
	\subsection*{Aufgabe 6}
	
	SAP erzielte im Jahr 2019 weltweit einen Umsatz von 27,55 Mrd Euro und wird gegenwärtig mit einem Marktwert von insgesamt 138,9 Mrd USD bewertet und ist damit das wertvollste Unternehmen Deutschlands.
	\href{https://www.gevestor.de/details/die-top-10-der-wertvollsten-unternehmen-in-deutschland-718613.html}{Marktwert SAP} \href{https://de.statista.com/statistik/daten/studie/28261/umfrage/umsatz-des-unternehmens-sap-seit-dem-jahr-2001/
	}{SAP Umsatz 2019}
	
	\subsection*{Aufgabe 7}
	VWL ist teil der Wirtschaftswissenschaft und befasst sich mit den Akthören einer Volkswirtschaft. In der Abgrenzung zur BWL werden Haushalte, Unternehmen und Staaten als ganzes betrachtet. VWL dient in der Praxis als Entscheidungsgrundlage wirtschaftspolitischer Entscheidungen. Bekannte Größen der VWL sind John Maynard Keynes mit der Theorie des Zinses, des Geldes und der Beschäftigung. Er gilt als Begründer der nachfrageorientierten Wirtschaftspolitik. Sein größter Gegenspieler war Milton Friedman, der Begründer des Monetarismus.
	
	\subsection*{Aufgabe 8}
	In der Bundesrepublik Deutschland liegt die Arbeitslosenquote gegenwärtig bei 6 Prozentpunkten.
	\href{https://de.statista.com/statistik/daten/studie/1239/umfrage/aktuelle-arbeitslosenquote-in-deutschland-monatsdurchschnittswerte/}{Mehr dazu hier.}

	\subparagraph*{}
	Im September 2020 sind in der Bundesrepublik Deutschland 44,58 Millionen Menschen abhängig beschäftigt.	
	\href{https://de.statista.com/statistik/daten/studie/1376/umfrage/anzahl-der-erwerbstaetigen-mit-wohnort-in-deutschland/
	}{Mehr dazu hier.}
	
\end{document}