\documentclass{article}
\begin{document}
	\section*{Lsg Vorschlag BWL Ü008 Maximilian Maag}
	\section*{Aufgabe 1}
	Beschaffung umfasst die Bereitstellung aller notwendigen Input-Faktoren einer Unternehmung. In der engsten Fassung umfasst dieser Begriff jedoch nur die Bereitstellung von Roh- Hilfs- und Betriebsstoffen. \\
	Betriebsstoffe sind Werksstoffe, die im laufenden Betrieb verbraucht werden um den Betrieb zu ermöglichen. Zum Beispiel Energie. \\
	Hilfsstoffe sind Nebenprodukte die unmittelbar in ein Produkt eingehen. Z.B. Leim. \\
	Die Materialwirtschaft umfasst alle Aufgaben der Materialbereitstellung insbesondere Einkauf, Lagerung, Logistik, Entsorgung. \\
	SCM steht für Supply Chain Management und betrachtet die ganzheitliche Optimierung der Wertschöpfungskette. \\
	Die ABC-Analyse ist ein Verfahren zur Klassifizierung von Objekten nach ihrer Wichtigkeit.
	\section*{Aufgabe 2}
	Minimale Bestellkosten \\
	Minimale Lagerungskosten \\
	Maximale Versorgungsbereitschaft
	\section*{Aufgabe 3}
	\begin{itemize}
		\item 1. Ordnungskriterium festlegen. 
		\item 2. Prozentualer Anteil am Gesamtvolumen ermitteln.
		\item 3. Daten Sortieren nach Ordnungskriterium. 
		\item 4. Sortierte Werte auf kumulieren.
		\item 5. Objekte Klassifizieren. 
	\end{itemize}
	\section*{Aufgabe 4}
	\begin{tabular}[h]{c|c|c|c|c}
		Material Nr. & Verbrauch p.a. Stück & Preis Stück in € & Verbrauchswert & Rang \\
		200 & 300 & 25 & 7500 & 3 \\
		201 & 1000 & 7 & 7000 & 4 \\
		202 & 150 & 90 & 13500 & 2 \\
		203 & 2500 & 0,5 & 1250 & 5 \\
		204 & 500 & 150 & 75000 & 1
	\end{tabular} \\
	Berechnung Gesamtvolumen: \\
	g = 75000 + 13500 + 7500 + 7000 + 1250 \\
	g = 104250 €\\
	\begin{tabular}[h]{c|c|c|c}
		Material Nr. & Verbrauchswert in € & Anteil Gesamtvolumen in \% & Klasse \\
		200 & 7500 & 7,19 & B \\
		201 & 7000 & 6,71 & B \\
		202 & 13500 & 12,95 & B \\
		203 & 1250 & 1,2 & C \\
		204 & 75000 & 71,94 & A 
	\end{tabular}
	\section*{Aufgabe 5}
	Global Sourcing beschreibt den globalen Wareneinkauf um Waren möglichst günstig zu erhalten. \\
	Modular Sourcing bezeichnet den Einkauf von vorgefertigten Modulen um die Anzahl der Lieferketten und die eigene Fertigungstiefe so gering wie möglich zu halten. 
	\section*{Aufgabe 6}
	Das Supply Chain Management soll die Qualität eines Produkts bzw. einer Dienstleistung entlang einer Wertschöpfungskette über mehrere Betriebe hinweg sichern. Als Verbindungsstück dient der gemeinsam durchgeführte Herstellungsprozess.
	\section*{Aufgabe 7}
	Der operative Einkauf wird automatisiert, Stichwort Einkauf 4.0. Dieser muss lediglich durch den strategischen Einkauf überwacht und gesteuert werden. \\
	Kunden können Ihre Einkäufe individualisieren und müssen nicht mehr aus einer fertigen Produktpalette wählen. Bestes Beispiel hierfür sind bedruckte Geschenkartikel wie Tassen, Poster, T-Shirts etc.
	\section*{Aufgabe 8}
	\begin{tabular}[h]{c|c|c}
		Kennzahl & Amazon & Otto-Group \\
		Umsatzerlös & 280.522 Mio USD  & 14,3 Mrd € \\
		Ergebnis nach Steuern & 11.588 Mio USD & Im Material nicht gefunden \\
		Eigenkapital & 62.060 Mio USD & 1.452 Mio € \\
		Eigenkapitalquote & 42,51 & Im Material nicht gefunden \\
		Bilanzsumme & 225.248 Mio USD & 10.741 Mio €  \\
		Anzahl Mitarbeiter & 790.000 & 51.982
	\end{tabular} \\
	Der Bericht der Otto-Group auf den Seiten 20 - 29 besteht im Wesentlichen aus Eigenlob und zusammenhanglosen nicht eingeordneten Kennzahlen.
\end{document}