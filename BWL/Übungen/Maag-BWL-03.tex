\documentclass{article}
\begin{document}
	\section*{Lösungsvorschlag BWL 003 Maag}
	\subsection*{Aufgabe 1}
	Mit dem Unersättlichkeitsaxiom beschreibt man das unerschöpfliche Vorhandensein von Bedarfen, mit Kaufkraft ausgestattete Bedürfnisse. \\
	Das Knappheitsaxiom geht davon aus, dass Ressourcen für die Bedürfnisbefriedigung knapp sind.
	
	\subsection*{Aufgabe 2}
	Das ökonomische Prinzip sieht vor knappe Güter und beinahe endlose Bedürfnisse in ein zweckrationales Verhältnis zu setzen. Aus den vorhanden, knnappen, Ressourcen sollen möglichst viele Bedürfnisse befriedigt werden. Es gilt das optimale Verhältnis zwischen Input und Output zu finden. \\
	Ansätze sind hierbei das Minimal- und Maximalprinzip. \\
	Das Maximalprinzip sieht vor aus allen vorhanden Ressourcen das beste, maximale, Ergebnis zu erzielen. Der Input ist klar definiert das Ziel jedoch nicht. Beispielsweise kann der maximale Smartphone kauf darin enden aus den angesparten 1000 € das beste Smartphone zu generieren. \\
	
	\subsection*{Aufgabe 3}
	Das ökonomische Prinzip steht bei wirtschaftlichem Handeln mit dem ökologischen und dem sozialen Handeln in einem Zielkonflikt und muss mit diesen Größen austariert werden.
	
	\subsection*{Aufgabe 4}
	Produktivität beschreibt das Verhältnis zwischen eingesetzten Faktoren und dem produziertem Ergebnis. Diese Kennzahl lässt sich auch auf Dienstleistungen übertragen \\
	$P = \frac{Ausbringungsmenge}{Einsatzmenge} = \frac{Output}{Input}$ \\
	Wirtschaftlichkeit liegt vor, wenn der eingesetzte Aufwand den Ertrag nicht übersteigt. \\
	$W = Aufwand \leq Ertrag$ \\
	Um Wirtschaftlichkeit zu erreichen kann nach ökonomischen Prinzip gehandelt werden. \\
	Rentabilität setzt das eingesetzte Kapital mit dem erzielten Gewinn in Relation zu einander. In der einfachsten Form ergibt sich daraus: \\
	$R = \frac{Gewinn}{Kapital}$ \\
	Multipliziert man das Verhältnis mit 100 erhält man die Verzinsung des Kapitals durch den Gewinn in Prozent. Daraus ergibt sich z.B. die Eigenkapitalrentabilität der BWL: \\
	$ER = \frac{Gewinn}{Eigenkapital} * 100$ \\
	
	
	
	\subsection*{Aufgabe 5}
	Sustainable Business bezeichnet einen Betrieb, welcher vorlaufend und nachhaltig wirtschaftet damit kommende Generationen weiter damit wirtschaften können. Ein Betrieb ist nachhaltig wenn er ökonomisch, ökologisch und sozial handelt.
	
	\subsection*{Aufgabe 6}
	Nachhaltigkeit stützt sich auf Ökologie, Sozialwesen und Ökonomie. Sie ist Maßstab für Staaten und Unternehmen und soll generationenübergreifend deren Fortbestand sichern ohne bleibende Schäden für kommende Generationen zu hinterlassen. \\
	Ökologische Nachhaltigkeit sieht vor Umweltschäden zu unterlassen, nachwachsende Rohstoffe nur in der nachwachsenden Menge zu verbrauchen und von nicht nachwachsenden Rohstoffen so wenig wie möglich und so viel wie nötig zu nutzen. \\
	Soziale Nachhaltigkeit stellt den Mensch in den Mittelpunkt. keinem Menschen darf seine Würde die Möglichkeit der persönlichen Entfaltung abgesprochen werden. \\
	Abgerundet wird das Modell durch die ökonomische Nachhaltigkeit. Auch ein nachhaltiges Unternehmen muss Gewinn erwirtschaften um am Markt bestehen zu bleiben.
	
	\subsection*{Aufgabe 7}
	Betriebswirtschaftliche Funktionen sind Aufgaben, die in der Regel von allen Betrieben bewältigt werden müssen. Hierzu zählen, Rechnungswesen, Personalwirtschaft, Einkauf und Absatz sowie Produktion. \\
	Leistungswirtschaftliche Funktionen sind an der Erstellung der Leistung eines Betriebes beteiligt. Bei Industriebetrieben handelt es sich in der Regel um die Produktion, Absatz und Marketing.
	
	\subsection*{Aufgabe 8}
	Im Folgenden erläutere ich drei verschiedene Veränderungen die sich aus der Digitalisierung ergeben haben. \\
	Zum Einen sind Informationen zu Ressourcen geworden. Ob persönliche Daten, Unternehmenskennzahlen oder Adressdaten. Informationen können systematisch verarbeitet und ausgewertet werden. Beispielsweise das Kaufverhalten auf Amazon kann von Amazon ausgewertet werden um zielgruppenorientiert Werbung an Kunden zu schalten. \\
	Weiter hat die Digitalisierung Unternehmen deutlich agiler werden lassen. Allein die Tatsache, dass man durch digitales Arbeiten Home Office ermögliche kann, zeigt eine massive Erhöhung der Flexibilität digitalisierter Unternehmen. \\
	Ferne hat die Digitalisierung eine massive Automatisierung zur Folge gehabt. Daraus resultiert, dass einfache Jobs zunehmend entfallen während dispositive Ingenieursberufe deutlich gefragter werden. \\
	Darüber hinaus ist ganz besonders auffällig, dass computergestützte Arbeitsschritte wesentlich genauer sind als vorher. Präzisere Maschinen verursachen weniger Materialverschleiß und haben so eine höhere Effektivität.
\end{document}