\documentclass{article}
\usepackage{mathtools}
\begin{document}
	\section*{Lsg Vorschlag BWL 007 Maximilian Maag}
	\section*{Aufgabe 1}
	Deutschland verbucht einen CO$_{2}$-Austoß von 805 Millionen Tonnen im Jahr 2019. \\
	Bis 2030 müssen 263 Millionen Tonnen eingespart werden. \\
	Die Im Material vorgelegte Studie sieht durch eine beschleunigte Digitalisierung die Chance den CO$_{2}$-Ausstoß zu minimieren. \\
	Für den Erfolg dieser Transformation bedarf es seitens der Wirtschaft bessere Rahmenbedingungen. 87 Prozent der Befragten Teilnehmer fordern den Ausbau von erneuerbaren Energien und 46 Prozent mehr Investitionsanreize für Unternehmen. \\
	Meiner persönlichen Einschätzung nach wird bei der Mobilität zu wenig auf die Eisenbahn geschaut und nicht berücksichtig, dass diese Forderungen auf eine durch Lobbyismus vertaubte und erblindete Politik stoßen. Eine technisch mögliche und tatsächliche Umsetzung stellt einen gigantischen Unterschied dar.
	\section*{Aufgabe 2}
	Die Material beigefügte Studie sieht Einsparpotenzial in den Bereichen: Fertigung, Mobilität, Gebäude, Argar, Energie, Arbeit und Gesundheit. \\
	Das größte Einsparpotenzial wird mit 61 Millionen Tonnen in der industriellen Fertigung gesehen. \\
	Die Aussagen der Studie werden nicht durch Quellen belegt, das Paper liest sich nur wie eine Werbebroschüre. Darüber hinaus wird der Verbrauch von kostbarem trinkbarem Wasser nicht beleuchtet.
	\section*{Aufgabe 3}
	Das beigefügte Material sieht in der Digitalisierung ein größeres Sparpotenzial als deren Verursachter Ausstoß. Bei einer moderaten Digitalisierung ergibt sich ein Ausstoß von 16 MT CO$_{2}$ bzw. bei beschleunigter Digitalisierung 22 MT CO$_{2}$.\\
	Es ist jedoch unklar ob die Studie nur den Betrieb oder die Fertigung oder aber beides zusammen betrachtet. Hinzu kommt der Verbrauch seltener und wertvoller Rohstoffe für die Halbleiterentwicklung und Produktion. Ferner wird die Erhebungsmethode der Studie nicht exakt erläutert. Es wird lediglich auf eine andere Studie verwiesen an die man sich anlehne. \\
	Die Ergebnisse dieser Studie seien dahingestellt. Diese können stimmen oder auch nicht, ohne die Erhebungsmethode und Einsicht in die Rohdaten der Auswertung sind sie für mich jedoch nicht nachvollziehbar und damit nutzlos. 
	\section*{Aufgabe 4}
	Eine Lieferkette umfasst alle, aus sicht des Endkunden, beteiligten Organisationen die für die Erstellung eines an ihn gelieferten Produkts beteiligt sind.
	\section*{Aufgabe 5}
	Das Suply-Chain-Management beschäftigt sich mit der Qualitätssicherung innerhalb der Supply-Chain, also der Lieferkette. Es ist eine Aufgabe die durch Organisation und Absprache mit vor- bzw. nachgelagerten Organisation die Qualität des Endproduktes sichern soll. Diese dispositive Aufgabe fällt dem Management zu, da es sich um eine organisatorische bzw. planende Aufgabe handelt. Diese trägt wesentlich zur Ausrichtung eines Unternehmens bei. Die Lenkung der Ausrichtung ist grundsätzlich Aufgabe des Managements.
	\section*{Aufgabe 6}
	Das Lieferkettengesetz verfolgt das Ziel Unternehmen in Deutschland dazu zu verpflichten bestimmte Standards innerhalb aller Organisationen der eigenen Lieferkette einzuhalten. \\
	Insbesondere sollen Unternehmen Sozialstandards, Kaufverträge, Lohnstandards einhalten und transparent offenlegen. Das Lieferkettengesetz sah in seiner finalen Form hierfür mehrere nicht finanzielle Veröffentlichungspflichten vor.
	\section*{Aufgabe 7}
	Ein Lieferkettengesetz würde insbesondere die Beschaffung und Produktion betreffen. \\
	Roh- Hilfs- und Betriebsstoffe dürften dann nur in Betrieben mit bestimmten Standards eingekauft werden. Dies umzusetzen und zu dokumentieren ist im Zweifel Aufgabe des Einkäufers. Ferner dürfen diese Stoffe auch nicht verwendet werden. Darüber hinaus müssen neue Handelswege aufgebracht werden sofern die bestehenden Partner die Normen nicht einhalten können. \\
	Um Absatz zu erzielen müssen die Einhaltung der Standards dem Kunden offengelegt werden. Und werden dadurch zum Verkaufsargument gegenüber der Konkurrenz.
	\section*{Aufgabe 8}
	Beschaffung bezeichnet den Einkauf von Material. Die Materialwirtschaft umfasst alle Aspekte rund um Material insbesondere dessen Transport, Verarbeitung und Verkauf. Die Beschaffungswirtschaft ist somit Teil der Materialwirtschaft.
\end{document}