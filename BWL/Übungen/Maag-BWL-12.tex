\documentclass{article}

\begin{document}
	\section*{Lsg Vorschlag BWL Ü012 Maximilian Maag}
	\section*{Aufgabe 1}
	\begin{itemize}
		\item Cash-Flow, bezeichnet den Kassenüberschuss. Differenz zwischen Einzahlungen und Auszahlungen.
		\item Innenfinanzierung sieht Kapitalbereitstellung ohne externe Kapitalgeber vor. Zum Beispiel durch den eigenen Gewinn. Außenfinanzierung nimmt für die Kapitalbereitstellung externe Kapitalgeber in Anspruch. Klassischerweise handelt sich um einen Bankkredit.
		\item Kapitalbeteiligung bezeichnet den Besitz von Eigenkapitalanteilen an einem Unternehmen.
		\item Eine Bilanz weißt die Herkunft und die Verwendung des Kapitals eines Unternehmens aus.
		\item Aktiva, Aktivseite einer Bilanz. Sie enthält das Vermögen eines Unternehmens.
		\item Passiva, ist die passive Seite einer Bilanz und weißt die Herkunft des Kapitals eines Unternehmens auf.
		\item EK-Rendite gibt die Relation des Gewinns zum Eigenkapital an. 
	\end{itemize}
	\section*{Aufgabe 2}
	\subsection*{a)}
	Bankgeschäfte werden nur noch online abgewickelt.
	\subsection*{b)}
	Angebote sind besser vergleichbar und überprüfbar. Die Loyalität der Kunden gegenüber den großen Finanzinstituten nimmt ab.
	\subsection*{c)}
	Crowd Founding entsteht als neue Form der Finanzierung.
	\subsection*{d)}
	Kryptowährungen wie Bitcoin u.ä. treten in Erscheinung.
	\section*{Aufgabe 3}
	\subsection*{a)}
	EK-Rentabilität: $\frac{Gewinn}{Eigenkapital}$
	\subsection*{b)}
	GK-Rendite: $\frac{Gewinn+FK-Zinsen}{Eigenkapital + Fremdkapital}$
	\section*{Aufgabe 4}
	\subsection*{a)}
	EKR: $\frac{150000}{1000000}*100$ \\
	EKR: $15 \%$
	\subsection*{b)}
	GKR: $\frac{150000 + 25000}{1000000 + 500000} * 100$ \\
	GKR: $11,66 \%$
	\section*{Aufgabe 5}
	Im Folgenden erläutere ich drei Exemplare unterschiedlicher Finanzierungsmöglichkeiten. \\
	Es besteht die Möglichkeit ein Unternehmen durch Fremdkapital zu finanzieren. Im Vordergrund steht dabei oft der klassische Bankkredit. Als FK-Geber hat die Bank im Unternehmen keine Mitbestimmungsrechte. Das Fremdkapital ist jedoch zeitlich gebunden und es entfallen mitunter Zinslasten für das Unternehmen. \\
	Darüber hinaus besteht die Möglichkeit eine Finanzierung durch Eigenkapital zu stemmen. Dabei könnte eine AG zum Beispiel neue Aktien ausgeben. Besonders von Vorteil: Das Kapital ist zeitlich unbefristet verfügbar. Allerdings verschieben sich auf Seiten der Shareholder Machtverhältnisse bei der Ausgabe junger Aktien. \\
	Als Vertreter der Hybridfinanzmittel sollten die Wandelanleihen nicht unerwähnt bleiben. Dabei wird das Erhaltene Kapital durch Anteile am Unternehmen verzinst. (Beispiel G.S Wandelanleihe bei Facebook.)
	\section*{Aufgabe 6}
	\subsection*{a)}
	Horizontale Finanzierungsregeln fordern, dass langfristiges Kapital langfristig finanziert werden muss. Während kurzfristiges Kapital kurzfristig finanziert werden muss.
	\subsection*{b)}
	Vertikale Finanzierungsregeln machen Vorschriften zur Kapitalstruktur.
	\section*{Aufgabe 7}
	Eine Bilanz hat die Aufgabe die Vermögensverteilung und die Herkunft des Vermögens eines Unternehmens darzustellen. Neben der Bilanz gibt die GuV Aufschluss über fortlaufende Erträge und Aufwendungen.
	\section*{Aufgabe 8}
	Buchführung wird mit kaufmännischer Sorgfalt ausgeführt. Das heißt alle Geschäftsfälle müssen stattgefunden haben und werden nicht ohne Beleg verbucht. \\
	Eine ordentliche Buchhaltung ist fortlaufend und lückenlos. \\
\end{document}