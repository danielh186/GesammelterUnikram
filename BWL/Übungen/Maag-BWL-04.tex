\documentclass{article}
\begin{document}
	
	\section*{Lsg Vorschlag BWL Übung 04 Maximilian Maag}
	\section*{Aufgabe 1}
	Zu unterscheiden sind Personen- und Kapitalgesellschaften. Eine Kapitalgesellschaft besteht aus dem eingelegten kapital. Das Kapital stellt selbst eine juristische Person dar und wird von den Gesellschaftern vertreten. Gesellschafter müssen nur mit dem eingebrachten Kapital haften. Eine Personengesellschaft besteht immer aus mindestens einer natürlichen Person. Die Gesellschafter der Personengesellschaft stehen im Vordergrund und bilden die Gesellschaft. Gesellschafter dürfen Privatentnahmen aus der Betriebskasse tätigen, müssen allerdings mit ihrem Privatvermögen bei Zahlungsunfähigkeit haften. Umgangssprachlich sind Personengesellschaften daher auch als Haus- und Hofgesellschaften. Stakeholder sind Personen mit Interesse an einem Unternehmen, zum Beispiel Mitarbeiter, Kunden, Lieferanten.
	\section*{Aufgabe 2}
	Die VW AG beschreibt in ihrem Nachhaltigkeitsbericht ihre Ziele und Herangehensweisen. Im 3. Nachhaltigkeitsbericht 2019 formuliert die VW AG  das Ziel ihren Handlungsrahmen aktiv mitzugestalten und Risiken über die Dekaden hinweg zu bewerten. Darüber hinaus strebt die VW AG eine höhere Recyclingquote an. Darüber hinaus soll der $CO_{2}$ Verbrauch je Fahrzeug von 43 Tonnen aktuell bis 2025 auf 31,6 Tonnen sinken. Ferner plant die VW AG insbesondere mit ihren Stakeholdern besser zu kommunizieren.
	\section*{Aufgabe 3}
	Die Wertschöpfung der VW AG gliedert sich wie folgt: \\
	\begin{itemize}
		\item Forschung und Entwicklung
		\item Beschaffung
		\item Produktion
		\item Marketing und Vertrieb
		\item Aftersales und Finanzdienstleistungen
		\item Verwertung
	\end{itemize}

Forschung und Entwicklung nimmt ein hohes Investitionsvolumen ein um zukünftigen Herausforderungen wie E-Mobilität, Digitalisierung und Vernetzung zu begegnen. \\

Auffallend ist die Wertschöpfung durch Aftersales und Finanzdienstleistungen. Mit der VW Bank leistet sich der VW Konzern eine eigene Bank um günstige Konditionen der EZB direkt in Form günstiger Konsumerkredite direkt an ihre Kunden weitergeben zu können. Hierdurch umgeht die VW AG hohe kosten durch eine dritte Bank und kann neben dem eigentlichen Verkauf an den Kunden zusätzlich Zinsen durch dazugehörigen Kredit durch die eigene Bank erwirtschaften. \\

Im Bereich Verwertung verspricht die VW AG zukünftig darauf zu achten bereits bei der Entwicklung dafür zu sorgen, dass verwendete Materialien wiederverwertbar bzw. recyclebar sind.

\section*{Aufgabe 4}
Das Nachhaltigkeits-Management gliedert sich im VW Konzern um den neu gebildeten Konzernsteuerkreis hier kommen wichtige Vertreter der Konzernverwaltung, der Kernmarken und des Betriebsrates zusammen um das Nachhaltigkeits-Management zu organisieren.

\section*{Aufgabe 5}
Zu den Stakeholdern von VW zählen unter anderem Politik, Zivilgesellschaft, Investoren und Mitarbeiter sowie Interessenverbände. Lokale Stakeholder-Engagements führen lokale Interessen konzernweit zu einem Framework zusammen.

\section*{Aufgabe 6}
VW We soll konzernweit digitale Kompetenzen bündeln und eine einheitliche Softwareplattform für alle Marken des VW Konzerns.

\section*{Aufgabe 7}
Mit GoToZero will Volkswagen seine Umweltbelastung von der Rohstoffgewinnung bis zum Lebensende seiner Produkte möglichst minimal halten. \\
Hierzu betätigt sich der VW Konzern auf den Feldern Resssourcen, Luftqaulität, Umwelt und Klimawandel mit dem Ziel den CO$_{2}$-Ausstoß so gering wie möglich zu halten. Darüber hinaus soll der VW Konzern hinsichtlich seiner Zielsetzungen und deren Erfüllung transparenter werden.

\section*{Aufgabe 8}
Der VW Konzern hat im Jahr 2019 insgesamt 23,42 Mio MWh/a verbraucht davon entfallen 1.051 KWh elektrische Energie auf ein Fahrzeug der VW Flotte und insgesamt 2.010 KWh Energie je Fahrzeug der VW Flotte. \\
Der CO$_{2}$ verbrauch der VW Neuwagenflotte stellt sich wie folgt dar: \\
\begin{itemize}
	\item 2010 1.096
	\item 2018 720
	\item 2019 675
\end{itemize}
Auf Basis dieser Zahlen lässt sich eine Verringerung von 2010 bis 2018 um 34 Prozentpunkte feststellen. Dies entspricht dem selbst formulierten Zielen des VW Konzerns. 

\end{document}