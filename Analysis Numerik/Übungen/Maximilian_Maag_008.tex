\documentclass{article}
\begin{document}
	\section*{Lsg Vorschlag A+N Ü008 Maximlian Maag}
	\section*{Aufgabe A}
	\begin{itemize}
		\item stimmt nicht
		\item stimmt nicht
		\item stimmt
		\item stimmt nicht
		\item stimmt
	\end{itemize}
	\section*{Aufgabe B}
	$f(x) = x^3 - x -2$; $f'(x) = 3x^2 -1$ \\
	Newton'sche Näherung: \\
	$x_{n+1} = x_n - \frac{f(x_n)}{f'(x_n)}$ \\
	$x_{n+1} = x_n - \frac{x^3 - x - 2}{3x^2 - 1}$ \\ \\
	$x_1 = 1 - \frac{1^3 - 1 - 2}{3*1^2 - 1} = 2$ \\
	$x_2 = 2 - \frac{2^3 - 2 - 2}{3*2^2 - 1} = 1,636363636$ \\
	$x_3 = 1,521441465$; $x_4 = 1,52137971$; $x_5 = 1,521379707$
	\section*{Aufgabe 1}
	\subsection*{a)}
	Kostenkehre entspricht Wendepunkt. \\
	$K(x) = \frac{1}{3}x^3 - 5x^2 + 16x + 10$ \\
	$K'(x) =  x^2 - 10x + 16$ \\
	$K''(x) =  2x - 10$ \\
	$K''(x) = 0$ \\
	$0 =  2x - 10$ \\
	$10 =  2x$ \\
	$5 =  x$ \\
	Die Kostenkehre liegt bei 5 produzierten Einheiten.
	\subsection*{b)}
	Zeichnen muss aufgrund meiner Sehbehinderung leider entfallen. 
	\subsection*{c)}
	$g(x) = 10x - (\frac{1}{3}x^3 - 5x^2 + 26x +10)$ \\
	$g(x) = 10x - \frac{1}{3}x^3 + 5x^2 - 26x -10$ \\
	$g(x) =- \frac{1}{3}x^3 + 5x^2 - 16x -10$ \\ \\
	Ableitungen: \\
	$g'(x) = -x^2 + 10x - 16$ $g''(x) = -2x + 10$ \\
	\\
	Der maximale Gewinn ist eine lokales Maximum der Funktion g(x). \\
	Ansatz: $g'(x) = 0$ \\ \\
	$g'(x) = 0$ \\
	$-x^2 + 10x - 16 = 0$ \\
	$x^2 - 10x + 16 = 0$ \\
	$x_{1} = 5 + \sqrt{5^2 - 16}$ \\
	$x_{2} = 5 + \sqrt{5^2 - 16}$ \\
	$x_{1} = 8$ \\
	$x_{2} = 2$ \\
	\\
	Hinreichendes Kriterium für Hochpunkt prüfen: \\
	$g''(8) < 0$ Hochpunkt (Gewinn wird maximal) \\
	$g''(2) > 0$ Tiefpunkt, uninteressant \\
	\\
	$g(8) = 11,333...$ € \\
	\\
	Der Gewinn maximiert sich bei einer Ausbringungsmenge von 8 Stück mit 11,33€. 
	\section*{Aufgabe 2}
	Vorgabe: $x_0 = 1$
	\subsection*{a)}
	$f(x) = x^2 + 1 - \sqrt{x}$; $f'(x) = 2x + \frac{1}{2}x^{-\frac{1}{2}}$ \\ \\
	$x_{n+1} = x_n - \frac{x^2 + 1 - \sqrt{x}}{2x + \frac{1}{2}x^{-\frac{1}{2}}}$ \\ \\
	$x_1 = \frac{3}{5}$ \\
	$x_2 = 0,5270675309$ \\
	$x_3 = 0,5248904311$ \\
	$x_4 = 0,5248885987$ \\
	$x_5 = 0,5248885987$
	\subsection*{b)}
	$f(x) = 2^x - \frac{1}{x}$; $f'(x) = ln(2)*2^x - (-x^{-2})$ \\ \\
	$x_{n+1} = \frac{ 2^x - \frac{1}{x}}{ln(2)*2^x - (-x^{-2})}$ \\ \\
	$x_1 = 0,5809402158$; $x_2 = 0,6373230798$; $x_3 = 0,6411711379$; $x_4 = 0,6411857443$;$x_5 = 0,6411857445$; $x_6 = 0,6411857445$ 
	\section*{Aufgabe 3}
	\subsection*{a)}
	Die Steigung einer Sekante wird immer weiter korrigiert bis sie sich der gesuchten Nullstelle annähert. 
	\subsection*{b)}
	$f(x) = x^3 + x - 1$ \\
	$x_{n+2} = x_n -f(x_n) * \frac{x_{n+1} - x_n}{f(x_{n+1}) - f(x_n)}$ \\
	$x_0 = 0$ \\
	$x_1 = 1$ \\
	$x_2 = \frac{1}{2}$ \\
	$x_3 = 0,6363636364$ \\
	$x_4 = 0,6900523560$ \\
	$x_5 = 0,6820204196$ $|f(x_5)|<0,005$ \\
\end{document}