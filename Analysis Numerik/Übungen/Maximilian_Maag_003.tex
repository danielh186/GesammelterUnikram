\documentclass{article}
\usepackage{mathtools}
\begin{document}
	\section*{Lsg Vorschlag Ü003 A+N Maximilian Maag}
	\section*{Aufgabe A}
	Richtig, die Schranke stellt den Grenzwert dar und eine Folge mit Grenzwert heißt konvergent. \\
	Richtig, die Schranke stellt bei einer konvergenten Folge den Grenzwert dar.
	Falsch, gegen Beispiel alternierende Folge kann konvergieren. \\
	Richtig, Schranken zwei beschränkter Folgen addieren sich. \\
	Richtig, Folgen addieren sich. \\
	
	\section*{Aufgabe B}
	$a_{n} = n$:  Schranke nach unten, monoton, divergent: n wird immer größer. \\
	$a_{n} = (-1)^{n} * n$: nicht monoton durch $(-1)^{n}$, divergent wird für großer Zahlen immer größer, dem entsprechend keine Schranken und divergent. \\
	$a_{n} = \frac{(-1)^{n}}{n}$: $(-1)^{n}$ sorgt für Vorzeichenwechsel also nicht monoton. Teiler wird immer größer Folge konvergiert gegen 0, nach oben und unten beschränkt \\
	$a_{n} = 1 + \frac{1}{n}$: konvergiert gegen 1 durch $\frac{1}{n}$, Schranke nach oben und unten ($a_{n} \leq 1$), steigt monoton, weil kein Vorzeichenwechsel.
	\section*{Aufgabe C}
	\subsection*{a)}
	$\sum\limits_{n=1}^{n=5} \frac{1}{n} = 2,28333$
	\subsection*{b)}
	$\sum\limits_{n=1}^{n=5} n^{2} = 55$
	\subsection*{c)}
	$\sum\limits_{n=1}^{n=100} n = (100 + 1) * 50  = 5050$
	\section*{Aufgabe 1}
	\subsection*{a)}
	$\sqrt{5}$ mit Heron-Verfahren Startwert 5 \\
	$x_{0} = 1$; $y_{0} = \frac{5}{x_{0} = 5}$ \\
	$x_{1} = \frac{1}{2} * (1 + \frac{5}{1}) = 3$ \\
	$x_{2} = \frac{1}{2} * (3 + \frac{5}{3}) = 2,33333$ \\
	$x_{3} = \frac{1}{2}*(2,33333 + \frac{5}{2,33333}) = 2,2381$ \\
	$x_{4} = \frac{1}{2} * (2,2381 + \frac{5}{2,2381}) = 2,2360689$ \\
	$x_{5} = \frac{1}{2}*(2,2360689 + \frac{5}{2,2360689}) = 2,236067978$ \\
	$x_{6} = \frac{1}{2}*(2,236067978 + \frac{5}{2,236067978}) = 2,236067978$ \\
	$\sqrt{5} = 2,236067977$
	\subsection*{b)}
	$\sqrt{5}$; $x_{0} = 100$ \\
	$x_{1} = \frac{1}{2} * (100 + \frac{5}{100}) = 50,025$ \\
	$x_{2} = 25,062475012494$ \\
	$x_{3} = 12,630988229141$ \\
	$x_{4} = 6,5134200372811$ \\
	$x_{5} = 3,6405329543165$ \\
	$x_{6} = 2,5069791182389$ \\
	$x_{7} = 2,2507056834229$ \\
	$x_{8} = 2,2361155764454$ \\
	$x_{9} = 2,2360679780064$ \\
	$\sqrt{5} = 2,236067977$
	\subsection*{c)}
	$x_{0} = -1$; $\sqrt{5}$ \\
	$x_{1} = \frac{1}{2} * (-1 + \frac{5}{-1}) = -3$ \\
	$x_{2} = -2,3333333333333$ \\
	$x_{3} = -2,2380952380952$ \\
	$x_{4} = -2,2360688956434$ \\
	$x_{5} = -2,2360679775$ \\
	$x_{6} = -2,2360679774998$ \\
	$-\sqrt{5} = -2,2360679774998$ \\
	$\sqrt{5} = 2,236067977$
	\section*{Aufgabe 2}
	\subsection*{a)}
	$a_{0} = 1; a_{1} = 2, a_{2} = 4$ \\
	$q = \frac{a_{n+1}}{a_{n}}$ \\
	$q = \frac{4}{2} = 2$ \\
	$a_{n} = a_{0} * q^{n}$ \\
	$a_{n} = 1 * 2^{n}$ \\
	$\sum\limits_{n=0}^{n=64} 1 * 2^{n} = 2^{0} + 2^{1} + 2^{2} + 2^{3} + 2^{4} + 2^{5} + 2^{6} + 2^{7} + 2^{8} + 2^{9} + 2^{10} + 2^{11} + 2^{12} + 2^{13} + 2^{14} + 2^{15} + 2^{16} + 2^{17} + 2^{18} + 2^{19} + 2^{20} + 2^{21}  + 2^{22} + 2^{23} + 2^{24} + 2^{25} + 2^{26} + 2^{27} + 2^{28} + 2^{29} + 2^{30} + 2^{31}  + 2^{32} + 2^{33} + 2^{34} + 2^{35} + 2^{36} + 2^{37} + 2^{38} + 2^{39} + 2^{40} + 2^{41}  + 2^{42} + 2^{43} + 2^{44} + 2^{45} + 2^{46} + 2^{47} + 2^{48} + 2^{49} + 2^{50} + 2^{51}  + 2^{52} + 2^{53} + 2^{54} + 2^{55} + 2^{56} + 2^{57} + 2^{58} + 2^{59} + 2^{60} + 2^{61} + 2^{62} + 2^{63} + 2^{64}$ \\
	$= 3,6893488147419 * 10^{19}$ Reiskörner
	
	\subsection*{b)}
	$a_{0} = 40$ \\
	$a_{2} = 45$ \\
	$a_{n+1} = a_{n} + 5$ \\
	$a_{n} = 40 + 5n$ \\
	$a_{250} = 40 + 5 * 250 = 1290$ Sitze \\
	$\sum\limits_{n=0}^{n=250} 40 + 5n$ \\
	$(\sum\limits_{n=0}^{n=250})*5 + 40 = N$ \\
	$\frac{250*(250+1)}{2} * 5 + 40 = N$ \\
	$N = 156915$ Sitze im Theater.
	
	\section*{Aufgabe 3}
	Tenärzahl 22222 22222 22222 22222 in dezimal. \\
	$\sum\limits_{n=0}^{n=20} 2*3^{n} = 3486784400$

\end{document}