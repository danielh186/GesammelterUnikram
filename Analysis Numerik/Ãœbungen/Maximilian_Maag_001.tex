\documentclass{article}
\begin{document}
	\section*{Lösungsvorschlag Ü01 A+N Maximilian Maag}
	\subparagraph*{Aufgabe A}
	\begin{itemize}
		\item 1. Falsch
		\item 2. Richtig
		\item 3. Richtig
		\item 4. Richtig
		\item 5. Falsch
		\item 6. Richtig
	\end{itemize}
	
	\subparagraph*{Aufgabe B}
	\subparagraph*{a)}
	  
	$a_{n} = 2n -1$ \\
	$a_{0} = -1$ \\
	$a_{1} = 2 * 1 - 1 = 1 $ \\
	$a_{2} = 2 * 2 - = 3 $ \\
	$a_{3} = 2*3 - 1 = 5 $ \\
	$a_{4} = 2*4 - 1 = 7 $ \\
	$a_{5} = 2*5-1 = 9 $ 
	
	\subparagraph*{b)}
	$a_{n} = 1 + \frac{1}{n}$ \\
	$a_{1} = 1 + \frac{1}{1} = 2$ \\
	$a_{2} = 1 + \frac{1}{2} = \frac{3}{2}$ \\
	$a_{3} = 1 + \frac{1}{3} = \frac{4}{3}$ \\
	$a_{4} = 1 + \frac{1}{4} = \frac{5}{4}$ \\
	$a_{5} = 1 + \frac{1}{5} = \frac{6}{5}$ 
	\subparagraph*{c)}
	$a_{n} = (-1)^{n} * 2n$ \\
	$a_{0} = 0$ \\
	$a_{1} = (-1)^{1} * 2 = -2$  \\
	$a_{2} = (-1)^{2} * 2*2$ \\
	$= 1 * 4 = 4$ \\
	$a_{3} = (-1)^{3} * 2*3$ \\
	$= -1 * 6 = -6$ \\
	$a_{4} = (-1)^{4} * 2*4 = 8$ \\
	$a_{5} = (-1)^{5} * 2*5 = -10$ 

	\subparagraph*{Aufgabe C}
	
	\subparagraph*{a)}
	d = 4 \\
	$a_{n} = 3 + 4n$
	\subparagraph*{b)}
	$q = \frac{1}{4}$ \\
	$a_{n} = 4 * (\frac{1}{4})^{n}$
	
	\subparagraph*{c)}
	$a_{n} = a_{0} * q^{n}$ \\
	$q = \frac{-8}{2} = -4$ \\
	$a_{0} = \frac{1}{8}$ \\
	$a_{n} = \frac{1}{8}*(-4)^{n}$
	
	\subparagraph*{Aufgabe D}
	Es wird Näherungsweise eine Entfernung von 384.400 km angenommen. Eine Zeitung misst aufgeschlagen eine Dicke von grob geschätzten 0,1 mm. \\
	
	A = 0,1 , 0,2, 0,4, 0,8, 1,6 .... \\
	Diese geometrische Reihe beschreibt die Dicke der angenäherten Zeitung. \\
	$a_{n} = a_{0}  * q^{n}$ \\
	$q = \frac{1,6}{0,8} = 2$ \\
	$a_{n} = 0,1 * 2^{n}$ \\
	Um zum Mond zu gelangen muss die Dicke der Zeitung der Entfernung Erde - Mond entsprechen. Daraus ergibt sich folgende Exponentialgleichung \linebreak
	
	$384400 km = 384400000000 mm = 3,844 * 10^{12}$
	$0,1 *2^{x} = 3,844 * 10^{12}$ \\
	$2^{x} = 3,844 * 10^{13} |log()$ \\
	$log(2^{x}) = log(3,844 * 10^{13})$ \\
	$x * log(2) = log(3,844 * 10^{13})$ \\
	$x = \frac{log(3,844 * 10^{13}}{  log(2)} $ \\
	$x = 45,1277$ $\approx$ 46 da die 45,12-te Faltung überschritten werden muss. 
	
 Der Praxistest ist mangels einer Zeitung entfallen.
	Die Rechnung kann als Reihe dargestellt werden und zeigt ein hohes Wachstum für ein geringes n.

	\subparagraph*{Aufgabe 1}
	
	\subparagraph*{a)}
	
	Zinsen pro Jahr: $500 * \frac{3}{100} = 15 €$ \\
	$a_{n} = 500 + 15n$ \\
	$2000 = 500 + 15n$
	$1500 = 15n$
	$n = 100$ Jahre
	
	
	\subparagraph*{b)}
	$a_{n} = 500 * 1,03^{n}$ \\
	$2000 = 500 * 1,03^{n}$ \\
	$500 * 1,03^{n} = 2000$ \\
	$1,03^{n} = 4$ \\
	$\log(1,03^{n}) = \log(4)$ \\
	$n * \log(1,03) = \log(4)$ \\
	$n   = \frac{\log(4)}{\log(1,03)} $ \\
	$n = 46,8995 \approx 47$ Jahre
		
	\subparagraph*{Aufgabe 2}
	
	\subparagraph*{a)}
	
	$a_{3} = 25 a_{6} = 46$ \\
	$3d = 46 - 25$ \\
	$d = 7$ \\
	$a_{n} = a_0 + n * d$ \\
	$a_{0} = 25 - 3 * 7$ \\
	$a_{0} = 4$ \\
	$a_{n} = 4 + 7n$ \\
	
	\subparagraph*{b)}
	$a_{n} = 16 * 2,5^{n}$
	
	\subparagraph*{c)}
	$a_{2} = 2000$ $a_{4} = 1280$ \\
	$q^{2} = \frac{1280}{2000}$ \\
	$q = \sqrt{\frac{1280}{2000}}$ \\
	$q = 0,8$ \\
	$a_{2} * (\frac{8}{10})^{2} = a_{4} \\
	a_{4} * (\frac{10}{8})^{2} = a_{2} \\
	a_{2} * (\frac{10}{8})^{2} = a_{0} \\
	2000 * (\frac{10}{8})^{2} = 3125 \\$ \\
	$a_{n} = 3125 * (8/10)^{n}$ \\
	
	\subparagraph*{d)}
	$a_{0} = 3$ \\
	$a_{n + 1} = 2a_{n} + 1$ \\
	$a_{1} = 2 * 3 + 1$ \\
	$a_{1} = 7$ \\
	$a_{0} * q = a_{1}$ \\
	$3 * q = 7$ \\
	$q = \frac{7}{3}$ \\
	$a_{n} = a_{0} * q^{n}$ \\
	$a_{n} = 3 * (\frac{7}{3})^{n}$ \\
	
	\subparagraph*{Aufgabe 3 - Zufallszahlen}
	$x_{n} = (ax_{n-1} + c)$ mod $m$ \\
	$a = 7$; $c = 4$; $m = 9$ \\
	$x_{n} = (7*x_{n-1} + 4)$ mod $9$ \\
	$x_{0}  = 3$
	$x_{1} = (7*3+ 4)$ mod $9$ = $7$ \\
	$x_{1} = (7*7 +4)$ mod $9$ = 8 \\
	$x_{2} = 6$\\
	$x_{3} = 1 $ \\
	$x_{4} = 2$ \\
	$x_{5} = 0$ \\
	$x_{6} = 4$ \\
	$x_{7} = 5$ \\
	$x_{8} = 3$ \\

	
	
				
\end{document}