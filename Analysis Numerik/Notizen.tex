\documentclass{article}

\usepackage{mathtools}
\usepackage[
colorlinks=true,
urlcolor=blue,
linkcolor=green
]{hyperref}
\begin{document}
	
	\section*{Notizen Analysis und Numerik}
	\section*{17.11.2020}
	\subsection*{Folgen}
	explizit und rekursiv.
	arithmetische und geometrische Folgen.
	explizit $a_n$ kann direkt berechnet werden. Rekursiv Rückbezüglich Folgeglied wird aus vorangegenen Glied berechnet. \linebreak
	
	Rekursive Funktionen können sich selbst aufrufen. Brauchen einen Startwert und eine Abbruchbedingung. \linebreak
	
	wechsel zwischen Darstellungen von Folgen. explizit: $a_n = 3 * 7^n$ rekursiv $a_n = 7 * a_{n-1}$. Bei anderen Folgen schwieriger. Bei arithmetisch und geometrisch einfach. Es gibt Folgen bei dem Wechsel nicht möglich ist. \linebreak
	
	Fakultät wird rekursiv definiert. Akkermannfunktion $A(m,n)$ Testfunktion für Computerleistung. Ackermannfunktion und n! kann nicht expliziet angegeben werden. \linebreak
	
	\subsection*{Grenzwerte}
	Folgen bieten eine tolle Möglichkeit Grenzwerte anzunähern. Kochsche Schneeflocke. Umfang wächst gegen unendlich. Fläche nähert sich einem Grenzwert. Genauer Grenzwert für Fläche A = $\frac{2}{5} * \sqrt{3}$ \\
	Eine reele Zahl g heißt Grenzwert der Folge ($a_n$), wenn für jede Zahl $\epsilon$ $>$ 0 die Ungleichung $|a_n -g| < \epsilon$ von einer gewissen Indexzahl N an erfüllt ist. Abstand von $a_n$ zu g muss also kleiner als $\epsilon$ werden. \\
	\section*{24.11.2020}
	Koch'sche Schneefolge mit Logo programmiert. (Nachreichung).  \\
	Umfang konvergiert gegen $\infty$ die Fläche hingegen einen bestimmten Grenzwert. \\
	Der Grenzwert kann erreicht werden. Beispiel $a_{n} = n - n; \epsilon = 0$
	\subsection*{Grenzwertsätze}
	Suche Grenzwerte der einzelnen Bestandteile.
	$\lim\limits_{n \to \infty} = \lim(n + m) = \lim(m) + \lim(n)$ \\
	$\lim(\frac{n}{m})$ n in der höchsten Potenz ausklammern \\
	$\lim\limits_{n \to \infty}$ $= \lim(n * m) = \lim(n) * \lim(n)$
	\subsection*{Nullfolge}
	Nullfolge ist eine Folge die gegen $0$ konvergiert. \\
	Altanierende Folgen. \\
	\subsection*{}
	Folgen ohne Grenzwert divergent mit Grenzwert konvergent \\
	Eine Folge kann nur einen Grenzwert haben!!!!!! \\
	Deshalb besonders vorsichtig bei alternierenden Folgen. \\
	Def. eine Folge heißt nach oben beschränkt, wenn es eine Zahl $k$ gibt, sodass für alle Folgenglieder gilt $a_{n} \leq k$ heißt sie nach oben beschränkt. \\ 
	Analog ist sie nach unten beschränkt wenn gilt $a_{n} \geq k$. \\
	 Ist eine Folge nach unten und oben beschränkt heißt sie beschränkt. \\
	Ein Punkt darf die Schranke berühren, da gilt $\leq k$ und $\geq k$. \\
	Def. Eine Folge wächst monoton, wenn gilt $a_{n+1} \geq a_{n}$. \\
	Anlaog gilt streng monoton steigend: $a_{n+1} > a_{n}$ \\
	Analog dazu gilt monoton fallend und streng monoton fallend mit $\leq a_{n}$ und $< a_{n}$ \\
	Eine monotone beschränkte Folge hat immer einen Grenzwert. \\
	Grenzwertgleichung $\lim(x_{n}) = \lim(x_{n-1})$ \\
	Wurzel kann durch Folge angenähert werden. \\
	\subsection*{Heron-Verfahren} 
	Für Annäherung beliebiger Wurzel.
	Quadrat mit a hat die Seitenlänge $\sqrt{a}$.
	Beginn mit Rechteck 1 * a mit quadriert. \\
	$x_{n} = \frac{1}{2}*(x_{n-1} + \frac{a}{x_{n-1}}))$ mit $x_{0} = 1$ \\
	Quadratische Konvergenz. \\
	Diese Rechnung wird durchgeführt bis das Ergebnis die gewünschte Näherung erreicht hat. \\
	Jede reelle zahl kann durch eine rationale Zahlen angenähert werden. Z.B. e oder $\pi$.
	
	\paragraph*{Reihen}
	Addition vieler reeller zahlen. \\
	$\sum\limits_{i=0}^{n} 2n$
	\subsection*{}
	Folge partialsummen, geometrische Summe \\
	Geometrische Summenfolge $s_{n} = \frac{1 - q^{n +1}}{1 - q}$ \\
	Def unendliche Reihe: $\sum\limits_{k=1}^{\infty}a_{k}$
	\section*{01.12.2020}
	Partialsummen sind Folgen bei denen alle Folgeglieder aufsummiert werden. \\
	geometrische Summe $\sum\limits_{k=0}^{n} q^{n}$
	hat einen Grenzwert. \\
	Reihen konvergieren, wenn die folge der Partialsummen der Reihe konvergiert. \\
	Geometrische Reihe konvergiert für q zwischen 1 und -1. \\
	Grenzwert für geometrische Folge wenn -1$\le$q$\le$1 = $\frac{1}{1 - q}$ \\
	 Perioden können als geometrische Folge geschrieben werden und der Grenzwert ist dann der dazugehörige Bruch.
	 Bsp: 0,999999999 ..... = 0,9 + 0,09 + 0,009 ... \\
	 0,9 * (0,1 + 0,01 + 0,001) \\
	 weitere Kriterien für Konvergenz: \\
	 notwendig: Eine Reihe muss hinter dem Sigma eine Nullfolge haben. $\sum\limits_{k=0}^{\infty} a_{k}$ muss $a_{k}$ eine Nullfolge sein. Wenn $a_{k}$ eine Nullfolge ist muss die zugehörige Reihe nicht konvergieren. \\
	 harmonische Reihe divergiert $\sum\limits_{k=0}^{\infty} \frac{1}{k}$
	 harmonische Reihe enthält Nullfolge aber divergiert. \\
	  Leibnitz-Kriterium. \\
	  $\sum\limits_{k=0}^{\infty} (-1)^{k} * a_{n}$
	  Wenn $a_{n}$ eine Nullfolge ist konvergiert die Reihe. \\
	  Majorantenkriterium, Qoutientenkriterium, Wurzelkriterium. \\
	  Grenzwert von Funktionen \\
	  \section*{08.12.2020}
	  Kapitel 2 Differenzialrechnung \\
	  Funktionen werden benötigt. \\
	  Grenzwert von Funktionen. \\
	  Def über Grenzwerte von Folgen. \\
	  setze allgemeine Testfolge ein. \\
	  Testfolge konvergiert gegen $\infty$ \\
	  Es entsteht eine Folge für die ein Grenzwert bestimmt werden kann mit Grenzwertsätzen. \\
	  Def. die Folge f hat für x  $\to$ $\infty$ den Grenzwert g wen gilt: ---- \\
	  Definitionslücke Annähern \\
	  Funktionen haben zusätzlich einen Grenzwert gegen eine Definitionslücke. \\
	  uneigentlicher Grenzwert $\infty$ \\
	  Funktionen mit Sprungstelle haben an der Sprungstelle keinen Grenzwert. \\
	 Stetigkeit Eine Funktion ist stätig wenn der Graph der Funktion gezeichnet werden kann ohne den Stift abzusetzen. \\
	 Def. eine Funktion heist stätig an einer Stelle $x_{0}$ wenn folgende Bedingungen gelten. \\
	 \begin{itemize}
	 	\item f ist an $x_{0}$  definiert
	 	\item Grenzwert an der stelle $x_{0}$ und
	 	\item Der Grenzwert ist gleich dem Funktionswert an der Stelle $x_{0}$.
	 \end{itemize}
	Definition gilt für die Stelle x. Ist f im gesamten Definitionsbereich stetig ist die Funktion f stetig. \\
	Bisektionsverfahren. \\
	f muss stetig sein. \\
	Im Intervall (a, b) müssen alle Funktionswerte zwischen f(a) und f(b) liegen. \\
	Nehme zwei Startwerte mit unterschiedlichem Vorzeichen. Teile das Intervall in der Mitte.
	Wenn f(a) und f(m) unterschiedliche Vorzeichen haben such in der linken Hälfte nach Nullstellen. Sonst suche in der rechten Hälfte. \\
	\section*{15.12.2020}
	Herleitung Ableitung \\
	Differentialkoeffizient \\
	Limes h  gegen 0. \\
	Def Grenzwert $\frac{f(x_{0} + h) - f(x_{0})}{h}$ \\
	Ableitungsregeln für Polynomfunktionen \\
	Produktregel: U * V' + U' * V
	Kettenregel: innere * äußere Ableitung
	Summanten werden einzelnen abgeleitet
	\section*{05.01.2021}
	Beispiel Kettenregel $f(x) = (3x + 7)^{100}$ $f'(x) = 3 * 100 * (3x+7)^{99}$
	Quotientenregel
	$(\frac{u}{v})' = \frac{u' * v - u * v'}{v^2}$ \\
	$tan(x) = \frac{sin(x)}{cos(x)}$ \\
	trigonometrischer Pythagoras \\
	$tan'(x) = \frac{1}{cos^2(x)}$ oder $tan'(x) = 1 + tan^2(x)$ \\
	Ableitung Exponentialfunktion
	am besten mit e. \\
	$(e^x)' = e^x$ \\
	$a^x = e^{ln(a^x)}$ \\
	$(a^x)' = ln(a) * a^x$ \\
	Ableitung Umkehrfunktion \\
	$(f^{-1})'(x) = \frac{1}{f'(f^{-1}(x))}$ \\
	$(log_a(x))' = \frac{1}{ln(a) * x}$ \\
	De L'Hospitalregel
	\section*{12.01.2021}
	Funktionsuntersuchungen \\
	Monotonie
	für f'(x) $>$ 0 streng monoton steigend. \\
	für f'(x) $<$ 0 streng monoton fallend. \\
	für f'(x) $\geq$ 0 monoton steigend. \\
	für f'(x) $\leq$ 0 monotn fallend. \\
	Extrema, Wendepunkte, Krümmung \\
	Rechts- in Linkskrümmung LRWP. \\
	Rechts- in Linkskrümmung RLWP. \\
	Wendepunkt Nullstelle der zweiten Ableitung. \\
	Extrema Nullstellen der ersten Ableitung. \\
	Rechtskrümmung f' fällt streng monoton. \\
	Linkskrümmung f' steigt streng monoton. \\
	Globales Maximum auf einem Intervall der größte Wert. \\
	Lokales Maximum größter Wert in einer Umgebung. \\
	Lokales Minimum analog. \\
	Globales Minimum kleinster Wert im Intervall. \\
	An Extremstellen ist die Steigung = 0. \\
	Krümmung entscheidet über Hoch- oder Tiefpunkt. \\
	Rechtskrümmung = Tiefpunkt. \\
	Linkskrümmung = Hochpunkt. \\
	Waagerechte Tangente und Krümmungsreich = Sattelpunkt. \\
	Eine Playlist von Simple Club zu diesem Thema: \url{https://www.youtube.com/watch?v=4L9s2GHZCq0&list=PLjaA00udJtOohSQL16R4UAQioBsqq_4n-} \\
	Wendestelle ist Extremstelle der Ableitung f'(x). \\
	Bei Funktionen wie $x^4$ versagt hinreichendes Kriterium. \\
	Deshalb allgemeines Kriterium: \\
	Ableiten bis Ableitung ungleich Null. Dann gilt: \\
	Es entscheidet wie oft abgeleitet wird. \\
	für gerade n: Minimum oder Maximum.\\
	Wenn n ungerade Sattelpunkt. \\
	Anwendung der Ableitung Gleichungen lösen. \\
	Transzendente Gleichungen Lösen näherungsweise mit Nullstellen. \\
	Gleichung in Nullstellenproblem umwandeln. \\
	Newton-Verfahren. \\
	Tangente anlegen. \\
	Nullstelle der Tangente bestimmen. \\
	An dieser Nullstelle Tangente anlegen. \\
	Nullstellen der Tangenten nähern sich der Nullstelle der approximierten Funktion numerisch an. \\
	Einfache Erklärung von Simple Club: \url{https://www.youtube.com/watch?v=xGemDmrCqEk} \\
	Newton's Näherungsformel: \\
	$x_{n+1} = x_n - \frac{f(x_n)}{f'(x_n)}$ \\
	Newten-Verfahren liefert nur eine Nullstelle. \\
	Für mehrere Nullstellen mehrere Startwerte verwenden. \\
	\section*{19.01.2021}
	Kapitel 3 Approximation und Interpolation \\
	Approximation durch Polynome \\
	Taylorreihen Potenzreihen Konvergenzradius \\
	Eine Funktion f(x) wird an der stelle $x_0$ und dessen Umgebung durch Taylorrpolynom approximiert \\
	Ableitungen und Funktionswert f(x) und Taylorrpolynom g(x) müssen an $x_0$ übereinstimmen. \\
	Je mehr Ableitungen übereinstimmen desto genauer die Approximation.
	Taylorrpolynom bestimmen durch Bedingungen der Taylorrgleichungen. \\
	1. Grad der Genauigkeit festlegen. \\
	2. Gleichungen für Bedingungen aufstellen. \\
	Beispiel f(0) = g(0) f'(0) = g'(0) usw. \\
	3. Aus sich ergebenden LGS alle Parameter bestimmen. \\
	4. alle Parameter in Ursprungsform einsetzen um Taylorpolynom zu erhalten. \\
	allgemeine Formel Taylorpolynom: \\
	 $f(x) = \frac{f^{(n)}(x)}{n!} * x^n + \frac{f^{(n-1)}(x)}{(n-1)!}*x^{n-1}$..... $ + \frac{f(0)}{0!}*x^0$ \\
	 Sinus kann durch Polynom nie genau dargestellt werden. \\
	 für n $\to \infty$ wird aus Taylorpolynom eine Taylorreihe. \\
	 $\sum\limits_{k=0}^{\infty} \frac{f^{n}(x_0)}{x!}(x - x_0)^k$ \\
	 Taylorreihe ist ein Beispiel für Potenzreihen \\
	 Jede Potenzreihe hat einen um die Entwicklungsstelle symmetrischen Bereich für den sie konvergiert. \\
	 Interpolation: gesucht ist ein Stützpolynom das bestimmte oder gemessene Punkte verbindet. \\
	 Malen nach Zahlen. \\
	 für x Werte gibt es ein Polynom mit dem Grad $\leq$ n. \\
	 1. Methode LGS. \\
	 2. Newton-Ansatz \\
	 \section*{22.01.2021}
	 Vollständige Indunktion \\
	 bewiesen werden Aussagen die von natürlichen Zahlen n abhängen. \\
	 Aussage hängt von n ab A(n). Zu zeigen ist, dass die Aussage A(n) für ein Startwert gilt (Indunktionsbasis) . \\
	 zusätzlich wird gezeigt, dass A(n+1) gilt.(Induktionsschritt) \\
	 \\
	 nehme an A(n) gilt und notiere A(n+1) ordentlich. Induktionsvoraussetzung anwenden und umformen.
	 \section*{26.01.2021}
	 Kapitel 4 Mehrdimensionale Funktionen \\ \\
	 Darstellung, Netzlinien, Niveaulinien \\
	 Tangential ebene Extrema, Gradienten Hesse-Matrix \\
	 Darstellung f(x,y). Darstellung als Fläche über x-y-Ebene. \\
	 Paraboloid $f(x,y) = x^2 + y^2$ \\
	 Netzlinien: für x Linien y festsetzen. für x Werte einsetzen. \\
	 wiederhole für andere y-Werte. \\
	 anschließend x festsetzen und y bestimmen für verschiedene x Werte. \\
	 ergibt Schrägdarstellung der Funktion \\
	 \\
	 Höhenlinien \\
	 horizontale Schnitte auf verschiedenen Ebenen. \\
	 Für Niveaulinie setze Term = Radius \\
	 Beispiel Niveau $1 = x^2 + y^2$ \\
	 Grundriss, Aufriss, Sattelfläche \\
	 Niveaulinien sehen von Oben wie Hyperbeln aus. \\
	 \\
	 Ableitung Mehrdimensionaler Funktionen \\
	 partielle Ableitungen sind Ableitungen in bestimmte Richtung \\
	 setze eine Variable fest und leite normal ab. \\
	 betrachte beim ableiten alle anderen nicht abgeleiteten Variablen als Zahl!!! \\
	 $f(x,y) = x^2 + y^2$ $f_x(x) = 2x$ \\
	 Tangentialebene als Ableitung einer Mehrdimensionalen Funktion. \\
	 Gleichung Tangentialebene: $f(x_0,y_0) + f_x(x,y) * (x-x_0) + f_y(x,y) * (y - y_0)$ \\
	 Gradient ist Vektor der Partielen Ableitungen der Mehrdimensionalen Funktion \\
	 Gradient zeigt in die steillste Richtung und stehtt senkrecht auf den Höhenlinien. \\
	 \section*{02.02.2021}
	 Steigung einer mehrdimensionalen Funktion.
	 normierter Vektor. \\
	 df nach dr. \\
	 grad(f) * r = f * cos(a) + f * sin(a) \\
	 Bilde Ableitung nach x und nach y \\
	 Richtungsableitung maximal wenn r in Richtung von Gradient zeigt. \\
	 Gradient steht senkrecht auf Höhenlinien. \\
	 Gradient zeigt in die Richtung des steilsten Anstiegs. \\
	 Gradientenfeld \\
	 $grad(f) = \vec{f_x, f_y}$ und Punkt einsetzen für Steigung an diesem Punkt. \\
	 Gradient kann auch angewendet werden auf mehrfach abgeleitete Funktionen angewendet werden. \\
	 $f_{xy} \equiv f_{yx}$ \\
	 Funktion kann zuerst nach x und dann nach y abgeleitet werden oder umgekehrt. \\
	 Extrema \\
	 $f_x$ und $f_y$ müssen = 0 sein für notwendiges Kriterium Extrema. \\
	 Daraus Folgt Gradient muss Nullvektor sein. \\
	 Für hinreichendes Kriterium gibt es vier zweite Ableitungen. Hesse Matrix \\
	 Determinante von Hesse Matrix nutzen \\
	 Wenn grad(f) = 0 dann ist Delta hinchreichend \\
	 $\delta$ und $f_xx$ $>$ 0 isoliertes Minimum für $f_xx$$<$ 0 isoliertes Maximium. \\
	 $\delta$ $<$ 0 Sattelpunkt. \\
	 $\delta = 0$ keine Aussage. \\
	 Gradientenabstiegsverfahren. \\
	 Gesucht Minimum. \\
	 Wähle Startpunkt \\
	 Bestimmte Steigung \\
	 Wenn positiv suche links davon und negativ suche rechts. \\
	 Schrittweite proportional zur Steigung
	 Steigung wird immer kleiner, wenn klein Genug Abbruch. \\
	 $x_{n+1} = x_n -s * f'(x_n)$ \\
	 \section*{09.02.2021}
	 Gradientabstieg mehrdimensional: \\
	 Startstelle wählen \\
	 negativen Gradienten bestimmen \\
	 Für den nächsten Schritt ergibt sich die Folge: \\
	 $\vec{x}_{n+1} = \vec{x}_n - s * grad(f)$ \\
	 evtl. Schritweite s anpassen. Abbruchbedingung festlegen zum Beispiel Betrag grad(f) < 0,01. \\
	 \\
	 Kapitel 5 - Integralrechnung \\
	 Obersumme Untersumme, Trapezverfahren, Keplersche Fassregel, Simpson-Verfahren \\
	 Obersumme, Untersumme \\
	 Funktion in Rechtecke unterteilen. \\
	 kleinste Funktionswerte geben Untersumme, größte Obersumme \\
	 Obersumme ein wenig zu groß Untersumme ein wenig zu klein. \\
	 Def. Bestimmtes Integral ist der Grenzwert der Ober- und Untersumme \\
	 $\int_{a}^{b}f(x)dx$ \\
	 Integration ist Umkehrung von Ableitung \\
	 Def. Stammfunktion: Eine Stammfunktion F(x) ist Stammfunktion von f(x) wenn F'(x) = f(x) \\
	 Def. Menge aller Stammfunktionen von f ist ein unbestimmtes Integral. \\
	 $\int_{a}^{b}f(x)dx = F(x) + c$ \\
	 bestimmtes Integral: $\int_{a}^{b}f(x)dx = F(a) - F(b)$ \\
	 Diese Formel liefert Flächenbilanz. Also Darauf achten ob sich im Intervall a - b eine Nullstelle befindet!!! \\
	 Simpson-Regel \\
	 Wenn Funktion nicht integrierbar, kann Integral angenähert werden. \\
	 Annäherung durch Trapeze. \\
	 Bestimmte Trapezfläche für n Streifen und addiere sie. \\
	 allgemein $\int_{a}^{b}f(x)dx = \frac{1}{2} * \frac{b - a}{2} * (y_0 + 2*y_1 + 2y_3.... +y_n)$ \\
	 linker und rechter Rand nur einmal. \\
	 Kepler'sche Fassregel \\
	 Approximation durch Parabel. Anschließend Integral der Parabel bestimmen. \\
	 Allgemein: $\int_{a}^{b}f(x)dx = \frac{b - a}{6} * (f(a) + 4*f(m)+f(b))$ \\
	 Simpson-Verfahren \\
	 Unterteilung der Funktion in mehrere Streifen und Anwendung der Kepler'sche Fassregel an. \\
	 
	 
\end{document}