\documentclass{article}
\begin{document}
	\section*{Lsg Vorschlag DB Ü01 Maximilian Maag}
	AIdual R+V Allgemeine Versicherung AG, Matrikelnummer 1246281
	\subsection*{Aufgabe 1}
	Relationale Datenbanken basieren auf Mengenlehre und Relationen. Sie besitzen einfache Datentypen und sind bei großen Zugriffszahlen vergleichsweise langsam. \\
	Objektrelationale Datenbanken besitzen komplexere Datentypen, z.B. ganze Videos, und sind eine typische Auskoppelung des Objektorientierungshypes der 90er. \\
	XML-Datenbanken sind darauf ausgelegt große mengen an Daten für den lesenden Zugriff bereitzustellen. Nach dem Grundgedanken des Internets ist dieser Datenbanktyp darauf ausgelegt vielen Nutzen möglichst schnell viele Daten zu präsentieren. Die Datentypen sind hierbei eher primitiv.
	\subsection*{Aufgabe 2}
	Zentrale Datenbanken: \\
	Pro: \\
	Einfache und schnelle Realisierung, günstige Wartung \\
	Cons: \\
	Höhere Anfälligkeit für Fehler, höhere Zugriffszeiten für weit entfernte Nutzer. \\ \\
	Dezentrale Datenbanken: \\
	Pro:
	Höhere Performance, geringere Fehleranfälligkeit \\
	Con: \\
	Höherer Wartungsaufwand, inkonsistente Performance sofern nicht optimiert
	\subsection*{Aufgabe 3}
	Unterschiede: \\
	Operationale Datenbanken verwenden aktuelle Produktionsdaten. \\
	Das Datawarehouse ist für die Speicherung historischer Daten verantwortlich. \\
	Operationale Datenbanken verzeichnen Zugriffe auf wenige Datenmengen. \\
	Das Datawarehouse verzeichnet regelmäßig Zugriffe auf große Datenmengen. \\
	Operationale Datenbanken sammeln Dateneingaben. \\
	Das Datawarehouse speichert Daten dient der Informationsgewinnung. \\
	Operationale Datenbanken sind auf hohe I/O Peformence ausgelegt und können somit viele gleichzeitige Zugriffe verarbeiten. \\
	Datawarehouses weisen eine geringe I/O Peformance auf. \\ \\
	Einsatzszenarien: \\
	Versicherungen verwenden Datawarehouses um aus historischen Daten Erkenntnisse über ihre Klienten zu gewinnen, damit das Versöhnungsangebot angepasst werden kann. \\
	Im Bereich Machinelearning sind große Datenmengen aus Datawarehouses nützlich. \\
	Eine operationale Datenbank könnte beispielsweise eine Webanwendung stützen. In der Regel zum Beispiel einen Onlineshop, der Kundendaten und Bestelldaten speichern und verarbeiten muss. \\
	Eine Hochschule könnte eine Datenbankanwendung nutzen um Mitteilungen über Noten zu digitalisieren.
	\subsection*{Aufgabe 4}
	\begin{itemize}
		\item BaseX
		\item SQLBase
		\item SAP Hana
		\item Access
		\item MariaDB
		\item MonetDB
		\item SAP MaxDB
		\item Redis
		\item UDS
		\item Zope Object Database
	\end{itemize}
	\subsection*{Aufgabe 5}
	Sammeln von Einkaufszettel. \\
	Nach einem Jahr werden die Einkaufszettel ausgewertet um durchschnittliche Ausgaben für Lebensmittel und die häufigste Einkaufszeit zu ermitteln.
\end{document}