\documentclass{article}
\usepackage{mathtools}
\begin{document}
	\section*{Lsg Vorschlag DB Ü02 Maximilian Maag}
	\section*{Aufgabe 1}
	\subsection*{a)}
	B S P 
	\subsection*{b)}
	R(B,S,P)\{(3,6,9),(1,5,9)\}
	\section*{Aufgabe 2}
	\subsection*{a)}
	R $\cup$ S = \{(1,2,3),(1,2,3),(3,5,7),(8,8,9)\}
	\subsection*{b)}
	R $\cap$ S = \{(1,2,3),(1,2,3)\}
	\subsection*{c)}
	R(a,b,c) = \{(3,5,7)\}
	\subsection*{d)}
	R(a,b,c) = \{(8,8,9)\}
	\section*{Aufgabe 3}
	\subsection*{a)}
	max: m+n; min: m+n
	\subsection*{b)}
	max: min(m.n); min: 0
	\subsection*{c)}
	max: m; min: 0
	\subsection*{d)}
	max: n; min: 0
	\section*{Aufgabe 4}
	\subsection*{a)}
	R $\cup$ S = \{(1,1),(1,1),(2,3),(3,2),(2,2)\}
	\subsection*{b)}
	R $\cap$ T = \{(1,1)\}
	\subsection*{c)}
	T = \{(1,1)(3,2)\}
	\subsection*{d)}
	R $\cup$ R = \{(1,1),(1,1),(2,3),(1,1),(1,1),(2,3)\}
	\section*{Aufgabe 5}
	R = 
	\begin{tabular}{c|c|c|c}
		a&b&c&d \\
		\hline
		1&3&2&4 \\
		\hline
		5&7&8&6 \\
		\hline
		9&11&12&10 \\
		\hline
	\end{tabular}
	\subsection*{a)}
	$\pi_{a,b}$ =
	\begin{tabular}{c|c}
		a&b \\
		\hline
		1&3 \\
		\hline
		5&7 \\
		\hline
		9&11 \\
		\hline
	\end{tabular} 
	\subsection*{b)}
	$\pi_{d}$ =
	\begin{tabular}{|c}
		d \\
		\hline
		4 \\
		\hline
		6 \\
		\hline
		10 \\
		\hline
	\end{tabular}
	\subsection*{c)}
	$\pi_{b,d}$ = 
	\begin{tabular}{c|c}
		b&d \\
		\hline
		3&4 \\
		\hline
		7&6 \\
		\hline
		11&10 \\
		\hline
	\end{tabular}
	\subsection*{d)}
	$\pi_{a,b,c,d}$ = 
	\begin{tabular}{c|c|c|c}
		a&b&c&d \\
		\hline
		1&3&2&4 \\
		\hline
		5&7&8&6 \\
		\hline
		9&11&12&10 \\
		\hline
	\end{tabular}
	\subsection*{e)}
	$\pi_{c,a}$ = 
	\begin{tabular}{c|c}
		c&a \\
		\hline
		2&1 \\
		\hline
		8&5 \\
		\hline
		12&9
	
	\end{tabular}
	\subsection*{f)}
	$\pi_{a}$ = 
	\begin{tabular}{|c}
		a \\
		\hline
		1 \\
		\hline
		5 \\
		\hline
		9 \\
		\hline
	\end{tabular}
	\section*{Aufgabe 6}
	Die Kardinalität beschreibt die Anzahl der Tupel. Der Aufruf $\pi_{c}$(R)  verändert nicht die Anzahl der Tupel sondern stellt eine Abbildung der Attribute von R dar. Daher bleibt die Kardinalität m. $\to$ max(m) = m und min(m) = m.
	\section*{Aufgabe 7}
	\subsection*{a)}
	$\pi_{a,b,c,d}(R)$
	\subsection*{b)}
	$\delta(\pi_{g})$
	\subsection*{c)}
	Drachen - $\delta$(Drachen)
	\section*{Aufgabe 8}
	($\pi_{name}$(Drachen)-$\delta$($\pi_{name}$(Drachen)))-$\delta$($\pi_{name}$(Drachen))
\end{document}