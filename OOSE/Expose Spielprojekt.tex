\documentclass{article}
\begin{document}
	\section*{Exposé - Halligalliruckzuck}
	In diesem kurzen Exposé möchte ich kurz meine Projekt Weltraum Sauerstoffpaketbote vorstellen.
	Dieses Eposé soll dazu dienen das Werk und seine Funktionen darzustellen und zur Nutzung anregen. Im Folgenden möchte ich das Werk in all seinen zahlreichen Facetten kurz vorstellen.
	\section*{Funktionen}
	Das Spiel verfügt über Basisfunktionen und Geheimfunktionen. Die grundlegende Herausforderung liegt zum einen im Spiel selbst als Deutsche Post Raumschiff Sauerstoffpaket an Raumstationen zu liefern und zum Anderen nicht durch einen Asteroiden getroffen zu werden. \\
	Dafür stehen die Basisfunktionen des Spiels zur Verfügung. Die Steuerung erfolgt über die Pfeiltasten und das Spiel kann mit R neu gestartet werden. \\
	Die Geheimfunktionen werden hier nicht erklärt weil das Spiel gezielt Achivement-Hunter ansprechen sollen die es verstehen den Nervenkitzel im Suchen von Eastereggs zu sehen. Die Geheimfunktionen können über zufällige Tasten auf dem Keyboard aufgerufen werden.
	\section*{Weitere Erläuterungen}
	Ich möchte an dieser Stelle ausdrücklich darauf hinweisen, dass das gesamte Werk nahezu vollständig aus urheberrechtlich geschützten Teilwerken besteht und daher für eine Veröffentlichung nicht in Frage kommen kann. \\
	Des Weiteren wurde der ursprüngliche Projekttitel "NakteWeiberHalligalliruckzuck" aufgrund feministischer Wiederstände abgeändert und diverse Inhalte aus dem Spiel entfernt. \\
	Darüber hinaus wird darauf hingewiesen, das Soundfeatures nur unter Windows getestet wurden. Ob diese unter Linux funktionieren hängt von der jeweiligen Distribution ab.
	\section*{Besonderheiten der JavaFX Implementierung}
	Für das Starten der JavaFX Implementierung ist es erforderlich der JavaVM in den Runsettings mit den Parametern --module-path das Verzeichnis der JavaFX SDK zu übergeben bzw. mit --add-modules javafx.controls,javafx.fxml die notwendigen JavaFXmodule.
\end{document}