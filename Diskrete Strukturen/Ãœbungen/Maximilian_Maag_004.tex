\documentclass{article}
\usepackage{mathtools}
\begin{document}
	\section*{Lsg Vorschlag DS Ü004 Maximilian Maag}
	\section*{Aufgabe 1}
	\subsection*{a)}
	\begin{itemize}
		\item Für alle Elemente x aus dem Universum aller Lebewesen gilt, dass aus $A(x) B(x)$ folgt. "Alle Männer sind Schweine"
		\item Alle Lebewesen, die Männer sind sind auch Schweine.
		\item Es existiert mindestens ein Lebewesen, dass männlich und ein Schwein ist.
	\end{itemize}
	\subsection*{b)}
	Eine Folge $a_{n}$ hat genau dann einen Grenzwert g, wenn eine natürliche Zahl N $<$ n ist und es eine Zahl $\epsilon$ gibt und wenn gilt: $| a_{n} - g| < \epsilon $
	\section*{Aufgabe 2}
	\subsection*{a)}
	\{r\}
	\subsection*{b)}
	A $\cup$ C = \{p, q, r, s, t, v\}
	\subsection*{c)}
	\{p, r, u, w\}
	\subsection*{d)}
	A $\cap$ B $\cap$ C = \{\}
	\subsection*{e)}
	A $\cup$ B = \{r\} \\
	A $\cap$ C = \{q, s\} \\
	(A $\cup$ B) $\cap$ (A $\cap$ C) = \{\} \\
	\subsection*{f)}
	\{t, v\}
	\subsection*{g)}
	A $\cap$ C = \{q, s\} \\ 
	A $\setminus$ C = \{p, r\}
	\subsection*{h)}
	\{p, r, t, v\}
	
	\section*{Aufgabe 3}
	\subsection*{a)}
	Die symmetrische Differenz beschreibt die Differenz zwischen der Vereinigungsmenge von A und B und der Schnittmenge von A und B. 
	\subsection*{b)}
	Zeichnen entfällt aufgrund meiner Sehbehinderung.
	\subsection*{c)}
	(A $\triangle$ $\bar{B}$ ) $\triangle$ A $\equiv$ (A $\triangle$ A ) $\triangle$ A \\
	$\equiv$ (A $\cup$ A $\setminus$ A $\cap$ A ) $\triangle$ A \\
	$\equiv$ (A $\setminus$ A) $\triangle$ A \\
	$\equiv$ (A $\setminus$ A) $\cup$ A $\setminus$ (A $\setminus$ A) $\cap$ A \\
	$\equiv$ A $\setminus$ \{\} $\cap$ A \\
	$\equiv$ A $\setminus$ \{\} \\
	$\equiv$ A
\end{document}