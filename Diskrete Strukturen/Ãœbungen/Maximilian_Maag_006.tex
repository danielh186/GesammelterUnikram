\documentclass{article}
\usepackage{mathtools}
\usepackage{graphicx}
\begin{document}
	\section*{Lsg Vorschlag DS Ü006 Maximilian Maag}
	\section*{Aufgabe 1}
	$|A \cup B \cup C| = |A| + |B| + |C| - |A \cap B| - |A \cap C| + |A \cap B \cap C|$ \\
	$= 50 + 33 + 20 - |A \cap B| - |A \cap C| + |A \cap B \cap C|$ \\
	A = \{$ 2, 4, 6, 8, 10, 12, 14, 16, 18, 20, 22, 24, 26, 28, 30, 32, 34, 36, 38, 40, 42, 44, 46, 48, 50 ...$\} \\
	B = \{$3, 6, 9, 12, 15, 18, 21, 24, 27, 30, 33, 36, 39, 42, 45, 48 ... $\} \\
	C = \{$5, 10, 15, 20, 25, 30, 35, 40, 45, 50 ...$\} \\
	$A \cap B$ enthält maximal 33. Nur jede zweite Zahl von B ist auch in A also insgesamt 16. \\
	$A \cap C$ nur jedes zweite Element von C ist in A enthalten. C hat insgesamt 20 Elemente also $A \cap C$ = 10 Elemente. \\
	$A \cap C$ = \{10, 20, 30, 40, 50, 60, 70, 80, 90, 100 \}\\
	
	$(A \cap C) \cap B = \{30, 60, 90\}$ \\
	$(A \cap C) \cap B = A \cap B \cap C$ \\
	$A \cap B \cap C= 50 + 33 + 20 - |16| - |10| + |3| = 80$ \\
	\section*{Aufgabe 2}
	$\binom{n}{k} = \binom{n}{n - k}$ \\
	Beispiel $\binom{10}{1}$ \\
	$\frac{10!}{1! * (10 -1))!} = \frac{10!}{9!}$ \\
	$\binom{10}{10 - 1} = \frac{10!}{9! * (10 - 9)!} = \frac{10!}{9!}$ \\
	In der Schreibweise als Formel lässt sich n - k umformen sodass es als k geschrieben werden kann. 
	\section*{Aufgabe 3}
	\begin{figure}
		\includegraphics*{pic}
	\end{figure}
\end{document}