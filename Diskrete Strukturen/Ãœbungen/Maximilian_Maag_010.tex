\documentclass{article}
\usepackage{mathtools}

\begin{document}
	\section*{Lsg Vorschlag D+S Ü010 Maximilian Maag}
	\section*{Aufgabe A}
	Nummer zwei stimmt.
	\section*{Aufgabe B}
	
	zu zeigen: $1 * 2 + 2*2^2 .... n * 2^n = (n-1) * 2^{n+1} + 2$
	\\
 	Induktionsbasis: \\
 	setze ein n = 1 \\
 	$ n * 2^n = (n - 1) * 2^{n+1} + 2$ \\
 	$1 * 2^1 = (1 - 1) * 2^{1 + 1} + 2$ \\
 	$ 2 = 2 $ \\
 	Induktionsschritt: \\
 	$1 * 2 + 2*2^2 .... n * 2^n = (n-1) * 2^{n+1} + 2 $ \\
 	$n \to n+1$ \\
 	$\to (n-1) * 2^{n+1} + (n+1) * 2^{n+1} + 2$ \\
 	$(n-1) * 2^{n+1} + 2 = (n-1+n+1) 2^{n+1} + 2$ \\
 	$= (n-1+n+1) 2^{n+1} + 2$ \\
 	$= 2n 2^{n+1} + 2$ \\
 	$= n 2^{n+2} + 2$ \\
 	q.e.d
	\section*{Aufgabe C}
	zu zeigen: $1 + f_1 + f_2 + f_3 ... + f_n = f_{n+2}$ \\
	Basis: $setze n = 1$ \\
	$f_1 + f_1 = f_2$
	$1 + 1 = 2$ \\
	Indikationsschritt: $n \to n+1$ \\
	$f_{n+1} + f_{n+2} = f_{n+3}$ \\
	$f_{2} + f_{3} = f_4$ \\
	$2 + 3 = 5$ \\
	q.e.d
	\section*{Aufgabe 1}
	zu zeigen: $1 + 2 + 3 ... n^2 = \frac{n * (n+1)*(2n+1)}{6}$ \\
	Basis: setze n = 1 \\
	$1^2 = \frac{1 * (1+1)*(2+1)}{6}$ \\
	$1^2 = \frac{2 * 3}{6}$ \\
	$1^2 = \frac{6}{6}$ \\
	$1 = 1$ \\ \\
	Induktionsschritt: $n \to n+1$ \\
	$1 + 2 + 3 ... n^2 + (n+1)^2 = \frac{(n+1) * ((n+1)+1)*(2(n+1)+1)}{6}$ \\
	$\frac{(n+1) * ((n+1)+1)*(2(n+1)+1)}{6} = \frac{n * (n+1)*(2n+1)}{6} + (n+1)^2$ \\
	$\frac{(n+1) * ((n+1)+1)*(2(n+1)+1)}{6}= \frac{n *(2n^2+n+2n+1)}{6} + \frac{(n+1)^2}{1}$ \\
	$\frac{(n+1) * ((n+1)+1)*((2n+2)+1)}{6}= \frac{(2n^3+3n^2+n)}{6} + \frac{n^2 + 2n + 1}{1}$ \\
	$\frac{(n+1) * ((n+2))*((2n+2)+1)}{6}= \frac{(2n^3+3n^2+n)}{6} + \frac{6n^2 + 12n + 6}{6}$ \\
	$\frac{(n+1) * ((n+2)*((2n+2)+1)}{6}= \frac{(2n^3+3n^2+n+6n^2 + 12n + 6)}{6}$ \\
	$\frac{(n+1) * (2n^2+2n+n+4n+4+2)}{6}= \frac{(2n^3+9n^2+13n + 6)}{6}$ \\
	$\frac{2n^3+2n^2+n^2+4n^2+4n+2n + 2n^2+2n+n+4n+4+2}{6}= \frac{(2n^3+9n^2+13n + 6)}{6}$ \\
	$\frac{2n^3+9n^2+13n+6}{6} = \frac{2n^3+9n^2+13n + 6}{6}$ \\
	q.e.d
	\section*{Aufgabe 2}
	Zu zeigen: $1+q+q^2+q^3...+q^n = \frac{1-q^{n+1}}{1 - q}$ \\
	Induktionsbasis: setze n = 0 \\
	$1+q+q^2+q^3...+q^1 = \frac{1-q^{n+1}}{1 - q}$ \\
	$q^0 =\frac{1-q^{0+1}}{1 - q} $ \\
	$1 =\frac{1-q}{1 - q} $ \\
	$1 = 1 $ \\
	Induktionsschritt: $n \to n+1$ \\
	$1 + q^1 + q^2 + q^n + q^{n+1} = \frac{1-q^{n+1+1}}{1-q}$ \\
	$\frac{1-q^{n+1}}{1-q} + q^{n+1} = \frac{1-q^{n+2}}{1-q}$ \\
	$1-q^{n+1} + (1-q)q^{n+1} = 1-q^{n+2}$ \\
	$1-q^{n+1} + q^{n+1}  - q^{n+2} = 1-q^{n+2} $ \\
	$1 - q^{n+2} = 1 - q^{n+2}$ \\
	q.e.d
	\section*{Aufgabe 3}
	zu zeigen: $1 +2^3 + 3^3 + ... n^3 = (1 + 2 + 2 ... + n)^2$ \\
	Tipp: $(1 + 2 + 2 ... + n)^2 \equiv (\frac{n(n+1)}{2})^2$ \\ \\
	Induktionsbasis: n = 1\\
	$1^3 = 1^2$ \\ \\
	Induktionsschritt: $n \to n+1$ \\
	$1 + 2^3 + 3^3 ... + n^3 + (n+1)^3 =  (\frac{(n+1)(n+2)}{2})^2$ \\
	$(\frac{n*(n+1)}{2})^2 + (n+1)^3 =  (\frac{(n+1)(n+2)}{2})^2$ \\
	$\frac{n^2*(n+1)^2}{4} + (n+1)^3 = \frac{(n+1)^2(n+2)^2}{4}$ \\
	$\frac{n^2*(n+1)^2}{4} + n^3 + 3n^2 + 1 = \frac{(n+1)^2(n+2)^2}{4}$ \\
	$n^2*(n+1)^2 + 4n^3 + 12n^2 + 12n + 4 = (n+1)^2(n+2)^2$ \\
	$n^2*(n^2 + 2n +1) + 4n^3 + 12n^2 + 12n + 4 = (n+1)^2(n+2)^2$ \\
	$n^4 + 2n^3 + n^2 + 4n^3 + 12n^2 + 12n + 4 = (n+1)^2(n+2)^2$ \\
	$n^4 + 6n^3 + 13n^2 + 12n + 4 = (n+1)^2(n+2)^2$ \\
	$n^4 + 6n^3 + 13n^2 + 12n + 4 = (n^2 + 2n + 1)(n^2 + 4n + 4)$ \\
	$n^4 + 6n^3 + 13n^2 + 12n + 4 = n^4 + 4n^3 + 4n^2 + 2n^3 + 8n^2 +8n + n^2 + 4n + 4$ \\
	$n^4 + 6n^3 + 13n^2 + 12n + 4 = n^4 + 6n^3 + 13n^2  +12n + 4$ \\
	q.e.d
\end{document}