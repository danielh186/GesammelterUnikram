\documentclass{article}
\usepackage{mathtools}
\begin{document}
	\section*{Lsg Vorschlag D+S Ü009 Maximilian Maag}
	\section*{Aufgabe A}
	\subsection*{a)}
	$f(x) = x + 1$; $g(x) = 2x$ \\ \\
	$f(g(x)) = 2x + 1$
	\subsection*{b)}
	Funktion besteht aus Polynom 1. Grades. \\
	Daraus ergibt sich immer eine eindeutige Zuordnungg von x $\to$ y und von y $\to$ x. \\ \\
	$\forall\{x \in R | f(g(x)): x \to y\}$ daraus folgt Injektivität \\
	$\forall\{y \in R | f(g(x)): y \to x\}$ daraus folgt Surjektivität \\ \\
	Aus Surjektivität und Injektivität folgt Bijektivität.
	\subsection*{c)}
	$f(x) = 2x + 1$ \\
	$y = 2x + 1$ \\
	$x = 2y + 1$ \\
	$x - 1 = 2y$ \\
	$y = \frac{1}{2}x - \frac{1}{2}$ \\
	$f^{-1}(x) = \frac{1}{2}x - \frac{1}{2}$
	\section*{Aufgabe B}
	Die Menge der natürlichen Zahlen kann auf die Menge ungeraden Zahlen abgebildet werden. \\ \\
	 mit f(n) = n+1 falls n gerade ist \\ \\
	 mit f(n) = n * (-1) falls n ungerade ist.
	\section*{Aufgabe 1}
	Wenn n + m eine ungerade Zahl ist, dann ist entweder m oder n ungerade. \\
	\\
	Annahme n oder m ungerade ist falsch. \\
	$\to$ n und m müssen gerade sein. \\
	Wenn n und m gerade sind ergibt ihre Addition aber keine ungerade Zahl. \\
	Daraus folgt: eine von beiden Zahlen muss ungerade sein.
	\section*{Aufgabe 2}
	Kategorieren sein der Rest der Division  für die Zahl x durch 8. Die Zahl x ergibt sich aus der Differenz zweier beliebiger Zahlen $\in$ N aus einer Menge von 9 Zahlen. \\ \\
	$K_0 \equiv \{x \in N | x \equiv_8 0\}$ \\
	$K_1 \equiv \{x \in N | x \equiv_8 1\}$ \\
	$K_2 \equiv \{x \in N | x \equiv_8 2\}$ \\
	$K_3 \equiv \{x \in N | x \equiv_8 3\}$ \\
	$K_4 \equiv \{x \in N | x \equiv_8 4\}$ \\
	$K_5 \equiv \{x \in N | x \equiv_8 5\}$ \\
	$K_6 \equiv \{x \in N | x \equiv_8 6\}$ \\
	$K_7 \equiv \{x \in N | x \equiv_8 7\}$ \\
	\\
	Unter dieser Annahme gibt es aus 9 Zahlen immer eine Kathegorie mit zwei x. \\
	Daraus ergibt sich, dass zwei zahlen den gleichen Restwert aufweisen müssen. Wodurch sich bei der Differenz eine Teilbarkeit durch 8 ergibt.
	\section*{Aufgabe 3}
	\subsection*{a)}
	Die Anzahl der Zahlen muss größer sein als die Anzahl an Kategorieren die wie in Aufgabe zwei gebildet werden können. Aus den A+M-Folien ergibt sich damit. \\
	9 muss größer als n * r sein.
	\subsection*{b)}
	Gegenbeispiel Summe aus 12 und 22. Beide haben bei einer Division durch 5 den gleichen Rest, ihre Summe ist allerdings nicht ohne Rest durch 5 teilbar.
	\subsection*{c)}
	Nach dem Taubenschlag-Prinzip sein $K_{0 \to 4}$ Kategorien in denen Zahlen insgesamt 11 Zahlen $\in$ N abgelegt werden. \\ \\
	$K_0 \equiv \{x \in N | x \equiv_5 0\}$ \\
	$K_1 \equiv \{x \in N | x \equiv_5 1\}$ \\
	$K_2 \equiv \{x \in N | x \equiv_5 2\}$ \\
	$K_3 \equiv \{x \in N | x \equiv_5 3\}$ \\
	$K_4 \equiv \{x \in N | x \equiv_5 4\}$ \\
	\\
	11 beliebige Zahlen können nicht gleichmäßig verteilt werden. Es gibt immer eine Kategorie in der alle drei Zahlen den selben Rest haben. Die Differenz zweier Zahlen mit dem selben Rest lässt sich durch 5 dividieren.
\end{document}