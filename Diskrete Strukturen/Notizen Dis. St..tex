\documentclass{article}
\usepackage{mathtools}
%\usepackage{hyperref}
\begin{document}
	\section*{Notizen Diskrete Strukturen}
	\section*{Was kommt in die Klausur? - Eine Roadmap zum lernen und üben}
	Was kommt in die Klausur dran? \\
	1. Logik: \\
	
	Wahrheitswertetafel ausfüllen, \\
	
	Beweise \\
	
	Umformungen \\
	
	Vereinfachungen \\
	
	(AND, OR, XOR, IMPLIKATION, FÜRALLE, ES EXISTIERT, ....) \\
	
	(sprachlich < -- > formal) \\
	
	
	2. Mengen: \\
	
	Mengen bestimmen: \\
	
	aus vorgegebenen Mengen andere Mengen bilden (schnittmenge / vereinigung / verneinung, etc...) \\
	
	Rechengesetze anwenden \\
	
	Mächtigkeiten von Mengen: \\
	
	Summenformel (bis zu 3 Mengen) \\
	
	Binomialkoeffizient können \\
	
	
	3. Relationen und Funktionen: \\
	Relation \\ 
	Umkehrrelation \\
	Komposition (R o S) \\
	Äquivalenzklassen \\
	Äquivalenzrelation \\
	
	reflexiv \\
	
	symmetrisch \\
	
	transitiv \\
	
	Funktionen: \\
	
	injektiv \\
	
	surjektiv \\
	
	bijektiv \\
	
	Umkehrfunktionen bestimmen! \\
	4. Beweisen: \\
	- Schubfachprinzip(auch erweitert) \\
	- vollständige Induktion  (Summenformeln, Fibonacci) \\
	
	5. Graphentehorie     \\
	
	- Eulersche Graphen \\
	
	- Eulerscher Kreis \\
	
	- Offene Eulersche Linien \\
	
	-Vollständige Graphen \\
	
	-Vollständig bipartide Graphen \\
	
	-planare Graphen \\
	
	-Eulersche Polyederformel \\
	6. Algebraische Strukturen \\
	
	-Mod, ggT((erweiterter)Euklid) \\
	
	-Vielfachsummendarstellung \\
	
	- Gruppen(Z ohne und mit Stern) \\
	
	-Multiplikationstafeln aufstellen \\
	
	-inverse bestimmen \\
	
	- Ringe \\
	
	- Körper \\
	
	
	
	\subsection*{06.11.2020}
	Allgemeine Hinweise. Zu Vorlesungen und Übungen etc.
	Jens betont mann könne ihm jederzeit eine E-Mail schreiben.
	Vorlesung ging mit Technischen Problemen vorüber.
	Diskrete Mathematik für Kryptographie. Endliche Strukturen.
	Vereint endliche Phänomene Beispiel Algebra, Graphentheorie. Dient als Grundlage für Informatik. \linebreak
	Logik, Mengen,, Relationen und Funktionen, Weweise, Graphentheorie, Algebraische Grundstrukturen \linebreak
	Induktionsbeweise
	Definitionen, Sätze, Beweis, Beispiele
	\subsection*{Logik}
	Aussagen, Distributionen, Aussagen sind wahr oder falsch nie beides \linebreak
	Aussage besitzt Wahrheitswert. Paradoxon, ist keine Aussage weil sie nicht wahr oder falsch ist.
	
	\subsection*{20.11.2020}
	Junktoren boolsche Algebra \\
	$\neq$ nicht; $\land$ und; $\lor$ oder\\
	Dualität ein Satz folgt aus einem anderen wenn $\land$ und $\lor$ vertauscht werden. \\
	Absorptionsgesetz \\
	Idempotenzgesetze \\
	Doppelte Negation (doppeltes nicht hebt sich auf)\\
	A $\land$ A = A Idepotenzgesetz \\
	Vereinfachen Sie: (A $\land$ B) $\lor$ ($\not$A $\land$ B ) \\
	$\equiv$ B $\land$(A $\lor$ $\not$A) \\
	$\equiv$ B $\land$ w \\
	$\equiv$ B \\
	Disjunktive Normalform umformen.\\
	Mindestens zwei aus drei Schaltung und Disjunktive Normalform \\
	WolframAlpha \\
	\subparagraph*{Aussageform}
	Aussageform Aussage entsteht erst indem im Term Variable eine Zahl bekommt \\
	$A(x) = x$ ist eine Primzahl \\
	$A(2) = w$; $A(0) = f$ Aussage entsteht durch Belegung der Variable x. \\
	Universum ist Definitionsbereich der Aussageform. \\
	Aussageform kann verschiedene Variablen haben von denen jede Variable ein eigene Universum haben kann. \\
	\subparagraph*{Quantoren}
	Allaussagen gelten für alle Elemente eines Universums einer Aussageform. Allquantor großer $\land$ Operator \\
	Existenzaussage $\equiv$ verkettete Oder-Aussage es gibt mindestens ein Element; Existenzquantor \\
	All- und Existenzaussagen können auch negiert werden \\
	
	\section*{27.11.2020}
	Mengen \\
	Mächtigkeiten von Mengen bestimmen \\
	Binomialzahlen. \\
	Def. Menge nach Cantor Mengen sind Zusammenfassung von wohlunterschiedenen Objekten unseres Denkens oder unserer Anschauung. \\
	Menge beschreiben durch Aufzählung und Beschreibung. \\
	Reihenfolge der Objekte einer Menge ist nicht entscheidend. Objekte müssen sich klar unterscheiden. \\
	a $\in$ M oder a $\notin$ M. \\
	Berühnmte Mengen Zahlenmengen. \\
	N $\in$ Z $\in$ Q $\in$ R. \\
	Menge durch Beschreibung Y = \{ $x \in X | A(x)$ \}
	Menge durch Aufzählung A = \{3, 5, 7, 9\} \\
	(a, b) = \{x $\in R | a < x < b$\} offenes Intervall \\
	Schnittmenge A $\cap$ B enthält alle Objekte, die in A und B enthalten sind. \\
	Vereinigungsmenge A $\cup$ B enthält alle objekte, die in A oder in B enthalten sind. \\
	Wenn A $\cap$ B = \{\} dann sind A und B disjunkte Mengen. \\
	A $\setminus$ B gesp. A ohne B. Differenzmenge \\
	A $\subset$ B A Teilmenge von B. \\
	A $\subseteq$ B A ist echte Teilmenge von B. \\
	Komplement einer Menge ist Differenz Universum und Menge. \\
	De Morgansche Gesetze für Mengen beachten. \\
	Beweise können relevant für Klausur sein. \\
	es gelten Rechengesetze für Mengen. \\
	Distributivgesetz, Komutativgesetz, Komplement, neutrales Element. \\
	\section*{04.12.2020}
	Kapitel 2 Mengen \\
	Kartesisches Produkt \\
	A x B = {(a,b) $|$ a $\in$ A b $\in$ B} \\
	Kombination von Menggen. \\
	Innerhalb der Ergebnismenge kommt es nicht auf die Reihenfolge an in den Ergebnispaaren allerdings schon. \\
	
	Für zwei nicht leere mengen A und B ist das kartesische Produkt das Kreuzprodukt von A und B. \\
	Kartesisches Produkt mit leerer Menge ist eine leere Menge. \\
	
	Distributivgesetz: (A $\cup$ B) x C $\equiv$ (A X C) $\cup$ (B X C) \\
	KP kann auch über n Mengen gebildet werden.
	
	\section*{11.12.2020}
	Mächtigkeit von Mengen. \\
	Anzahl der Elemente einer menge. \\
	Mächtigkeit kann eine endliche natürliche Zahl. \\
	Mächtigkeit kann aber auch $\infty$ sein. \\
	Mächtigkeit Vereinigungsmenge $| A \cup B | = |A| + |B| - |A \cap B|$ \\
	Disjunkte Menge |A| +  |B| \\
	Mächtigkeit von Drei endlichen Mängen \\
	 $|A \cup B \cup C| = |A| + |B| + |C| - |A \cap B| - |A \cap C| - |B \cap C| + |A \cap B \cap C|$ \\
	 Summe beliebig vieler Mengen (Siebformel)  \\
	 $|A_{1} \cup A_{2} \cup ... A_{n}| = \alpha_{1} - \alpha_{2} + \alpha_{3} +- ....$ \\
	 $\alpha_{1}$ wird mit Durschnitt einer Menge gebildet. \\
	 $\alpha_{2}$ Wird mit Durchschnitt von zwei Mengen gebildet. \\
	 Siebformel kann nicht negativ sein. \\
	 $|A x B| = |A| * |B|$ \\
	 Potenzmenge \\
	 Def. Die Menge aller Teilmengen von M heißt Potenzmenge $\mathcal{P}$(M) von M. \\
	 $|\mathcal{P}(M)| = 2^{n}$ \\
	 k-elementige Teilmengen \\
	 Binomialzahl $\binom{n}{k}$ \\
	 Anzahl der k-teiligen Teilmengen einer n teiligen Menge berechnen. \\
	$ \binom{n}{k} = \frac{n!}{k!*(n-k)!}$ \\
	\subsection*{Kaptiel 3 - Relationen und Funktionen}
	Definition einer Funktion. \\
	Relationen, Funktionen, Abzählbarkeit \\
	Relationen \\
	Relationen schränken ein KP ein und betrachten eine Teilmenge mit bestimmter Beziehung. \\
	R $\subset$ A x B; x R y
	\section*{18.12.2020}
	Eine Relation ist eine Teilmenge eines Kathesischen Produktes. \\
	Umkehrrelation. \\
	$R^{-1}$ enthält Paare aus B x A anstatt aus A x B, für alle Paare die in R enthalten sind. \\
	Komposition verbindet zwei Relationen. \\
	a-Element aus A b-Element aus C wenn es ein b gibt, dass in B und C verbindet. \\
	Relationen zwischen gleichen Mengen \\
	reflexsiv: x R x \\
	symetrisch: $x R y \to y R x$ \\
	transitiv: x R y $\land$ y R z $\to$ x R z \\
	Gleichheitsrelation: x = y \\
	Symmetrie \\
	$R^-1 = R$ \\
	Äquivalenzrelation sind Relationen, dir reflexiv, symmetrisch und transitiv.
	\section*{08.01.2021}
	Def. Kongruenz Relation ist $\subset$ des kathesischen Produkts X x X. \\
	Kongruenz Relation \\
	$\equiv_n = \{(x, y) \in Z^2 y -x\}$ Ist ein ganzzahliges Vielfaches von n. \\
	$\equiv_n$ ist eine Äquivalenzrelation und damit transitiv, reflexsiv und symmetrisch. \\
	Def. modulo ist der kleinste nicht negative Rest, mit einem Teiler $>$ 1. 0 eingeschlossen. \\
	Def. Äquivalenzklasse sei ~ eine Äquivalenzrelation auf der Menge X und sei x $\in$ X dann heißt die Menge: \{y $\in$ X $|$ y $\tilde{}$ x\} die Äquivalenzklasse von x. \\
	Anmerkung: Beweisen wird teil der Klausur. Vollständige Indunkttion. \\
	Äquivalenzklassen sind gleich oder disjunkt. \\
	Jedes Element x $\in$ X liegt in einer Äquivalenzklasse. \\
	Def. Partition Menge von Äquivalenzklassen von X sind paarweise disjunktgge Teilmengen von X die zusammen Z ergeben. \\
	Def. injenktiv Eine Funktion f bildet für zwei x werte auf einen y wert ab. \\
	Def. surjektiv wenn es für jedes y ein x. Kein y bleibt unbesetzt. \\
	Def. bijektiv wenn f surjektiv und injektiv. \\
	Einache Erklärung auf Youtube \\
%	\url{https://www.youtube.com/watch?v=OMCD3KRp5_k}
	\section*{15.01.2021}
	jede bijektive Funktion ist umkehrbar. \\
	Umkehrfunktion heißt: $f^{-1}(x)$. \\
	$f^{-1}(x)$ hebt $f(x)$ auf. \\
	Definitions- und Wertebereich tauschen. \\
	Funktion umkehren: \\
	1. ersetze f(x) durch y. \\
	2. ersetze x durch y. \\
	3. löse nach y auf. \\
	Verfahren gilt nur für bijektive Funktionen. \\
	Falls $f(x)$ nicht bijektiv schränke Definitionsbereich ein!!!!!!! \\
	Abzählbarkeit \\
	Def. Zwei Mengen A, B heißen gleich mächtig, wenn es eine bijektive Abbildung (Funktion) von A nach B gibt.
	Def. Eine Menge heißt Abzählbar, wenn sie endlich ist oder gleichmäßig zu N ist. \\
	Überabzählbare Menge z.B. Potenzmenge. \\
	Kapitel 4 Beweise \\
	Beweisarten, Schubfachprinzip, Vollständige Induktion \\
	Mathematische Sätze müssen allgemein bewiesen oder durch Gegenbeispiel widerlegt werden. \\
	Jeder Satz ist eine Wenn-Dann-Aussage (A $\to$ B) \\
	Aus einer Voraussetzung folgt eine Behauptung. \\
	Kontraposition Anstatt A $\to$ B wird $\not$ A $\to$ $\not$ B \\
	Widerspruch um A $\to$ B zu zeigen nimmt man an B ist falsch. Dies führt zu einem Widerspruch. \\
	Beweis von Äquivalenzen A $\to$ B $\land$ B $\to$ A. Beide Richtungen zeigen. \\
	\section*{29.01.2021}
	Kapitel 5 Graphentheorie \\
	Bäume Planarität \\
	Graphen bestehen aus Ecken und kannten. Eine Kante verbindet zwei Punkte. \\
	Ein Graph ist ein Tupel aus der Menge der Knoten und Kanten. \\
	ungerichteter Graph Linie verbindet zwei Punkte (Teilmenge der Punkte) \\
	gerichteter Graph Punkte werden durch einen Pfeil verbunden (Relation auf die Eckenmenge) \\
	Vollständiger Graph alle Ecken sind mit je einer Kante verbunden. \\
	vollständiger Graph hat n über zwei Kanten. \\
	Graphen sind zusammenhängend, wenn jede Ecke über eine Folge von Kanten erreicht werden kann. \\
	Grad einer Ecke ist die Anzahl der ausgehenden Kanten. \\
	Bei einem vollständigen Graphen $K_n$ hat jede Ecke den Grad n - 1. \\
	Kantenzug ist eine Folge von Kanten. \\
	eulerscher Kreis Kantenzug wenn jede Kante unter den Kanten k1 bis ks des Kantenzugs auftaucht. \\
	zzgl. der letzte Punkt ist der erste Punkt. \\
	Eulerscher Graph ist ein Graph der einen eulerschen Kreis besitzt. \\
	Wenn ein Graph eulersch ist hat jede Ecke einen geraden Grad. \\
	Offene eulersche Linie durchläuft jede Kante einmal. Die Anfangsecke darf aber ungleich der Endecke sein. \\
	Graph enthält eulersche Linie wenn er genau zwei Ecken ungeraden Grades enthält. \\
	Bäume \\
	Ein Baum ist ein Graph, der zusammenhängend ist und keinen Kreis enthält. \\
	Planarität \\
	Graph kann ohne Überschneidung von Kanten in der Ebene gezeichnet werden. \\
	eulersche Polyederformel: n - m + g = 2 \\
	n Ecken, m Kanten, g Gebiete. \\
	bipartit ein Graph bekommt schwarze und weiße Ecken. Jede Kante verbindet einen schwarzen und einen weißen Punkt. \\
	vollständig bipartit wenn alle weißen mit allen schwarzen Ecken verbunden. \\
	\section*{05.02.2021}
	Gruppen
	Abgeschlossenheit, Assoziativität, inverses Element, Kommutativität \\
	diverse Beispiele für Gruppen. \\
	Eine Gruppe muss ein neutrales Element enthalten. \\
	teilerfremd \\
	Zwei Zahlen sind teilerfremd wenn sie nur die 1 als gemeinsamen Teiler haben. 
	ggT Algorithmus, größter gemeinsamer Teiler. \\\
	ggT kann als Vielfachsumme der Originalzahlen dargestellt werden. \\
	Sind die Originalzahlen teilerfremd kann die 1 als Vielfachsumme dargestellt werden. \\
	Auf diesen Operationen beruht z.B. der RSA Algorithmus. \\
	\section*{19.02.2021}
	Ringe, Körper \\
	Ring sind diskrete Strukturen in denen man addieren und multiplizieren kann. \\
	FÜr die Addition müssen alle Gruppenaxiome gelten. \\
	Addition: abgeschlossen, assoziativ, neutrales Element, kommutativ \\
	Multiplikation: Abgeschlossenheit, Assoziativität \\
	Distributivgeseteze \\
	gerühmter Ring: Menge der ganzen Zahlen. \\
	Ring: Addition mod n und Multiplikation mod n \\
	Polynomring R[x] \\
	Ganze Zahlen und Polynome sind beides Ringe. Darauf basiert Polynomdivision \\
	algebraischer Körper: alle Gesetze für Ring + multiplikative Inverse \\
	Beispiel Q und R\\
	Endliche Körper \\
	Wenn p eine Primzahl ist ist $Z_p$ ein endlicher Körper \\
	Es gibt einen Körper mit q Elementen, genau dann wenn q Primzahlpotenz \\
\end{document}