\documentclass{article}
\usepackage{mathtools}
\begin{document}
	\section*{Lsg Vorschlag DS Ü007 Maximilian Maag}
	\section*{Aufgabe A}
	\subsection*{a)}
	$R^{-1} = \{(x,a), (x,b), (y,c), (z,c)\}$ \\
	$S^{-1} = \{(u,x), (v,z)\}$
	\subsection*{b)}
	$R o S = \{(a,u), (b,u), (c,v)\}$
	\subsection*{c)}
	$B x C = \{(u, x), (u,y), (u,z), (v,x), (v,y), (v,z)\}$ \\
	$\overline{S} = \{(u,y), (u,z), (v,x), (v,y)\}$
	\section*{Aufgabe B}
	$A = \{0,1,2,3\}$ \\
	$R_< = \{(0<1), (0<2), (0<3), (1<2), (1<3), (2<3)\}$ (absolut) transitiv \\
	$R_= = \{(0 = 0), (1 = 1), (2 = 2), (3 = 3)\}$ reflexiv, symmetrisch, transitiv \\
	$R_\geq = \{\}$ transitiv, reflexsiv \\
	$R_\neq = \{\}$ symmetrisch
	\section*{Aufgabe 1}
	\subsection*{a)}
	Max hat zwei Kinder. Moritz ist annas Opa.
	\subsection*{b)}
	R o S =  \{(Max, Bert), (Max, Eva), (Moritz, Hilde)\} \\
	Petra ist mit Moritz verheiratet also die Oma von Anna. Die Relation R o S gibt das Verhältnis Schwiegereltern Schwiegersohn/Tochter an.
	\section*{Aufgabe 2}
	\subsection*{a}
	symmetrisch, denn $y^2 = x^2$ ist zulässig. \\
	reflexiv $x^2 = x^2$ \\
	transitiv durch Gleichheit und damit äquivalent.
	\subsection*{b}
	$x+y = 42 \equiv y+x = 42$ symmetrisch \\
	$x + x = 42$ reflexiv nicht erfüllt für z.B. x = 20 und y = 22  \\
	nicht äquivalent
	\subsection*{c}
	transitiv $x + y = z, z$ ist gerade $\equiv x + y$ ist gerade \\
	reflexiv $x + y \to$ ist gerade $\equiv y + x \to$ ist gerade. \\
	Eine gerade Zahl sei $x + y$, symmetrisch. \\
	äquivalent
	\subsection*{d}
	$y * x$ ist gerade. symmetrisch \\
	$x * x$ ist gerade. reflexiv für (3,2) nicht erfüllt \\
	nicht äquivalent
	\subsection*{e}
	$R = \{(x,y) \in Z^2 | x \leq y\}$ \\
	Diese Relation enthält u.a. das Tupel (-2,1) für das keine Symmetrie festgestellt werden kann, daher kann die Relation R auch nicht äquivalent sein.
	\section*{Aufgabe 3}
	\subsection*{a}
	$R_1 = \{(x,y) | x,y \in R, x^2 + y^2 \leq 1\}$
	\subsection*{b}
	$R_2 = \{(x,y) | x,y \in R, x \geq 1, y \geq 1\}$
	\subsection*{c}
	Unterteilung der Fläche im Graphen c in zwei Hilfsflächen mit eigener Relation: \\
	h1: Dreieck: \\
	$R_{h1} = \{(a,b) | \in R, 0 \leq a \leq 1, 0 \leq b \leq 1, (a * b)\frac{1}{2} \leq \frac{1}{2} \}$ \\
	h2: $\frac{1}{4}$Kreis im oberen rechten Quadranten.	(Überdeckt Dreieck): \\
	$R_{h2} = \{(c,d) | \in R, 0 \leq c \leq 1, 0 \leq d \leq 1, x^2 + y^2 \leq 1 \}$ \\
	Daraus ergibt sich: legt man die Relation für den Kreis an und subrahiert ein Dreieck mit der richtigen Größe so ergibt sich die markierte Fläche. \\
	Das Dreieck mit der richtigen Größe muss folgende Punkte enthalten A(0,0), B(0,1), C(1,0). Die Fläche dieses Dreiecks lässt sich bestimmen und ist $\frac{1}{2}$. \\
	Die Relation aus a wird angepasst sodass sie einen viertel Kreis im 1. Quadranten ergibt. Davon wird das Dreieck h1 abgezogen. \\	
	$R_{h2 - h1} = \{(c,d) | \in R, ((c^2 + d^2) - \frac{1}{2} \leq 1), 0 \leq c \leq 1, 0 \leq d \leq 1\}$ \\
	Diese Relation ergibt mit x und y geschrieben. \\
	$R_3 = \{(x,y) | \in R, ((x^2 + y^2) - \frac{1}{2} \leq 1), 0 \leq x \leq 1, 0 \leq y \leq 1\}$ \\
\end{document}