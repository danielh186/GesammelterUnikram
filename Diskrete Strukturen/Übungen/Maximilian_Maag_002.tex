\documentclass{article}
\usepackage{mathtools}
\begin{document}
	\section*{Lösungsvorschlag Ü DS 002 Maximilian Maag}
	\section*{Aufgabe A}
	a) f w = f
	b) [w] w = w \\
	c) = w \\
	d) = w \\
	e) = w \\
	
	Wenn oder-Tabelle bekannt dann zählt das oder.
	
	\section*{Aufgabe B}
	
	a) heute ist nicht Samstag w
	b) w Logik ist nicht nützlich und der mond ist nicht \\ \\ aus Käse f \\
	c) f 8 * 7 $\not 55 w \\
	d) f \not(3 + 4 = 7) \lor \not(3 * 4 > 12) w \\
	e) f  dann w$ \\
	
	\section*{Aufgabe C}
	
	a) w \\
	b) f \\
	c) f \\
	d) w \\
	e) w \\
	
	\section*{Aufgabe 1}
	\subsection*{a)}
	Der Oktober hat 31 und der November 30 Tage. \\
	A = Der Oktober hat 31 Tage; B = Der November hat 30 Tage. \\
	$A$ = w; $B$ = w \\
	$A \land B$ = w \\
	\subsection*{b)}
	Entweder der Juni oder der November hat 30 Tage.
	A = der Juni hat 30 Tage; B = Der November hat 30 Tage \\
	$A$ = f; $B$ = w; \\
	$A \lor B$ = w \\
	\subsection*{c)}
	Es ist kein Tag genau dann, wenn Nacht ist. \\
	A = Es ist kein Tag; B = Es ist Nacht. \\
	$A \to B$ = w; \\
	
	\subsection*{d)}
	Heute ist Samstag und kein Sonntag.
	A = Heute ist Samstag; B = heute ist kein Sonntag. \\
	$A \land B$ = f; \\
	
	\subsection*{e)}
	Wenn Samstag ist, dann ist kein Sonntag. \\
	A = Es ist Samstag B = Es ist kein Sonntag. \\
	$A \to B$ = w; \\
	
	\section*{Aufgabe 2}
	\subsection*{a)}
	Mein Gegner ist schneller als, das bedeutet ich verliere. \\
	\subsection*{b)}
	Wenn eine Zahl $n$ nicht durch 3 teilbar ist, ist sie auch nicht durch 6 teilbar.
	\subsection*{c)}
	Wenn eine zahl $n$ gerade ist, kann sich nicht gleichzeitig ungerade sein und umgekehrt.
	\subsection*{d)}
	Ist ein Auto kaputt oder dessen Tank leer, muss man es schieben.
	\subsection*{e)}
	Ich bin weder doof noch faul, also schaffe ich auch mein Studium.
	
	\section*{Aufgabe 3}
	\subsection*{a)}
	$A$ = f; \\
	$B$ = w; \\
	$A \to B$ = f; \\
	\subsection*{b)}
	$A$ = f; $B$ = w; $C$ = w; \\
	$(A \to B) \to C$ = f; \\
	\subsection*{c)}
	$A$ = w; $B$ = f; \\
	$A \leftrightarrow B$ = f; \\
	
	\subsection*{d)}
	$A \to B$ = f; \\
	\subsection*{e)}
	$A \to B$ = w; \\
\end{document}