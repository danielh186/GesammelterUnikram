\documentclass{article}
\usepackage{mathtools}
\begin{document}
	\section*{Lsg Vorschlag D+S Ü008 Maximilian Maag}
	\section*{Aufgabe A}
	\subsection*{a)}
	Zuordnung eindeutig x $\to$ y, deshalb Funktion.   
	\subsection*{b)}
	Zuordnung eindeutig, injektiv x $\to$ y eindeutig. y $\to$ x wird ein y ausgelassen nicht surjektiv.
	\subsection*{c)}
	Zuordnung eindeutig, Funktion. Surjektiv.
	\subsection*{d)}
	3 wird von X nicht eindeutig auf Y zugeordnet, es handelt sich um keine Funktion.
	\subsection*{e)}
	Eindeutige Zuordnung, Funktion. Alle X werden nur auf ein Y zugeordnet injektiv. y $\to$ x eindeutig also surjektiv und damit bijektiv.
	\section*{Aufgabe B}
	\subsection*{a)}
	$f(x) = 3x - 5$. \\
	Für jedes x nur ein y. für jedes y nur ein x. bijektiv. \\
	$f(x) = 10^x$ ähnelt e-Funktion. Nähert sich x-Achse für x $\to$ -$\infty$ und steigt streng monoton für x $\to$ $\infty$ \\
	x $\to$ y eindeutig, y $\to$ x  nicht (keine negativen Zahlen ) eindeutig \\
	$f(x) = x^4$ ähnelt Parabel, daraus folgt weder injektiv noch surjektiv.
	\subsection*{b)}
	$f(x) = 10x^3 + 2x^2$
	\section*{Aufgabe C}
	\{0\} $\equiv \{y \in Z | y \equiv_2 0 \}$
	\{1\} $\equiv$ \{y $\in$ Z $| y \equiv_1 1$\}
	
	\section*{Aufgabe 1}
	\subsection*{a)}
	$2n+1 mod n = 1$
	\subsection*{b)}
	$n^2 mod n = 0$
	\subsection*{c)}
	$(2n+2)(n+1) = (n+1)^2$ binomische Formeln \\
	$(n+1)^2 mod_n 1^2 = 2$
	\subsection*{d)}
	$n! = n * (n-1) * (n-2) ... * 2 * 1)$ \\
	$n! mod n = 0$
	\section*{Aufgabe 2}
	\subsection*{a)}
	$\{0\} \equiv \{y \in Z | y \equiv_5 0\}$ \\
	$\{1\} \equiv \{y \in Z | y \equiv_5 1\}$ \\
	$\{2\} \equiv \{y \in Z | y \equiv_5 2\}$ \\
	$\{3\} \equiv \{y \in Z | y \equiv_5 3\}$ \\
	$\{4\} \equiv \{y \in Z | y \equiv_5 4\}$ \\
	\subsection*{b)}
	Es sei die Menge M: \{1111, 1110, 1100 ... 0000\} eine Menge mit allen vierstelligen Binärzahlen. \\
	$\{1111\} \equiv \{y \in M | y \equiv_{1111}  1111\}$ \\
	$\{0110\} \equiv \{y \in M | y \equiv_{0110}  0110\}$ \\
	$\{0111\} \equiv \{y \in M | y \equiv_{0111}  0111\}$ \\
	$\{1110\} \equiv \{y \in M | y \equiv_{1110}  1110\}$ \\
	\section*{Aufgabe 3}
	\subsection*{a)}
	für $\leq$
	a, b $\in$ R \\
	a $\leq$ b $\land$ b $\leq$ a zulässig. \\
	Zum Beispiel für 10,1 und 10,1. \\
	a $\leq$ a ist wahr \\
	a $\leq$ y $\leq$ b $\to$ a $\leq$ b \\
	Alle Bedingungen für Halbordnung erfüllt. \\
	$\frac{10}{2}$ $\leq$ 5 $\lor$ 5 $\leq$ $\frac{10}{2}$ Beispiel für alle a, b $\in$ R \\
	Ordnung.
	\subsection*{b)}
	M sei eine Menge. \\
	P(M) Sei die Potenzmenge von M und somit die Menge aller möglichen Teilmengen von M. \\
	P =  \{M, \{\}\} \\
	M $\subseteq$ \{\} $\land$ \{\} $\subseteq$ M \\
	y $\in$ P \\
	M $\subseteq$ y $\subseteq$ \{\} \\
	M $\subseteq$ \{\} \\
	M $\subseteq$ \{\} $\lor$ \{\} $\subseteq$ M \\
	
	\subsection*{c)}
	für =; \\
	a, b $\in$ N; \\
	a = b $\to$ b = a \\
	a = a \\
	y $\in$ N; \\
	a = y; y = b; a = b \\
	a = b $\equiv$ b = a \\
	Es handelt sich um eine Ordnung \\
	für $<$ \\
	a, b $\in$ N \\
	a $<$ b \\
	y $\in$ N \\
	a $<$ $<$ y $<$ b $\to$ a $<$ b \\
	a $<$ a ist nicht erfüllbar. \\
	Weder Halb- noch Ordnung \\
	für $\geq$; \\
	a, b $\in$ N \\
	$a \geq b$ $\land$ $b \geq a$ nicht erfüllt Beispiel 2 und 1. 2 $\ge$ 1 aber 1 $\not\geq$ 2.\\
	Keine Ordnung \\
	für x $\equiv_x$ y; x teilt y \\
	y teilt x aber x muss y nicht teilen. \\
	5 teilt 10, 10 aber nicht 5. Keine Antisymmetrie $\to$ keine Halb- oder Ordnung.
	
\end{document}