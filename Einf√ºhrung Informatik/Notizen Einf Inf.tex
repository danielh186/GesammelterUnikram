\documentclass{article}
\usepackage{mathtools}
\begin{document}
	\section*{Notizen Einführung in die Informatik}
	\section*{Roadmap zur Klausur}
	Wechsel der Zahlenbasis \\
	Hornerschema Division durch fallende Potenzen  \\
	Huffmancodierung Shanon-Farno-codierung \\
	CRC-Prüffelder \\
	Auf Sender- und Empfängerseite \\
	Wissensfragen quer durch das Modul \\
	Gleitpunktzahlen IEEE Umwandlung von und nach \\
	Multipalchoice querbet \\
	vollständige Linux-Comandos \\
	Schaltnetze Boolsche Funktionen \\
	KNF und DNF \\
	Hamming-Codes \\
	Codewort ermitteln oder Parität prüfen
	
	\section*{06.11.2020}
	\subsection*{Einführung}
	Organisation \\
	Methoden Softwareentwicklung, Mischform V und P. Ziel: Student soll konstruktive Fragen stellen können. Weiteres Wissen selbstständig erarbeiten. \\
	Von Neumann Rechner, Prinzip für meiste CPUs heute. Clusterung Gerätekunde.
	Keine Musterlösungen Material muss selbst erarbeitet werden. 
	Informatik: Systematische Darstellung, Verarbeitung und Übertragung von Informationen.
	Hot Swopping, Mainframes hohe Verlässigkeit. Abgrenzung Supercomputer. SC besteht aus vielen zusammengeschalteten Prozessoren. \\
	Kerninformatik: theorie, pratisch, technische \\
	Angewandte Informatik: Softwareentwicklung \\
	Theoreitsche Informatik Beispiel KI.
	Informatikstudium sollte sich auf Kerninformatik beziehen. Informatik besteht nicht nur aus Programmieren.
	\subsection*{Geschichte Informatik}
	Zahlsysteme älteste Zahlzeichen 4000 v.Chr. Ägypter hatten Zeichen für Zahlen. \\
	Frühe Stellenwertsysteme. Rechenbuch Indische Ziffern Adam Ries. Dualsystem basiert auf Leipnitz. Basiert auf 2-er Potenzen. 1550 Dezimalnotation und Logarythmus. Komplexe Rechnungen mit Logarythmus auf Addition etc zurückführen. Traum Mechanisierung von mühevollem Rechnen. Erste Konstruktion 1592. Problem Feinmechanik war nicht präzise genug. Mauser-Cordt 1929 erste kommerzielle elektrische Rechenmaschine. \\
	Erste Webstühle programmierbar 1728 Mit Löchern in Holzplatten. Difference Engine Charles Babbage. Analytical Engine Programmgestuerte Rechenmaschine. Weist alles auf was ein moderner Rechner hat. Lochkarten. 1833 \\
	1886 Automat Volkszählung. Ausgründung Wurzel der Firma IBM. 1924f
	
	\subsection*{20.11.2020}
	Visionen aus 1954. Vorstellung von Wissenschaftlern aus den 50 Jahren wie ein PC heute aussehen könnte \\
	Personal Computer \\
	\subparagraph*{Geschichte Computernetze}
	Netzentwicklung Sputnikschock \\
	USA wollten mit UDSSR nachziehen \\
	ARPANET zum Internet \\
	Dateitransfer, E-Mail, Remote-Login \\
	IETF Internet Ingeneering Taskforce \\
	RFC = Request for Command \\
	IETF veröffentlicht RFCs und basis der Meinung der Comunity wird der neue Standard etabliert. \\
	das WWW Protokoll World-Wide-Web \\
	Wiege des WWW in Zern  um wissenschaftliche Dokumente zu organisieren 1990 es entstand HTML\\
	1991 50 Webserver \\
	1995 Browserkrieg zwischen MS und Netscape \\
	Netscape verliert Browserkrieg \\
	1995 HTML CSS sollen WWW Standardisieren \\
	W3C Organisation für das WWW \\
	Aktuell nur graduell keine fundamentale Entwicklung \\
	Gesetz von Moore: Alle n Monate verdoppelt sich die Anzahl Transistorfunktionen auf der gleichen Fläche. 1965 \\
	Neuanwendungen von Rechenleistung verteilte Systeme, open-sourche, Heterogenität, Skalirbarkeit, Sicherheit, Echtzeitfähigkeit (garantierte Höchstdauer), Fehlertoleranz / Robustheit, Multimedia, WWW-Anbindung, Managebarkeit, \\
	\subsection*{Gesellschaftliche Auswirkungen}
	Informatik ist zwingend Teil des Alltags. \\
	Jeder nutzt Informatik und ist abhängig davon \\
	\subparagraph*{Grundbegriffe der Informatik}
	Information und Definition, Relationen, Funktionen, Graphen, Bäume, Algorithmen und Datensrukturen \\
	Partielle und totale Ordnung \\
	System \\
	Grenzen nach außen und innen \\
	Schnitstellen nach Außen \\
	Komponenten und ihre Beziehungen innerhalb des Systems \\
	Abstraktion erkennt durch Eigenschaften und Beziehungen eines Ausschnitts der realen Welt \\
	Modell als Ersatz für Realität \\
	Modell muss einfacher sein als Wirklichkeit \\
	billiger und oder sicherer \\
	Informatiker sind systemorientiert \\
	Gatter \\
	Register Transfer Sprache \\
	Bildet einen Prozessor der programmiert werden kann.
	Prozessorsystem \\
	Anwendungsprogramme \\
	Geschäftsprozesse \\
	Informatik ist die Wissenschaft der Informationsverarbeitung \\
	Information = Die Essenz von etwas \\
	Darstellung einer Information ist der einzige Kommunikationsweg \\
	Kommunikation nur über Darstellung \\
	Semantik = Informationsgehalt, \\
	Representation = Darstellung der Semantik \\
	Deutung \\
	Deutung kann mehrere Informationen haben. \\
	Verstehen Herstellung von Beziehung aus Repräsentation und Wirklichkeit \\
	
	\subsection*{formelle Sprachen}
	Zeichen, Zeichenvorrat \\
	Bit \\
	Bsp. 0 teil Zeichenvorrat 0, 1 \\
	Zeichen abstrakt wertneutral \\
	Symbol Zeichen mit Bedeutung \\
	Alphabet $\sigma$ Zeichenvorrat mit linearer Ordnung (Reihenfolge) z.B. das ABC \\
	Wörter sind endliche Folgen von Zeichen eines Alaphabets $\in$ $\sigma$ als Sequenz geschrieben. |w|=n für Länge des Worts, |w| kann auch 0 sein.\\
	$\sigma*$ Menge aller Wörter. \\
	$\sigma+$ Menge aller nicht lehrer Zeichenketten. \\
	$\sigma^{n}$ = alle Zeichenketten mit Länge n. \\
	Präfix Anfangsstück, Suffix Endstück \\
	Lexikographische Ordnung: suchen nach alphabetischer Ordnung Beispiel Lexikon. \\

	Alle $w \in \sigma^{*} : \epsilon <=_{lex} W$  \\
	
	lexikographische Ordnung definiert eine lineare Ordnung für ein Alphabet.
	
	\section*{27.11.2020}
	Bsp.: $\sum^*$ = (0,1) $\leq_{lex}$ 1 \\
	Mengenoperationen lassen sich auf formale Sprachen angewendet werden.
	Sprache L kann mit M geschnitten werden: L $\cap$ M \\
	Oder vereinigt werden: L $\cup$ M
	Sei A und B Zeichenvorräte ist ein code eine Abbildung c. \\
	Decodierung ist die Abbildung eines Codes injektiv. \\
	Zu Codezeichen wird Klarzeichen gesucht. \\
	Graphen und Bäume werden in Mathematik behandelt. \\
	
	\section*{Algorithmen}
	Verarbeitungsvorschrift müssen exakt festgelegt sein, formal definiert sein, durch einen Prozessor ausführbar sein. Darunter versteht man einen Algorithmus. \\
	Ein Algorithmus ist ein Verfahren in einer genau festgelegten Sprache zur Lösung gleichartiger Probleme. \\
	Algorithmus liefert für eine Eingabe immer eine Ausgabe. \\
	Problemklasse und Lösungsverfahren unterscheiden. \\
	Merkmale Algorithmus: \\
	terminiert nach n endlichen Schritten. \\
	deterministisch feste Ablaufschritte. \\
	determiniert Endergebnis ist immer eindeutig(korrekt). \\
	klassische Elemente: Wiederholung Rekursion arithmetische Operatoren Vergleichsoperatoren. \\
	Komplexität meint Abhängigkeit zwischen Aufwand und Anfangszustand. Z.B. Speicheraufwand, Rechenzeit. \\
	unified moduling language = UML \\
	Def. Syntax ist die zulässige Anordnung einer Sprache. \\
	Eine Programmiersprache ist eine formale Sprache zur Darstellung von Algorithmen. \\
	Ein in einer solchen Sprache dargestellten Algorithmus heist Programm. \\
	Def. Prozess Ausführung eines Programmes für ein bestimmtes Problem. \\
	\section*{Kap 3. Repräsentierung von Informationen in Rechensystemen}
	Def Daten sind Informationen die nach eindeutiggen Regeln in Rechensystemen gespeichert und verarbeitet werden. 
	Def. Nachrichten sind Daten für Übertragungen.\\
	Bitfolgen variabler länge Bits, Vektoren bestimmte Länge. \\
	1 byte ist eine Folge aus 8 Bit!!! \\
	byte ist die kleinste adressierbare Einheit in heutigen Arbeitsspeichern. \\
	fortlaufende Bytes sind durchnummeriert. Diese Nummern nennt man Adressen. \\
	Maschinenwörter meist Binärwörter weniger fester Längen. \\
	typisch 32 Bit Wortlänge bei 32-bit Prozessoren 64-bit Desktop. \\
	Wortlänge hängt von der Hardware ab. \\
	B KB MB GB TB PB
	Unterscheide Kilobyte und Kibibyte
	KB = 1000 Byte Kibibyte = 1024 Byte \\
	Auf Bitvektoren können boolsche Funktionen angewendet werden. \\
	Elementare Maschinenbefehle. Verschieben rotieren (links rechts) \\
	Bitoperatoren in Hochsprachen \\

	Summe der Potenzen der Basis \\
	Horner-Schema \\
	Q Darstellung R nur Aproximation \\
	Q setzt sich periodisch vor R nicht periodisch \\
	Umrechnung der Darstellung \\
	Division durch fallende Potenzen der Zielbasis. \\
	Horner-Schema Division durch B \\
	Konvertierung gebrochener Zahlen \\
	Darstellung ganzer Zahlen als Maschinenwort. \\
	64 bit = $2^{64}$ \\
	Bitmuster in Maschinenwort. \\
	Bitmuster (byte) werden unterschiedlich in Speicher geschrieben. Big-Endian, Little-Endian
	BE höchstes Byte als letztes. \\
	LE höchstes Byte als erstes. \\
	Vorzeichen-Betragsdarstellung \\
	Excess-Darstellung Bsp Analog-/Digitalwandler \\
	Komplement-Darstellung \\
	\section*{11.12.2020}
	Stellenkomplement \\
	Darstellung in n-Bit Maschinenwort \\
	symetrischer Bereich darstellbarer Zahlen \\
	doppelte Darstellung der 0 \\
	Einfache Arithmetik \\
	Zweierkomplementdarstellung \\
	Einerkomplementdarstellung \\
	Unterschied im Wertebereich beachten. \\
	unterschiedliche Wortlänge verursacht größeren Wertebereich. signed integer \\
	Speicherüberlauf alle Bits klappen um. \\
	Arithmetik in der Zweierkomplementdarstellung \\
	Subtraktion und Addition kann mit Addition dargestellt werden. \\
	Vorzeichenbit = Normale Stelle \\
	Bsp: 0110 + 0011 = (01)1001 \\
	(01) Übertragbits beachten um feststellen zu können ob Wertebereich verlassen wird. \\
	Wenn Übertragbits ungleich Indikator für Fehler. \\
	Multiplikation Komplementdarstellung \\
	Mit Addition und Multiplikation können alle Grundrechenarten abgebildet werden. \\
	Darstellung Festkommazahlen \\
	Festes Bitmuster bleibt, Komma wird nur gedacht und ist eine Vereinbarungssache. \\
	Probleme: Ungenaue Konvertierung, Rundungsfehler \\
	Dezimalarithmetik (BCD) als Antwort auf die Anforderung Buchhaltung auf den Cent genau. \\
	BCD Darstellung \\
	Dezimalzahl wird in 4-stelligen Dualzahlen dargestellt.
	Darstellung Ziffernweise, nur für 0 - 9. \\
	Rest Pseudotetrade. \\
	Darstellung von Gleitkommazahlen. \\
	Wird angenähert. Es werde nur die wichtigsten Stellen gezeigt. \\
	Komma wird zum Teil der Zahlendarstellung \\
	Gleitkommazahl x = m * $B^{e}$ \\
	m ist eine vorzeichenbehaftet Festkommazahl \\
	B ist die Basis des Systems typisch 2, 8, 16, 32 \\
	e ist ganzzahliger exponent (Bestimmt Größe der zahl) \\
	Mantisse m muss normalisiert werden. \\
	$\frac{1}{B}<|m|<1$ oder 1 $\leq |m| < B$
	Maschinenwort wird in S ch m unterteilt. \\
	Wahl der Grenzen bestimmmt Wertebereich und Genauigkeit \\
	Mehr Platz für Mantisse = mehr Genauigkeit \\
	Mehr Platz für Exponent (ch) = größerer Wertebereich \\
	Arithmetik dieser Darstellung lehnt an Potenzgesetze an. \\
	IEEE Gleitkommadarstellung \\
	Bitmuster alleine macht keine Darstellung \\
	\section*{18.12.2020}
	Buchstaben, Ziffern, Steuerzeichen = alphanumerischer Code \\
	heute UTF-8 und UTF-16 \\
	US-ASCII code \\
	PC-8 führt Sonderzeichen ein. \\
	Anfang der 90er Alle Zeichen der Welt sollen eigenen Code bekommen. \\
	Zeichenbereiche in U sind Skripte. \\
	Skripte können angesprochen werden auf Programmierebene. \\
	little oder bigg engine byte order mark \\
	Jede ASCII ist gültiger UNICODE gemäß UTF-8!!! \\
	Unicode kann in Ruby verwendet werden.
	Serielle Übertragung Zeittakt \\
	Parallele Übertragung nutzt mehrere Leitungen parallel. \\
	Signal zeitlicher Verlauf einer physikalischen Größe. \\
	Def. Rasterung: Abtastung eines Signals zu bestimmten Zeitpunkten (Intervallen). \\
	PCM \\
	\section*{08.01.2021}
	Codierung \\
	Komprimierung, Fehlererkennung -behebung etc, \\
	Codierung ist eine Abbildung über zwei Zeichenvorräte. \\
	Ziele von Codierung: Geheimhaltung, Sicherung gegen Fälschung, Effitienz \\
	Codebaum \\
	Def. Blockcodes = wörter fester Länge. \\
	Def. Dichter Code ist surjektiv. \\
	Gray-Code wechselt bei Zustandswechsel immer ein Bit. \\
	Blockcode kann zum Beispiel 7-Segmentanzeige steuern. \\
	Def. Farno-Bedingung Kein Codewort ist Präfix eines anderen Codeworts. \\
	Häufigkeitsabhängige Codierung \\
	Länge der Codwörter häufiger Zeichen soll möglichst kurz sein. \\
	Mittlere Codelänge. \\
	Codierungstheorem von Shanon: es gibt eine untere Grenze Mittlere Codelänge. \\
	\section*{15.01.2021}
	nicht verlustfreie codierung typisch Bild und Ton. \\
	Fehler erkennende Codes \\
	Fehler korrigierende Codes \\
	Def. Ein Bitfehler seines Signals ist seine Umkehrung. \\
	Fehler werden erkannt indem geprüft wird ob es sich um ein gültiges Codewort handelt. \\
	Fehler Eingabefehler, Übertragungsfehler (Burst-Fehler $\lor$ Bündel-Fehler), Speicherfehler. \\
	Fehlercodes während auf häufigste Fehler angepasst. \\
	möglich durch Hinzufügen von Redundanz. \\
	Hamming-Gewicht und Hamming-Abstand \\
	Hamming-Gewicht g(b) Anzahl der 1 im Code. \\
	Hammingdistanz zwischen AB ist die Anzal der Stellen an denen sich die Codewörter unterscheiden. \\
	Fehlerkorrektur problemlos möglich für einen Code mit Hammingabstand d bis d -1. \\
	Paritätsbit \\
	Fehlererkennung durch Paritätsprüfung \\
	Feherlcode beheben für d = 2k+1 können Störungen mit höchstens k-Bitfehler behoben werden. Beispiel d = 3 k = 1 \\
	Binärer Rechteck-Code \\
	Paritätsbit je Codewort und je Spalte. \\
	lineare codes symmetrische codes. \\
	Blockcode Länge n dan $2^m$ Codewörter mit m$\leq$n daraus folgt r = n - m Prüfstellen. \\
	Def. Hammingcode ist ein linearer symmetrischer code. \\
	Prüfbit an Stelle $2^{r-1}$ r wächst damit logarithmisch. \\
	\section*{22.01.2021}
	Optimalität Hamming-Code \\
	Satz: m+r+1 $\leq 2^r$ für Hammming-Distanz $\geq$3. \\
	SEC DED Codes Single Error Correction und Double Error Detection \\
	Error Correcting Code Memmory ECC Speicher \\
	Hamming-Code für k wörter wird in Tabelle zeilenweise geschrieben und spaltenweise übertragen. \\
	Spalten können mit Hamming-Code gesichert werden. \\
	\\
	Zyklische Codes \\
	Burst-Fehler = Folgen verfälschter Bitstellen \\
	n Bits eines Datenblocks B der Länge n werden als Koeffizienten eines Polynoms P(x) mit dem Grad n - 1 interpretiert. \\
	an M bits werden R Prüfbits angehängt. \\
	$M(x) * x^r$ Division durch Generator Polynom muss einen Rest mit dem Grad r-1. \\
	Empfänger prüft P(x) / G(x) muss R ergeben. Fehler wird erkannt indem der Rest R = 0 ist. \\
	Generatorpolynom ist Vereinbarung zwischen Sender und Empfänger. \\
	Wenn P(x) : G(x) ohne Rest teilbar ist sind alle Daten heil angekommen \\
	Je länger Burstfehler desto länger muss CRC Feld sein. \\
	Generatorpolynome in der Praxis standardisiert. \\
	CRC Feld mit Länge r kann r-1 Fehler erkennen.
	\\
	\\
	Kapitel 5 Schaltwerke und Schaltnetze \\
	Def. Eine Schaltfunktion wird durch eine Abbildung definiert: \\
	$f_s: \{0,1\}^n \to \{0,1\}^m$ \\
	zu Beginn m = 1. Dann bildet eine n stellige Boolsche-Funktion auf ein binäres Tupel ab. \\
	Boolsche-Funktion wird durch Wertetafel beschrieben. \\
	n-stellige Boolsche-Funktionen können durch verkettete 2-stellig Boolsche-Funktionen dargestellt werden. \\
	Boolsche-Algebra \\
	Dualitätsprinzip \\
	Beispiel Idempotenzgeset De Morgansche Regeln. \\
	Boolsche-Terme \\
	\section*{29.01.2021}
	Normalform boolsche Terme \\
	dnf und knf \\
	Minterm mit oder verknüpfen \\
	Maxterme mit und verknüpfen. \\
	Normalform soll elektronische Schaltungen vereinfachen. \\
	Komplexitätsmaße: Größe: Chipfläche und Tiefe: Klammerausdrücke
	dnf und knf sind nicht auf minimaler Hardware optimiert. \\
	Kann zum Beispiel mit KV-Diagramm erzeugt werden. \\
	Schaltnetze \\
	Graphische Repräsentierung für Schaltungen basierend auf boolscher Algebra. \\
	gerichteter zyklenfreier Graph. \\
	Gatter eingehende Kante und ausgehende Kante \\
	Gatter sind unär oder binär. \\
	2 stellige logische Gater. \\
	es gibt und oder nichtund nichtoder xor und equiv Gatter. \\ 
	NAND-Gatter ist universell. \\
	technische Realisierung von Gattern. \\
	And zwei Schalter gegeneinander in Serie \\
	Or zwei Schalter parallel. \\
	Gatter höherer Ordnung n-Oder-Gatter n-And-Gatter. \\
	\\
	praktische Beispiele für Schaltnetze. \\
	Übersicht:
	Tore, Encoder, Decoder, Multiplexer, Demultiplexer, Halbaddierer, Volladdierer, Arithmetisch-Logischen-Einheit (ALU) \\
	Tor, kontrollierte Durchleitung mit Steuersignal. \\
	Encoder 1 aus n code Übersetzung aus sperlichem in dichten Code. \\
	für jeden gewünschten Zustand hat der Decoder einen Ausgang. \\
	z.B. drei Ausgänge für 8 Inputsignale \\
	Decoder dekomprimiert code. \\
	Encoder und Decoder können Leitungen sparen. \\
	Multiplexer n Eingänge werden auf einen Ausgang, nacheinander (seriell) durchgeschaltet. Decoder wird für Steuersignal verwendet. \\ 
	Demultiplexer Gegnstück zum Multiplexer. \\
	Halbaddierer addiert zwei Bits. \\
	bei (1,1) wird Übertrag benötigt Carry Bit. \\
	Volladdierer: Berücksichtigt Übertrag aus vorheriger Stelle. \\
	Viele VAs ergeben hohe Tiefe und damit hohe Laufzeit. (Ripple Carry) \\
	Lösung: Carry-Look-Ahead. Alle Überträge werden vor dem Addieren berechnet. \\
	ALU kann undieren odieren addieren etc. \\
	Schaltwerke \\
	Graphen mit Rückkopplung um Speicherung zu ermöglichen. \\
	\section*{05.02.2021}
	Rechner Architektur \\
	Ziele universell oder Speziell \\
	Def. OS/BS Betriebssystem ist das grundlegendeste Systemprogramm. Verwaltet alle Systemprogramme offeriert eine Virtuelle Maschine. \\
	Abstraktion der Datei. \\
	Def. Resourcen: Alle zuteilbaren Hard- und Softwarekomponenten. \\
	Betriebssystem als VM \\
	Ziel: Schirmt vor Komplexität ab. OS richtet sich danach. \\
	Typische Abstraktionen: Prozess, Threads (Aktivitätsträger), Datei, Speichersegment, Nachrichten, Synchronisierungsobjekt \\
	OS Als Betriebsmittelverwalter \\
	OS Ordnet Zuteilung von Ressourcen. \\
	Vereinfachte Schichtung, Assembler als hardware nahe Sprache. \\
	Von-Neumann-Architektur \\
	Idee Programm soll im Hauptspeicher gehalten werden. \\
	Von-Neumann-Architektur prinzipielle bis heute. \\
	Ziel der Hardwareersparnis ist gefallen. \\
	Elemente: CPU (Rechenwerk, Steuerwerk). \\
	I/O, Speicherwerk. \\
	Speicherwerk enthält Maschinenprogramm. \\
	Elemente werden mit Bus verbunden. \\
	Control Unit liest Steuerbefehle. \\
	Rechenwerk liest Daten und schreibt Ergebnisse \\
	I/O gibt Ergebnisse aus und liest eingaben ein. \\
	Speicherwerk \\
	MAR = Memmory Adress Register \\
	Decodierung \\
	MDR = Memmory Data Register \\
	Rechenwerk enthält ALU, und Akkumulator \\
	ACC = Ergebnisregister \\
	Steuerwerk: Instruction Register IR, Programm Counter \\
	Ablaufsteuerung \\
	Ablauf: streng zyklisch und sequenziell. \\
	Von-Neumann-Flaschenhals \\
	Engpass Prozessor und Speicher \\
	häufige Transportbefehle. \\
	Bus zwischen Speicher und CPU macht Langsam. \\
	Lösung schnelle Zwischenspeicher an der CPU sogn. Caches. \\
	Leistungssteigerung durch Transistorgröße oder durch Anzahl der Transistoren. \\
	Erhöhung der Wortbreite eines Maschinenworts. \\
	Prozessorfamilie: Maschinenprogramme bleiben kompatibel zu früheren Mitgliedern einer Familie. \\
	Speicherarten: RAM (Random Acces Memmory); ROm(read only); EPROM(eraseable programmable ROM); \\
	EEPROM electric EPROM vorreiter für flash Speicher. \\
	S(tatic)RAM speicher aus Flip Flops schnell und teuer \\
	D(ynamic)RAM Basis Kondensatoren \\
	zerstörendes Lesen muss durch anschließendes Schreiben ausgeglichen werden. \\
	Kondensatoren verlieren durch Leckstrom speicher. \\
	refresh erforderlich in deutlich unter 100 Nanosekunden. \\
	DRAM Grundlage der Hauptspeicher. \\
	SSDs nutzen Kondensatoren reduzieren aber Leckstrom. \\
	Kondensatoren können mehrere Spannungsstufen unterscheiden. \\
	Interleaving \\
	multible Nutzung von Speicherbänken.(RAM immer im dual Channel) \\
	\section*{12.02.2021}
	Steuerwerke \\
	Steuerwort \\
	Jedes Maschinenwort braucht Operationscode \\
	Steuerwerk im Steuerwerk \\
	Mikroprogrammiertes Steuerwerk \\
	Sequencer legt nächsten Befehl fest \\
	Microprogramme im Steuerwerk \\
	Leistungsteigerung CPU \\
	Ausstattung Spezialeiheiten Koprozessoren. \\
	Pipelining, Vervielfältigung von Modulen \\
	CPU/Speicher Schnittstelle \\
	Prefetching \\
	Caching \\
	Pipelining: Mehrere Spezialisten bearbeiten Parallel ein Problem in Teilaufgaben. \\
	Pipeline haben Nachlauf und Hochlauf. \\
	Besonders problematisch Sprungbefehl \\
	Superskalarität \\
	CPU Verfügt über mehrere Verarbeitungseinheiten \\
	Caches \\
	L1 schnell und klein L1 Cache hat einen Data und einen Code Cache halten CPU Takt mit \\
	L2 Cache KiB, langsamer als L1 \\
	L3 nimmt Daten auf\\
	Swapping Schieben von Daten aus Hauptspeicher Sekundarspeicher. \\ 
	CISC / RISC \\
	CISC möglichst viele komplexe Befehle in Hardware umsetzen \\
	RISC Reduziertes Set von Anweisungen. Verschlankung der CPU \\
	Spekulative Ausführung \\
	Ausführung eines Bedingten Sprunges bevor der Befehl kommt. \\
	CISC/RISC-Grenzen verwischen \\
	SIMD-Technik, Vektorverarbeitung \\
	Systemarchitektur \\
	Bisher eine CPU und jetzt ganzes System. \\
	Kopplung mehrerer Prozessoren \\
	Enge Kopplung bei geteiltem Speicher, lose Kopplung Nachrichtenaustausch \\
	hommogene Systeme haben identische Komponenten (mindestens binärkompatibel und gleich schnell) \\
	symmetrische Systeme Alle Komponenten haben die gleiche Rolle und sind austauschbar. \\
	SI / MI single instruction / Multi instruction\\
	SD / MD single data / multi data \\
	können zu vier Fälle kombiniert werden \\
	Multiprozessorsystem / Multicomputersystem \\
	Multiprozessorsystem bus oder switchbasiert \\
	bus eine line. Switch: Kreuzschienenmatix  mehrere Schalter schalten mehrere Leitungen durch. Zugriffe können parallel erfolgen. \\
	(Crossbar Switch) hoher Hardwareaufwand, schlechte Skalierbarkeit. \\
	hirachischer Bus \\
	nutzt Kommunikationsprozessor für gelegentliche Fernzugriffe. gute Skallierbarkeit. \\
	Grundlage für Supercomputer \\
	Stichwort NUMA \\
	\section*{19.02.2021}
	Multiprozessorsysteme \\
	Maschinen werden auf die Klasse zu lösender Probleme angepasst. \\
	Multicomputersysteme \\
	Koppelung lose über message passing \\
	Caching ist reduziert auf lokale Rechner \\
	konzentration auf leicht parallelisierbare Aufgaben  \\
	Netzwerkgitter-Topologie \\
	Netzwerk mit n dimensonalen Würfel alss Topologie = Hypercube. \\
	n Kommunikationswege pro Knoten \\
	längster Weg wächst logaritthmisch mit der Anzahl der Knoten. \\
	
\end{document}