\documentclass{article}
\begin{document}
	\section*{Lsg Vorschlag RuT Ü04 Maximilian Maag}
	\subsection*{Aufgabe 4.1}
	Durch Schnelle Umwandlung
	\subsection*{a)}
	192.168.1.2 \\
	11000000 10100100 00000001 00000010
	\subsection*{b)}
	10.0.123.7 \\
	00001010 00000000 01111011 00000111
	
	\subsection*{Aufgabe 4.2}
	\subsection*{c)}
	Subnetzmaske ist aufgebraucht.
	\subsection*{d)}
	Diese Adresse ist ungültig, da sie nur 3 Adressblöcke enthält.
	\subsection*{e)}
	hat keinen Host, da IP-adressen nie görßer als 255 sind.
	\subsection*{Aufgabe 4.3}
	\subsection*{a)}
	13.34.254.0/23 \\
	13.34.11111110 00000000 \\
	Netzmaske: 255.255.254.0 \\
	\subsection*{b)}
	20.30.0.0/15 \\
	20.11111110 00000000 00000000 \\
	255.254.0.0
	\subsection*{Aufgabe 4.4}
	\subsection*{a)}
	128 kann mit einem Bit an 8. Stelel dargestellt werden. \\
	21.17.128.0/17
	\subsection*{b)}
	Für die 192 sind sowohl das 8. und das 7. Bit gesetzt. \\
	194.99.17.96/26
	\subsection*{Aufgabe 4.5}
	Bei der Broadcast-Adresse besteht der Hostteil nur aus Einsen. Daraus folgt: \\
	IA: IP-Adresse \\
	NM: Netzmaske \\
	BC: Broadcast-Adresse
	\subsection*{3a)}
	12.34.254.0/23 \\
	IA: 00001100 00100010 1111111-0 00000000 \\
	NM: 11111111 11111111 11111110 00000000 \\
	BC: 00001100 00100010 1111111-1 11111111 \\
	BC als Dezimal: 12.34.255.255
	\subsection*{3b)}
	20.30.0.0/15 \\
	IA: 00010100 0001111-0 00000000 00000000 \\
	BC: 00010100 0001111-1 11111111 11111111 \\
	BC dezimal: 20.31.255.255 
	\subsection*{4a)}
	21.17.128.0/17 \\
	IA: 21.17. 1-0000000 00000000 \\
	BC: 21.17. 1-1111111 11111111 \\
	BC dezimal: 21.17.255.255
	\subsection*{4b)}
	194.99.17.96/26 \\
	IA: 194.99.17 01-100000 \\
	BC: 194.99.17 01-111111 \\
	BC dezimal: 194.99.17.127
	\subsection*{Aufgabe 4.6}
	\subsection*{a)}
	172.30 0 001000 .... \\
	172.30.0.0
	\subsection*{b)}
	10. 000101 00... 10.20.0.0
	\subsection*{c)}
	20.15.23.245/25 \\
	20.15.23 1... 20.15.23.128
	\subsection*{Aufgabe 4.7}
	\subsection*{a)}
	172.30.24.0 \\
	10101100 00011110 00011000 0 0000000 \\
	10101100 00011110 00011000 1 0000000 \\
	172.30.24.0 \\
	172.30.24.128 \\
	\subsection*{b)}
	10.20.240.0 \\
	00001010 00010100 111100 00 00000000 \\
	00001010 00010100 111100 10 00000000 \\
	00001010 00010100 111100 01 00000000 \\
	00001010 00010100 111100 11 00000000 \\
	10.20.240.0 \\
	10.20.244.0 \\
	10.20.248.0 \\
	10.20.252.0 \\
	
	\subsection*{c)}
	172.30.24.127 \\
	172.30.24.255
	\\ \\
	10.20.240.255 \\
	10.20.244.255 \\
	10.20.248.255 \\
	10.20.252.255 \\
	\subsection*{Aufgabe 4.8}
	Netzmaske: 11111111 11111111 11110000 00000000 \\
	Netzadresse: 00001010 000010100 11000000 00000000
	\subsection*{c)}
	10.20.241.12 \\
	Netzadresse: 00001010 00010100 11110001 00001100 \\
	Netzmaske: 11111111 11111111 11110000 00000000 \\
	00001010 00010100 11110001 00000000 \\
	Ungleich Netzadresse daher nicht enthalten. 
	\subsection*{d)}
	10.20.100.13 \\
	00001010 00010100 01100100 00001101 \\
	11111111 11111111 11110000 00000000 \\
	00001010 00010100 01100000 00000000 \\
	nicht enthalten.
	\subsection*{e)}
	10.21.200.14 \\
	00001010 00010101 11001000 00001110 \\
	11111111 11111111 11110000 00000000 \\
	00001010 00010101 11000000 00000000 \\
	Adresse ist nicht enthalten.
	\subsection*{f)}
	10.20.200.15 \\
	00001010 00010100 11001000 00001111 \\
	11111111 11111111 11110000 00000000 \\
	00001010 00010100 11000000 00000000 \\
	Ist enthalten.
	\subsection*{g)}
	10.20.191.16 \\
	00001010 00010101 10111111 00010000 \\
	11111111 11111111 11110000 00000000 \\
	00001010 00010101 10110000 00000000 \\ 
	nicht enthalten.
	\subsection*{Aufgabe 4.9}
	\subsection*{a)}
	gw 10.10.10.20 eth0
	\subsection*{b)}
	gw 10.10.20.140 eth1
	\subsection*{c)}
	sl0
	\subsection*{d)}
	gw 10.10.20.140 eth1
	\subsection*{e)}
	gw 10.10.10.20 eth0
	
\end{document}
