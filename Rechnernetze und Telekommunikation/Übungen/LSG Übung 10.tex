\documentclass{article}
\begin{document}
	\section*{Lsg Vorschlag RUTÜ10 Maximilian Maagg}
	\subsection*{Aufgabe 10.1}
	\subsection*{a)}
	Classic-Stun, H.263, SIP, IGMPv2, RTP SDP
	\subsection*{b)}
	208.97.25.11; 208.97.25.12;
	\subsection*{c)}
	Asterisk PBX
	\subsection*{d)}
	PCMA; PCMU; Telefonevent
	\subsection*{e)}
	g.711(PCMU)
	\subsection*{f)}
	33 Sekunden
	\subsection*{g)}
	Leichte Intensitätsschwankung bei der Übertragung. \\
	Bei empfangenen Paketen 13,77 ms
	\subsection*{h)}
	\subsection*{i)}
	Übertragung ist nicht verschlüsselt.
	Herr Gergeleit führt einen Echotest durch.
	\subsection*{j)}
	Die Protokolle dienen dem Verbindungsaufbau über das NATing hinweg.i
	\subsection*{Aufgabe 10.2}
	\subsection*{a)}
	sipload.de
	\subsection*{b)}
	Diverse Prepaid und Staffeltarife
	\subsection*{c)}
	1. Voll digital und Standortunabhängig \\
	2. Keine Bindung an einen lokalen Provider \\
	3. Neben einem PC ist kein zusätzliches Device erforderlich
\end{document}