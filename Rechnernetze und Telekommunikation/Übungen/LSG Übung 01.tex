\documentclass{article}
\usepackage{hyperref}
\hypersetup{
	colorlinks=true,
	linkcolor=blue,
	filecolor=magenta,      
	urlcolor=cyan,
}
\begin{document}
	\section*{Lsg Vorschlag RuT Ü01 Maximilian Maag}
	\subsection*{Aufgabe 1.1}
	\subsection*{a)}
	Router: \\
	Router sind teil eines Netzwerks und leiten Pakte zwischen einzelnen Teilnehmern des Netzwerks oder zwischen Netzen weiter. \\
	Router sind Schicht 3 aus dem OSI-Modell zuzuordnen, da sie im Kern mit der Transportkontrolle betraut sind.
	\subsection*{b)}
	FTP: \\
	Das File Transport Protocoll ist ein Protokoll, dass den Dateiaustausch zwischen Rechnern im Internet. \\
	FTP baut eine Verbindung auf, erfasst Daten und kann in Anwendungen verwendet werden. Es ist meiner Ansicht nach den Schichten 5 - 7 zuzuordnen.
	\subsection*{c)}
	DSL: \\
	Die Digital Subscriber Line bezeichnet Übertragungsstandards der 1. ISO OSI Schicht.
	\subsection*{d)}
	VOIP:
	Das Voice Over IP Protokoll ist ein Protokoll um über das Internet zu telefonieren. Es kann eine Verbindung zwischen zwei Teilnehmern IP-basiert aufbauen und Sprache übertragen. \\
	Es ist der 5. Schicht zuzuordnen da hier zwischen Endgeräten Datenpakte ausgetauscht werden.
	\subsection*{e)}
	HTTP(s): \\
	HTTP bzw. seine Variante mit SSL-Verschlüsselung HTTPS. Stellt eine einfache Auszeichnungssprache für Internetseiten dar. \\
	HTTP ist der 7. ISO OSI Schicht zuzuordnen.
	\subsection*{f)}
	Security: \\
	Sicherheit ist ein Konzept um Rechner und Netzwerke zu schützen. \\
	Sicherheitsrelevante Elemente gibt es in allen Schichten des ISO OSI Modells.
	\subsection*{Aufgabe 1.2}
	\subsection*{a)}
	Wlan: WLA - Wirerless Lan Association
	\subsection*{b)}
	5G: ITU - International Telecommunication Union
	\subsection*{c)}
	Bluetooth: Bluetooth special interessed group
	\subsection*{d)}
	IMAP: IETF
	\subsection*{e)}
	CSS: W3C
	\subsection*{Aufgabe 1.3}
	\subsection*{a)}
	Netzneutralität: \\
	Pro: Ein neutrales Netz ermöglicht allen Teilnehmern die gleiche Chance zur Teilhabe. Dienstanbieter müssen bei Providern keine Bevorzugung erkaufen. Die Gewährleistung von kritischem Traffic kann auch durch das Verbessern der Infrastruktur ermöglicht werden.\\
	Con: Ein prioitätsgestteuertes  Netzwerk ermöglicht zuverlässigen Traffic kritischer Echtzeitdaten. \\
	Beispiel: Der Traffic für die Kommunikation autonomer Fahrzeuge sollte bevorzugt werden, damit er auf jeden Fall rechtzeitig übertragen wird.
	\href{https://www.spiegel.de/netzwelt/netzpolitik/eu-leitlinie-zur-netzneutralitaet-deshalb-ist-die-verordnung-so-wichtig-a-1109899.html}{Artikel zum Thema}
	\subsection*{b)}
	Datenschutz: \\
	Pro: Anonymität ermöglicht freien Meinungsaustausch, ohne Angst davor haben zu müssen eine "falsche" Meinung zu haben. \\
	Con: Anonymität schafft ungesundes Selbstvertrauen. Im Schutze der Anonymität werden Beleidigungen geäußert oder Straftaten begangen. \\
	Sample: Das NetzDG sollte Abhilfe schaffen und immer wieder wird von Seiten der Politik die Forderung nach einer Klarnamenpflicht laut, in deren Rahmen sich Nutzer mit ihrem echten Vor- und Nachnamen bei Onlineplattformen registrieren sollen. \\
	Exemplarisch ein Artikel mit der entsprechenden Forderung aus Niedersachsen. \\
	Diese \href{https://www.deutschlandfunknova.de/beitrag/identifizierungspflicht-fuer-weniger-anonymitaet-auf-facebook-instagram-twitch-und-co}{Forderung} wirft seit Ihrem Bestehen die Bedenken der Datenschützer auf den Plan. Mit bestätigtem Klarnamen und Adresse lässt sich von Seiten diverser Social-Media betreiber viel leicht ein Werbeprofil erstellen als bisher. Schutzinteressen der Nutzer spielen in der Diskussion hier leider oft eine nachgeordnete Rolle.
	\subsection*{c)}
	Internetzensur: \\
	Pro: Ein zensiertes, besonders staatlich zensiertes, Internet hindert die Innovationskraft und den Austauschgedanken des Internets. Der schnelle ungefilterte und unkomplizierte Austausch hat viele nennenswerte Projekte der Modderszene oder Social-Media hervorgebracht. \\
	Menschen auf der ganzen Welt waren noch nie so vernetzt wie jetzt. \\
	Con: Das Internet vergisst nichts, garnichts. Einmal hochgeladen bleiben Inhalte in der Regel auch verfügbar. Dabei spielt es keine Rolle ob es sich um Katzenvideos, kritische Berichte oder Kinderpornos handelt. \\
	Sample: Die allseits bekannte Kontroverse über das Overblocking durch Uploadfilter bezüglich der EU-Urheberrechtsreform wurden heiß diskutiert. Die deutsche Auskopplung dieser EU-Richtlinie befindet sich in Deutschland aktuell im Gesetzgebungsverfahren. In der Hochphase dieser Diskussion hat unter anderem Golem.de darüber \href{https://www.golem.de/news/urheberrechtsdebatte-eine-einzige-piratin-hat-die-debatte-versaut-2002-146496.html}{berichtet}. \\
	Bekannt wurde die ursprünglich im Geheimen geführte Debatte durch eine Abgeordnete der Piratenpartei im EU-Parlament. \\
	Geprägt war die öffentliche Debatte von wenig Sachlichkeit, insbesondere seitens der Politik, und persönlichen öffentlichen Verunglimpfungen gegen den bekannten Gegner der Reform, Christian Solmeke und gipfelte in Morddrohungen gegen den Urheber der Reform, Axel Voss. Auf polnischen Antrag befasst sich auch der Europäische Gerichtshof mit diesem Thema.
	\subsection*{d)}
	Sharing-Kultur: \\
	Pro: Freier Zugriff auf Informationen fördert Innovationen. So bietet beispielsweise Github eine tolle Möglichkeit im gemeinsamen Austausch neue Software zu entwickeln. \\
	Con: Nicht alle Inhalte sollten ohne Erlaubnis geteilt werden. \\
	Sample: Siehe obige mit Zensurvorwurf behaftete Urheberrechtsreform. \\
\end{document}