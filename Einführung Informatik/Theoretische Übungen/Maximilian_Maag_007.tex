\documentclass{article}
%Losnummer 15
\begin{document}
	\section*{Lsg Vorschlag E I Ü007 Maximilian Maag}
	\section*{Aufgabe 7.1}
	\subsection*{a}
	Unterschiedene werden in der Regel folgende Rechnergenerationen: \\
	\begin{itemize}
		\item Direkte Programmierung, kein Betriebssystem.
		\item Stapelverarbeitung, Auftrag wird aus Programm, Daten und Steueranweisung erfasst und ausgeführt Benutzer erhält Ergebniss.
		\item Dialogverarbeitung, Benutzer kann Programme mithilfe von Bildschirm und Tastatur mit Rechner kommunizieren.
		\item Dialogsystem, Dialog zunächst mit Text dann grafisch.
	\end{itemize}
	\subsection*{b}
	Im Time-Sharing-Betrieb wird einem Prozess in Abhängigkeit des Verteilungsmodells Prozessorleistung für eine Bestimmte Zeit zur Verfügung gestellt.
	\subsection*{c}
	Steve Jobs $\to$ Apple Computer Apple I und II \\
	Bill Gates $\to$ Windows Betriebssystem \\
	Marc Anreessen $\to$ MosaiC Browser
	\subsection*{d}
	Kahn und Cerf erfanden die grundlegenden Verbindungsprotokolle IP und TCP. Sie gelten als Väter des Internets.
	
	\section*{Aufgabe 7.2}
	\subsection*{a}
	Das Gesetz von Moore geht davon aus, dass sich auf einem integriertem Schaltkreis fester Größe die Anzahl der Transistoren alle 2 Jahre verdoppelt. Daraus wurde später die Interpretiert, dass sich die Leistung von Prozessoren alle 2 Jahre verdoppelt.
	\subsection*{b}
	WWW entsteht 1989 in Genf mit dem Ziel wissenschaftliche Artikel mit einander teilen zu können. 
	\subsection*{c}
	W3C, IET, Linux Foundation 
	\subsection*{d}
	FSF = freie softwarenutzung freie Verbreitung. \\
	OSI = Benutzung gerne Verbreitung von Varianten ungern.
	\section*{Aufgabe 7.3}
	\subsection*{a}
	Information ist ein semantischer Bedeutungsbegriff und abstrakt. Informationen können nie vollständig verstanden werden um sie besser verstehen zu können werden sie unterschiedlich dargestellt bzw. konkretisiert, Repräsentiert.
	\subsection*{b}
	Bauplan für den Architekt mit allen Details über Statik, Materialverbrauch etc. Werbebild/Modell für die zukünftigen Besitzer/Auftraggeber im Reihenhauskatalog.
	
	\section*{Aufgabe 7.4}
	\subsection*{a)}
	\{0, 0000, 001, 01, 01, 0100, 0101, 01011, 011, 1, 10, 100, 110, 11000, 111, 1110, 11100\}
	\subsection*{b)}
	\{$\epsilon$, 0, 01, 013, 014, 05, 051, 053, 0530, 146, 20111, 203, 3, 310, 3112, 32, 5, 777\}
	\section*{Aufgabe 7.5}
	\subsection*{a)}
	$R_1 = \{0101, 0110, 0111, 1001, 1010, 1011, 1101, 1110, 1111\}$
	\subsection*{b)}
	$R_2 = \{00100101, 00111001, 11000101, 11011001\}$
	\subsection*{c)}
	$R_3 = \{110110\}$
	\subsection*{d)}
	Leere Menge nach Definition
	\section*{Aufgabe 7.6}
	\subsection*{a)}
	ggT(144,54) \\
	ggT(90, 54) \\
	ggT(36, 54) \\
	ggT(36, 18) \\
	ggT(18, 18) \\
	ggT(18, 18) = 18
	Der Algorithmus terminiert nach dem 5. Durchlauf.
	\subsection*{b)}
	ggT(-9,-9) Terminiert nach Schritt 1. \\
	ggT(-6,-9) \\
	ggT(3,-9) \\
	ggT(-6,-9) terminiert nicht. \\
	ggT(-6, 9) \\
	ggT(-6, 15) \\
	ggT(-6, 21) terminiert nicht. 
	\section*{Aufgabe 7.7}
	Antwort lautet c.
\end{document}